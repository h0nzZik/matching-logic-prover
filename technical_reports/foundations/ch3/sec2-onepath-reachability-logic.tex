\section{One-path reachability logic}

We consider one-path reachability logic
as introduced in~\cite{SPY+2016}.
This section is a very short but comprehensive presentation
about how to define reachability logic in matching logic.

Reachability logic is parametric on three components:
(1) a many-sorted signature $(S,\Sigma)$ which contains a distinguished sort
$\Cfg$,
(2) a sort-wise countably infinite set $\Var$ of variables,
and (3) a partial $\Sigma$-algebra $\TT$.
Validity and provability of reachability logic are denoted as
$\vDashRL$ and $\vdashRL$ respectively.

\subsection{Reachability logic as a matching logic theory}

We define a matching logic theory $\MLRL$ that faithfully captures
reachability logic.

Sorts in $\MLRL$ contains all that in $S$, including the distinguished sort
$\Cfg$.

Symbols in $\MLRL$ contains all functions and partial functions in $\Sigma$.
In addition, it contains a unary symbol $\snext \in \Sigma_{\Cfg, \Cfg}$.

Axioms in $\MLRL$ contains all equations $t_1 = t_2$ that are valid in
the algebra $\TT$.

\begin{notation}
We adopt the following notations in $\MLRL$.
\begin{align*}
\snext \varphi &\equiv \snext(\varphi) \\
\wnext \varphi &\equiv \neg \snext \neg \varphi \\
\
\varphi \Toonepath \psi &\equiv 
\end{align*}
\end{notation}

Given these three components, we can define
a matching logic theory $\MLRL$.
$\MLRL$ has the same variable set $\Var$ and the sort set $S$.
It contains all symbols in $\Sigma$ plus a unary symbol 
$\snext \in \Sigma_{\Cfg}$

The matching logic theory contains
The matching logic theory of one-path reachability logic contains
a unary symbol $\snext$. 