\section{Untyped lambda calculus}

\subsection{An overview}

Untyped lambda calculus, initially proposed by Alonzo Church in the 1930s,
captures computation using function abstraction and application.
The syntax of untyped lambda calculus is simple,
except the binding behavior and variable-capture-free substitution,
which we already saw when introducing the matching logic syntax,
so we assume readers are familiar with it.
In short, alpha-conversion is assumed, and two lambda terms 
(defined below) are regarded as \emph{the same term} 
if they are alpha-equivalent.
Given and fix a countably infinite set of variables.
A lambda term is either a variable,
or an application $e e'$ where $e$ and $e'$ are lambda terms,
or a function abstraction $\lambda x . e$ 
where $x$ is a variable and $e$ is a lambda term.
We abbreviate $\lambda x_1 \dots \lambda x_n . e$ as
$\lambda x_1 \dots x_n . e$.
The scope of $\lambda$ goes as far as possible to the right,
so $\lambda x y . x y$, for example, means $\lambda x y . (x y)$.
The set of all lambda terms is $\Lambda$.

The famous beta-reduction axiom, as shown below,
gives the intuitive meaning of functions. 
\begin{equation}
\label{eq:beta}
\tag{$\beta$}
(\lambda x . e) e' = e[e'/x].
\end{equation}
We leave two remarks.
Firstly, \eqref{eq:beta} is an axiom schema
where $x$ is any lambda calculus variable
and $e, e'$ are any lambda terms,
so there are infinitely many instances of it.
Secondly, equality ($=$) is not part of the syntax of the calculus,
and one should think of \eqref{eq:beta}
defining a binary relation on $\Lambda$, the set of all lambda terms.

We are interested in those binary relations on $\Lambda$
that are congruence relations with respect to
function abstraction and application.
We call them \emph{lambda theories}, if these congruence relations
contain all instances of \eqref{eq:beta}.
The set of all lambda theories is $\lambdaT$,
which is a nonempty set, 
as the total relation $\Lambda^2$, a trivial lambda theory, is in it.
We call $\Lambda^2$
an inconsistent lambda theory, because it makes any two lambda terms equal.
Any lambda theories that are not $\Lambda^2$ are called consistent theories.
One can show that the intersection of arbitrarily many lambda theories
is also a lambda theory.
Denote the intersection of all lambda theories, i.e. the smallest lambda theory,
as $\lambdabeta$.

The set of all lambda theories $\lambdaT$ forms a complete lattice.