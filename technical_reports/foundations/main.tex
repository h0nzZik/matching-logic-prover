\documentclass{amsart}

%\includeonly{}

% Use new TX fonts for texts and maths.
% \usepackage{newtxtext}

\usepackage{amssymb, mathtools}

% Use mathbb fonts on greek letters.
\usepackage{mathbbol}

% Use smart space
\usepackage{xspace}

\newtheorem{theorem}{Theorem}[section]
\newtheorem{lemma}[theorem]{Lemma}

\theoremstyle{definition}
\newtheorem{definition}[theorem]{Definition}
\newtheorem{example}[theorem]{Example}
\newtheorem{xca}[theorem]{Exercise}

\theoremstyle{remark}
\newtheorem{remark}[theorem]{Remark}

\numberwithin{equation}{section}



% General

\newcommand{\set}[1]{\{ #1 \}}

\newcommand{\doubleslash}{//\xspace}

% Matching logic section.

\newcommand{\sig}{\mathbb{\Sigma}}

\newcommand{\Var}{\textsc{Var}}
\newcommand{\Pattern}{\textsc{Pattern}}

\newcommand{\cln}{\mathbin{\!:\!}}








\newcommand{\K}{K\xspace}
\newcommand{\imp}{\to}
\newcommand{\dimp}{\leftrightarrow}
\newcommand{\To}{\Rightarrow}
\newcommand{\Name}{\textnormal{\textsc{Name}}}
\newcommand{\SVar}{\textnormal{\textsc{SVar}}}
\newcommand{\Nom}{\textnormal{\textsc{Nom}}}
\newcommand{\ddd}{,\dots,}

\newcommand{\FV}{\mathit{FV}}
\newcommand{\CC}{\mathcal{C}}
\newcommand{\MM}{\mathcal{M}}
\newcommand{\WW}{\mathcal{W}}
\newcommand{\YY}{\mathcal{Y}}
\newcommand{\II}{\mathcal{I}}
\newcommand{\interp}[1]{\__{#1}}
\newcommand{\interpM}{\interp{\MM}}
\newcommand{\interpY}{\interp{\YY}}
\newcommand{\interpW}{\interp{\WW}}
\newcommand{\sigmaM}{{\sigma_{\MM}}}
\newcommand{\sigmaW}{{\sigma_{\WW}}}
\newcommand{\sigmaY}{{\sigma_{\YY}}}
\newcommand{\SigmaSub}[1]{\Sigma_{#1}}
\newcommand{\SetOF}[1]{\{ #1 \}}
\newcommand{\pset}[1]{2^{#1}}
\newcommand{\rhox}[1]{{\rho_{#1}}}
\newcommand{\rhop}{\rho'}
\newcommand{\rhopx}[1]{{\rhop_{#1}}}
\newcommand{\barrho}{\bar{\rho}}
\newcommand{\widebar}[1]{\overline{#1}}
\newcommand{\barrhop}{\widebar{\rhop}}
\newcommand{\barrhox}[1]{\widebar{\rhox{#1}}}
\newcommand{\barrhopx}[1]{\widebar{\rhopx{#1}}}
\newcommand{\simon}[1]{\overset{#1}{\sim}}
\newcommand{\simx}{\simon{x}}
\newcommand{\simz}{\simon{z}}
\newcommand{\txtand}{\text{ and }}
\newcommand{\MLLambda}{\mathsf{\Lambda}}
\newcommand{\MLMu}{\mathsf{Mu}}
\newcommand{\MLLTL}{\mathsf{LTL}}
\newcommand{\MLCTL}{\mathsf{CTL}}
\newcommand{\MLSfive}{\mathsf{S5}}
\newcommand{\MLSL}{\mathsf{SL}}
\newcommand{\MLhybridML}{\mathsf{hybridML}}
\newcommand{\MLpolyadicML}{\mathsf{polyML}}
\newcommand{\MLTS}{\textnormal{\textsf{TS}}\xspace}
\newcommand{\MLUTS}{\textnormal{\textsf{UTS}}\xspace}
\newcommand{\MLinfLTL}{\mathsf{infLTL}}
\newcommand{\MLfinLTL}{\mathsf{finLTL}}
\newcommand{\MLPDL}{\mathsf{PDL}}
\newcommand{\MLuRL}{\mathsf{uRL}}
\newcommand{\MLRL}{\mathsf{RL}}
\newcommand{\hybridModalLogic}{\textit{hybrid}}
\newcommand{\polyadicModalLogic}{\textit{polyadic}}
\DeclarePairedDelimiter{\ceil}{\lceil}{\rceil}
\DeclarePairedDelimiter{\floor}{\lfloor}{\rfloor}
\newcommand{\infLTL}{\mathrm{infLTL}}
\newcommand{\finLTL}{\mathrm{finLTL}}
\newcommand{\uRL}{\textsc{uRL}}
\newcommand{\SSS}{\mathcal{S}}

% name of the proof rules
\newcommand{\prule}[1]{\textsc{(#1)}}

\newcommand{\modusponens}{\prule{Modus Ponens}\xspace}
\newcommand{\universalgeneralization}{\prule{Universal Generalization}\xspace}
\newcommand{\necessitation}{\prule{Necessitation}\xspace}
\newcommand{\existence}{\prule{Existence}\xspace}
\newcommand{\singletonvariable}{\prule{Singleton Variable}\xspace}
\newcommand{\propagationbottom}{\prule{Propagation$_\bot$}\xspace}
\newcommand{\propagationvee}{\prule{Propagation$_\vee$}\xspace}
\newcommand{\propagationexists}{\prule{Propagation$_\exists$}\xspace}
\newcommand{\variablesubstitution}{\prule{Variable Substitution}\xspace}
\newcommand{\framing}{\prule{Framing}\xspace}
\newcommand{\propositionaltautology}{\prule{Propositional Tautology}\xspace}
\newcommand{\forallrule}{\prule{$\forall$}\xspace}
\newcommand{\membership}{\prule{Membership}\xspace}
\newcommand{\membershipintroduction}{\prule{Membership Introduction}\xspace}
\newcommand{\membershipelimination}{\prule{Membership Elimination}\xspace}
\newcommand{\membershipneg}{\prule{Membership$_\neg$}\xspace}
\newcommand{\membershipwedge}{\prule{Membership$_\wedge$}\xspace}
\newcommand{\membershipexists}{\prule{Membership$_\exists$}\xspace}
\newcommand{\equalityelimination}{\prule{Equality Elimination}\xspace}
\newcommand{\membershipsymbol}{\prule{Membership Symbol}\xspace}
\newcommand{\membershipvariable}{\prule{Membership Variable}\xspace}
\newcommand{\functionalsubstitution}{\prule{Functional Substitution}\xspace}
\newcommand{\circularity}{\prule{Circularity}\xspace}
\newcommand{\Lfp}{\prule{Lfp}\xspace}
\newcommand{\Gfp}{\prule{Gfp}\xspace}
\newcommand{\Fix}{\prule{Fix}\xspace}
\newcommand{\Fixmu}{\prule{Fix$_\mu$}\xspace}
\newcommand{\Fixnu}{\prule{Fix$_\nu$}\xspace}
\newcommand{\FIX}{\Fix}
\newcommand{\LFP}{\Lfp}
\newcommand{\xif}[1]{\xLeftarrow{\text{#1}}}

\newcommand{\Nat}{\mathit{Nat}}
\newcommand{\Map}{\mathit{Map}}
\newcommand{\emp}{\mathit{emp}}
\newcommand{\Cfg}{\mathit{Cfg}}
\newcommand{\simp}{\mathbin{-*}}
\newcommand{\merge}{\mathbin{*}}
\newcommand{\pto}{\rightharpoonup}
\newcommand{\SL}{{\text{SL}}}
\newcommand{\FOL}{{\text{FOL}}}
\newcommand{\TT}{\mathcal{T}}

% contexts
\newcommand{\CSub}[1]{C_{#1}}
\newcommand{\Csigma}{\CSub{\sigma}}
\newcommand{\Csigmai}{\CSub{\sigma,i}}
\newcommand{\Csigmaapp}[1]{\CSub{\sigma}[#1]}
\newcommand{\Csigmaiapp}[1]{\CSub{\sigma,i}[#1]}
\newcommand{\Capp}[1]{C[#1]}

% compliment of symbols contexts
\newcommand{\Csigmabar}{\overline{\CSub{\sigma}}}
\newcommand{\sigmabar}{\bar{\sigma}}
\newcommand{\Cbar}{\bar{C}}

\newcommand{\Prop}{\textsc{Prop}}
\newcommand{\AP}{\textsc{AP}}
\newcommand{\APgm}{\textsc{APgm}}
\newcommand{\Label}{\textsc{Label}}
\newcommand{\true}{\mathit{true}}
\newcommand{\false}{\mathit{false}}
\newcommand{\wnext}{{\medcirc}}
\newcommand{\snext}{{\medbullet}}
\newcommand{\sprev}{{\widebar{\snext}}}
\newcommand{\wprev}{{\widebar{\wnext}}}
\newcommand{\always}{{\square}}
\newcommand{\eventually}{{\lozenge}}
\newcommand{\AAA}{{\mathsf{A}}}
\newcommand{\EEE}{{\mathsf{E}}}
\newcommand{\EE}{{\mathsf{E}}}
\newcommand{\XX}{{\mathsf{X}}}
\newcommand{\UU}{\mathbin{\mathsf{U}}}
\newcommand{\FF}{{\mathsf{F}}}
\newcommand{\GG}{{\mathsf{G}}}
\newcommand{\AG}{{\mathsf{AG}}}
\newcommand{\EG}{{\mathsf{EG}}}
\newcommand{\AF}{{\mathsf{AF}}}
\newcommand{\EF}{{\mathsf{EF}}}
\newcommand{\AX}{\mathsf{AX}}
\newcommand{\AU}{\mathbin{\mathsf{AU}}}
\newcommand{\EU}{\mathbin{\mathsf{EU}}}
\newcommand{\EX}{\mathsf{EX}}
\newcommand{\Us}{\mathbin{\mathsf{U}_s}}
\newcommand{\Uw}{\mathbin{\mathsf{U}_w}}
\newcommand{\rt}{\mathsf{root}}
\newcommand{\subtrees}{\mathsf{subtrees}}
\newcommand{\fullpaths}{\mathsf{fullpaths}}
\newcommand{\world}{\mathit{world}}
\newcommand{\statesort}{\mathit{state}}
\newcommand{\pgm}{\mathit{pgm}}
\newcommand{\hole}{\square}
\DeclarePairedDelimiter{\bracket}{\llbracket}{\rrbracket}
\newcommand{\bracketM}[1]{{\bracket{#1}_\MM}}
\DeclarePairedDelimiter{\angleBraces}{\langle}{\rangle}

\newcommand{\Words}{\textsc{Words}}
\newcommand{\infTraces}{\textsc{Traces$^\omega$}\xspace}
\newcommand{\finTraces}{\textsc{Traces$^*$}\xspace}
\newcommand{\infTrees}{\textsc{Trees$^\omega$}\xspace}
\newcommand{\Inf}{\prule{Inf}\xspace}
\newcommand{\Fin}{\prule{Fin}\xspace}
\newcommand{\Lin}{\prule{Lin}\xspace}
\newcommand{\LL}{\mathcal{L}}
\newcommand{\NN}{\mathbb{N}}
\newcommand{\QED}{\text{QED}}

\newcommand{\textiff}{\text{ iff }}
\newcommand{\xto}[1]{\xrightarrow{#1}}
\newcommand{\xtoa}{\xto{a}}
\newcommand{\pred}{\mathit{pred}}
\newcommand{\succc}{\mathit{succ}}
\newcommand{\CTL}{{\textsc{CTL}}}
\newcommand{\PDL}{\textsc{PDL}}
\newcommand{\textthen}{\text{ then }}
\newcommand{\PDLseq}{\mathbin{;}}
\newcommand{\PDLunion}{\mathbin{\cup}}
\newcommand{\PDLstar}{^*}
\newcommand{\PDLquestion}{{?}}


\newcommand{\sfk}{\textsf{k}\xspace}
\newcommand{\sfstate}{\textsf{state}\xspace}
\newcommand{\ttInt}{\texttt{Int}\xspace}
\newcommand{\ttId}{\texttt{Id}\xspace}
\newcommand{\ttExp}{\texttt{Exp}\xspace}
\newcommand{\ttplus}{\ \texttt{+}\ }
\newcommand{\ttminus}{\ \texttt{-}\ }
\newcommand{\KResult}{\texttt{KResult}\xspace}

\newcommand{\Int}{\textit{Int}}
\newcommand{\Exp}{\textit{Exp}}
\newcommand{\AExp}{\textit{BExp}}
\newcommand{\BExp}{\textit{BExp}}
\newcommand{\ttrue}{\textit{true}}
\newcommand{\ffalse}{\textit{false}}
\newcommand{\Id}{\textit{Id}}
\newcommand{\Stmt}{\textit{Stmt}}
\newcommand{\ite}{\textit{ite}}

\newcommand{\parametric}[2]{{#1}\raisebox{.2ex}{\texttt{\footnotesize{\{}}}#2\raisebox{.2ex}{\texttt{\footnotesize{\}}}}}
\newcommand{\parametricscript}[2]{{#1}\raisebox{.2ex}{\texttt{\tiny{\{}}}#2\raisebox{.2ex}{\texttt{\tiny{\}}}}}

\newcommand{\Context}{\mathit{Context}}
\newcommand{\compose}{\circ}
\newcommand{\strict}[1]{\textsf{strict(#1)}}




\begin{document}

\title{Foundation of Language-Independent Verification}
\author{Xiaohong Chen}
\address{University of Illinois at Urbana-Champaign, USA}
%\email{xc3@illinois.edu}
%\date{}

\begin{abstract}
content...
\end{abstract}

\maketitle

\tableofcontents

% Main sections.

\section{Preliminaries}


\section{Matching logic and its complete deduction}

\subsection{The need for another logic}

\subsection{Syntax and semantics}

A signature is a triple
$\sig = (\Var, S, \Sigma)$ where
$\Var$ is
a countably infinite set of variable names,
$S$ is an nonempty countable (finite or infinite) sort set,
and $\Sigma = \{ \Sigma_{s_1 \dots s_n ,s} \}_{s_1 \ddd s_n , s \in S}$
is an $(S^* \times S)$-indexed countable (finite or infinite) symbol set.
We write $\Sigma_{s_1 \dots s_n , s}$ as
$\Sigma_{\lambda , s}$ if $n = 0$.
When $\Var$ is clear from the context,
we often omit it and just write the signature $\sig = (S, \Sigma)$.
If $S$ is also clear, we refer to a signature
by just $\Sigma$.
The set of all $\sig$-patterns of sort $s$,
denoted as $\Pattern_s(\sig)$ or just $\Pattern_s$ when the signature is clear
from the context,
is defined by the following grammar:
\begin{center}
\begin{tabular}{rcll}
$\varphi_s$
& $\Coloneqq$
& $x \cln s$
  with $x \in \Var$
& \doubleslash Variable
\\
& $|$
& $\sigma(\varphi_{s_1} \ddd \varphi_{s_n})$
  with $\sigma \in \Sigma_{s_1 \dots s_n , s}$
& \doubleslash Structure
\\
& $|$
& $\neg \varphi_s$
& \doubleslash Complement
\\
& $|$
& $\varphi_s \wedge \varphi_s$
& \doubleslash Intersection
\\
& $|$
& $\exists x \cln s'. \varphi_s$
  with $s' \in S$ (no need be the same as $s$)
& \doubleslash Binding
\end{tabular}
\end{center}
The set of all variables of sort $s$ is denoted as $\Var_s$.
Let $\Var = \{ \Var_s \}_{s \in S}$ and $\Pattern = \{ \Pattern_s \}_{s \in S}$
be the $S$-indexed family set of all variable and patterns.
For simpler notation,
we often blur the distinction between
a family of sets and their union,
and use $\Var$ and $\Pattern$
to denote the set of all variables and patterns respectively.
We write
$\varphi \in \Pattern$ to mean that $\varphi$ is a pattern,
and $\varphi_s \in \Pattern$ or $\varphi \in \Pattern_s$ to mean that
it has sort~$s$.
Similarly, $\sigma \in \Sigma$ means $\sigma$ is a symbol.
If $\sigma \in \Sigma_{\lambda, s}$, we say $\sigma$
is a constant symbol of sort $s$, and
we write $\sigma$ instead of~$\sigma()$.
We often drop the sort when writing variables, 
so instead of $x \cln s$, we just write $x$.
We can define the conventional notions of free variables and
alpha-renaming as in first-order logic.
We write $\varphi [\psi/x]$ for variable-capture-free substitution,
in which alpha-renaming happens implicitly to prevent
free variable capturing.

A $\sig$-model (or simply a model) is a pair
$\MM = (M, \interpM)$ 
where $M = \{M_s\}_{s \in S}$ 
is an $S$-indexed family of nonempty carrier sets
and $\interpM = \{ \sigmaM \}_{\sigma \in \Sigma}$
maps every symbol $\sigma \in \Sigma_{s_1 \ddd s_n , s}$
to a function
$\sigmaM \colon M_{s_1} \times \dots \times M_{s_n} \to \pset{M_s}$,
where $2^{M_s}$ means the set of all subsets of $M_s$.
In particular, each constant symbol
$\sigma \in \Sigma_{\lambda,s}$ 
maps to a subset $\sigmaM \subseteq M_s$.
An $\MM$-valuation (or simply a valuation) is a mapping
$\rho \colon \Var \to M$
such that $\rho(x \cln s) \in M_s$ for every sort $s \in S$.
Two valuations $\rho_1$ and $\rho_2$ are $x$-equivalent for some variable $x$,
denoted as $\rho_1 \simx \rho_2$,
if $\rho_1(y) = \rho_2(y)$ for every $y$ distinct from $x$.
A valuation $\rho$ can be extended to a mapping
$\barrho \colon \Pattern \to \pset{M}$
such that $\barrho(\varphi_s) \subseteq M_s$,
in the following inductive way:
\begin{itemize}
\item $\barrho(x) = \{ \rho(x) \}$, for every $x \in \Var_s$;
\item $\barrho(\sigma(\varphi_1 \ddd \varphi_n))
       = \sigmaM( \barrho(\varphi_1) \ddd \barrho(\varphi_n) )$,
      for every $\sigma \in \Sigma_{s_1 \dots s_n , s}$
      and appropriate $\varphi_1 \ddd \varphi_n$
\item $\barrho(\neg \varphi) = M_s \setminus \barrho(\varphi)$,
      for every $\varphi \in \Pattern_s$;
\item $\barrho(\varphi_1 \wedge \varphi_2)
       = \barrho(\varphi_1) \cap \barrho(\varphi_2)$,
      for every $\varphi_1,\varphi_2$ of the same sort;
\item $\barrho(\exists x . \varphi) 
       = \bigcup \{ \barrhop(\varphi) 
                    \mid \text{for every $\rhop \simx \rho$} \}$.
\end{itemize}
{
We also need to give some intuition for "matching" 
(the patterns are "matched" 
by the values in their interpretation, 
which becomes literal when the model contains terms.
}
Intuitively,
$\barrho(\varphi)$ 
is the set of elements that match the pattern $\varphi$.
Derived constructs are defined as follows for convenience:
\begin{center}
\begin{tabular}{rclp{1cm}rcl}
$\top_s$ & $\equiv$ & $\exists x \cln s . x \cln s$
&&
$\bot_s$ & $\equiv$ & $\neg \top_s$
\\
$\varphi_1 \vee \varphi_2$ & $\equiv$ & 
$\neg (\neg \varphi_1 \wedge \neg \varphi_2)$
&&
$\varphi_1 \imp \varphi_2$ & $\equiv$ &
$\neg \varphi_1 \vee \varphi_2$
\\
$\varphi_1 \dimp \varphi_2$ & $\equiv$ &
$(\varphi_1 \imp \varphi_2) \wedge (\varphi_2 \imp \varphi_1)$
&&
$\forall x . \varphi$ & $\equiv$ &
$\neg (\exists x . \neg \varphi)$
\end{tabular}
\end{center}
Interested readers are encouraged to prove these derived constructs
have the intended semantics,
or refer to~\cite{bibid} for details.
We often drop the sort subscripts when there is no confusion.

Given a model $\MM$ and a valuation $\rho$,
we say $\MM$ and $\rho$ satisfy a pattern $\varphi_s$,
denoted as $\MM,\rho \vDash \varphi_s$,
if $\barrho(\varphi) = M_s$.
We say $\MM$ satisfies $\varphi_s$
or $\varphi_s$ holds in $\MM$,
denoted as $\MM \vDash \varphi_s$,
if $\MM , \rho \vDash \varphi_s$ for every valuation $\rho$.
We say $\varphi_s$ is valid if
it holds in every model.
Let $\Gamma$ be a pattern set.
We say $\MM$ satisfies $\Gamma$, if
$\MM \vDash \varphi$ for every $\varphi \in \Gamma$.
We say $\Gamma$ semantically entails $\varphi$,
denoted as $\Gamma \vDash \varphi$,
if for every model $\MM$ such that $\MM \vDash \Gamma$,
$\MM \vDash \varphi$.
When $\Gamma$ is the empty set, we abbreviate
$\emptyset \vDash \varphi$ as just $\vDash \varphi$,
which is equivalent to say $\varphi$ is valid.

{
Say that we use mathcal fonts ${\MM}$, ${\II}$, \dots to denote models,
and the corresponding $M$, $I$,\dots to denote carrier sets.
Use capital Greek letters $\Gamma, \Delta, T, \dots$ to denote pattern set.
In particular, use $\II$ to denote intended models and
$\SSS$ to denote standard models.
}

Given a signature $\sig$, matching logic gives us all $\sig$-models.
Sometimes, we are only interested in some models, instead of all models.
There are typically two ways to restrict models.
One way is to define a syntactic matching logic theory
$(\sig, H)$ where $H$ is a set of patterns called axioms.
A model $\MM$ belongs to the theory $(\sig, H)$ if $M \vDash H$.
Syntactic theories are preferred if the models of interest can be
axiomatized by a recursively enumerable set $H$.
Most syntactic theories defined in this paper have finite axiom sets.
Alternatively, we can define a semantic matching logic theory
$(\sig, \CC)$ where $\CC$ is a collection of $\sig$-models.
A model $\MM$ belongs to the theory $(\sig, \CC)$ if
$\MM \vDash \varphi$ for all $\varphi$ that holds in all models in $\CC$.
Usually, the collection $\CC$ is a singleton set containing exactly one model 
$\II$,
often referred as the intended model or the standard model,
and we abbreviate $(\sig, \{ \II \})$ as $(\sig, \II)$.
Semantic theories are preferred if the intended model is not axiomatized by any
recursively enumerable set of axioms;
this is often the case for 
initial algebra semantics or domains which are defined by induction, such as
natural numbers (with multiplication) and finite maps.
We will see an example of semantic theories in 
Section~\ref{sec_separation_logic}.

Syntactic theories support a notion of proofs (see 
Section~\ref{sec_pure_matching_logic})
as they have an axiom set,
but semantic theories offer arbitrary expressiveness.
They both are means to restricting models,
and we simply say ``theories'' when their distinction is not the emphasis.





\subsection{Known results about matching logic expressiveness power}

{
I think it is also important to state that when all models are assumed, 
then ML has been shown to have the same expressiveness as FOL=, 
but like it is the case with capturing SL, 
or in FOL with induction, 
or in initial algebra semantics, 
one can reduce the set of models and thus get arbitrary expressiveness.
}



A few important logics and calculus are shown to be definable 
in matching logic, including
propositional calculus, predicate logic,
algebraic specification, first-order logic with equality,
modal logic S5, and separation logic.
In this section, we only discuss a few of them and refer interested readers
to~\cite{bibid} for more details.

\subsection{Useful symbols and notations}

\subsection{Sound and complete deduction}


\section{Matching logic subsumes other logics and calculi}

\subsection{Propositional logic}

\subsection{First-order logic}

\subsection{Modal logic S5}

\subsection{Untyped lambda calculus}

\subsubsection{Syntax}

A $\lambda$-term is either a variable,
or an application $e e'$ where $e$ and $e'$ are $\lambda$-terms,
or a lambda abstraction $\lambda x . e$ where $x$ is a variable
and $e$ is a $\lambda$-term.
The set of all $\lambda$-terms is denoted as $\Lambda$.
A $\lambda$-theory is a congruence on $\Lambda$.

Let $\Lambda$ be the set of all $\lambda$-terms.



% Bibliography.
\bibliographystyle{amsalpha}
\bibliography{}

\end{document}