\documentclass{article}

\usepackage{amsmath,amssymb,amsthm}
\usepackage{wasysym}
\usepackage{mathtools}
\usepackage{longtable}
\usepackage{fullpage}
\usepackage{prftree}


% Define the implication symbol ->
\newcommand{\imp}{\rightarrow}
\newcommand{\dimp}{\leftrightarrow}

% Define ceiling and flooring symbols.
\DeclarePairedDelimiter\ceil{\lceil}{\rceil}
\DeclarePairedDelimiter\floor{\lfloor}{\rfloor}

\newcommand{\miniPS}{\mathcal{S}}

\newtheorem{theorem}{Theorem}
\newtheorem{prop}{Proposition}


\title{Minimal Proof System of Matching Logic}
\author{Formal Systems Laboratory}

\begin{document}
	\maketitle

We conjecture that the following is a sound and complete proof system of 
matching logic. 
\begin{figure}[hbtp]
\def\arraystretch{1.25}
\begin{longtable}{ll}
	 \textsc{Propositional$_1$}
  &  $\varphi_1 \imp (\varphi_2 \imp \varphi_1)$
  \\
     \textsc{Propositional$_2$}
  &  $(\varphi_1 \imp (\varphi_2 \imp \varphi_3)) 
      \imp ((\varphi_1 \imp \varphi_2) \imp (\varphi_1 \imp \varphi_3))$
  \\
     \textsc{Propositional$_3$}
  &  $(\neg \varphi_1 \imp \neg \varphi_2) \imp (\varphi_2 \imp \varphi_1)$
  \\
     \textsc{Variable Substitution}
  &  $\forall x . \varphi \imp \varphi[y/x]$
  \\
     \textsc{$_\forall$}
  &  $\forall x . (\varphi_1 \imp \varphi_2) 
     \imp (\varphi_1 \imp \forall x . \varphi_2)$
  \\
  &
     if $x$ does not occur free in $\varphi_1$
  \\
     \textsc{K}
  &  $\sigma(\dots,\varphi_1 \vee \varphi_2,\dots) 
      \dimp
      \sigma(\dots,\varphi_1,\dots) \vee \sigma(\dots,\varphi_2,\dots)$
  \\
     \textsc{N}
  &  $\neg \sigma(\bot)$
  \\
     \textsc{Barcan}
  &  $\sigma(\dots,\exists x . \varphi,\dots) \dimp \exists x . 
      \sigma(\dots, \varphi,\dots)$
  \\
  &
     if $x$ does not occur free
  \\
    \textsc{Name}
  & $\exists x . x$
  \\
    \textsc{Modus Ponens}
  &  From $\varphi_1$ and $\varphi_1 \imp \varphi_2$ deduce $\varphi_2$
  \\
    \textsc{Universal Generalization}
  &  From $\varphi$ deduce $\forall x . \varphi$
\end{longtable}
\caption{Minimal Proof System $\miniPS$}
\end{figure}

Matching logic is proved to have a complete proof system in the presence of 
definedness symbols.
Notice the above minimal proof system $\miniPS$ does not depend on definedness 
symbols.
In the following, we show how to establish all axioms and rules of the old 
proof system using the new minimal proof system $\miniPS$ plus the 
axiom $\forall x . \ceil{x}$ for definedness symbols.

\begin{prop}
	$\vdash \varphi = \varphi$
\end{prop}
\begin{proof}
	content...
\end{proof}


%\bgroup
%\def\arraystretch{1.25}
%\begin{longtable}{ll}
%	\textsc{Propositional$_1$}
%	&  $\varphi_1 \imp (\varphi_2 \imp \varphi_1)$
%	\\
%	\textsc{Propositional$_2$}
%	&  $(\varphi_1 \imp (\varphi_2 \imp \varphi_3)) 
%	\imp ((\varphi_1 \imp \varphi_2) \imp (\varphi_1 \imp \varphi_3))$
%	\\
%	\textsc{Propositional$_3$}
%	&  $(\neg \varphi_1 \imp \neg \varphi_2) \imp (\varphi_2 \imp \varphi_1)$
%	\\
%	\textsc{Variable Substitution}
%	&  $\forall x . \varphi \imp \varphi[y/x]$
%	\\
%	\textsc{$_\forall$}
%	&  $\forall x . (\varphi_1 \imp \varphi_2) 
%	\imp (\varphi_1 \imp \forall x . \varphi_2)$
%	\\
%	&
%	if $x$ does not occur free in $\varphi_1$
%	\\
%	\textsc{K}
%	&  $\sigma(\dots,\varphi_1 \vee \varphi_2,\dots) 
%	\dimp
%	\sigma(\dots,\varphi_1,\dots) \vee \sigma(\dots,\varphi_2,\dots)$
%	\\
%	\textsc{N}
%	&  $\neg \sigma(\bot)$
%	\\
%	\textsc{Barcan}
%	&  $\sigma(\dots,\exists x . \varphi,\dots) \dimp \exists x . 
%	\sigma(\dots, \varphi,\dots)$
%	\\
%	&
%	if $x$ does not occur free
%	\\
%	&  in the left-hand side of the double implication
%	\\
%	\textsc{*Membership Symbol}
%	&  $x \in \sigma(\dots,\varphi_i,\dots) \imp \exists y . y \in 
%	\varphi_i \wedge x 
%	\in \sigma(\dots,y,\dots)$
%	\\
%	\textsc{*Membership $\neg$}
%	&  $x \in \neg \varphi \imp \neg (x \in \varphi)$
%	\\
%	\textsc{*Membership $\exists$}
%	&  $x \in \forall y . \varphi
%	\imp \forall y . (x \in \varphi)$
%	\\
%	&
%	where $x$ is distinct from $y$
%	\\
%	\textsc{*Definedness}
%	&  $\ceil{x}$
%	\\
%	\textsc{Modus Ponens}
%	&  From $\varphi_1$ and $\varphi_1 \imp \varphi_2$ deduce $\varphi_2$
%	\\
%	\textsc{Universal Generalization}
%	&  From $\varphi$ deduce $\forall x . \varphi$
%	\\
%	\textsc{Equational Substitution}
%	&  From $\varphi_1 \dimp \varphi_2$ deduce $\varphi_3[\varphi_1/x] \dimp 
%	\varphi_3[\varphi_2/x]$
%	\\
%	\textsc{*Introduction$_\in$}
%	&  From $\varphi$ deduce $x \in \varphi$
%	\\
%	&
%	If $x$ does not occur free in $\varphi$
%	\\
%	\textsc{*Elimination$_\in$}
%	&  From $x \in \varphi$ deduce $\varphi$
%	\\
%	&
%	If $x$ does not occur free in $\varphi$
%\end{longtable}
%\egroup
\end{document}