\documentclass{article}
\usepackage[utf8]{inputenc}
\usepackage{listings}

\setlength{\parskip}{1em}

\lstdefinestyle{code}{
    basicstyle=\footnotesize\ttfamily,
    frame=single,
    breaklines=true
}

\begin{document}

Hi Grigore,

I have a feeling that we in fact disagree on nothing. 

Let me use a simple \texttt{IMP} program as an example to illustrate what I mean. 

Remind that we have Maude module called \texttt{IMP} that builds all patterns. 


\lstset{style=code}

\begin{lstlisting}
fmod IMP is
  sorts Variable Trm Trm? Predicate Pattern ... .
  ...
  op ite : Pattern{Bool} Pattern{Pgm} Pattern{Pgm} 
        -> Pattern{Pgm} [frozen(2 3)] .
  op while : Pattern{Bool} Pattern{Pgm} 
          -> Pattern{Pgm} [frozen(1 2)] .
  ---- dereferencing a pointer
  op dereference : Pattern{Nat} -> Pattern{Nat} . 
  ...
endfm

\end{lstlisting}

My objective is to write a Maude module called \texttt{EXE} that reduces configurations of \texttt{IMP} programs according to their semantics. 

\lstset{style=code}

\begin{lstlisting}
fmod EXE is
  vars P Q : Pattern{Pgm} . var B : Pattern{Bool} .
  eq ite(tt, P, Q) = P .
  eq ite(ff, P, Q) = Q .
  eq while(B, P) = ite(B, seq(P, while(B, P)), skip) .
  ...
endfm

reduce

#cfg(#k(seq(while(...), ...)),
     #state(merge(binder(three, one), ...))))
     
.

\end{lstlisting}

Notice how ``frozen'' attributes play a role here.

The look-up rule is the problem. 

\lstset{style=code}

\begin{lstlisting}
   #cfg(C[dereference(X)], C'[binder(X, V)]) 
=> #cfg(C[V], C'[binder(X, V)]) .
\end{lstlisting}

This look-up rule has two context variables $C$ and $C'$. It is easy to instantiate $C'$ as follows

\lstset{style=code}

\begin{lstlisting}
   #cfg(C[dereference(X)], #state(merge(binder(X, V), H)))
=> #cfg(C[V], #state(merge(binder(X, V), H))) .
\end{lstlisting}

\noindent where variable $H$ can match the rest of the heap, thanks to the associativity and commutativity of \texttt{merge} and that \texttt{emp} being the identity of \texttt{merge}.

The context $C$, though, has an infinite number of instances. For example,

\lstset{style=code}

\begin{lstlisting}
   #cfg(#k(asgn(Y, dereference(X))), 
        #state(merge(binder(X, V), H)))
=> #cfg(#k(asgn(Y, V)), 
        #state(merge(binder(X, V), H)))
        
   #cfg(#k(asgn(Y, succ(dereference(X)))), 
        #state(merge(binder(X, V), H)))
=> #cfg(#k(asgn(Y, V)), 
        #state(merge(binder(X, V), H)))
        
   #cfg(#k(asgn(Y, succ(succ(dereference(X))))), 
        #state(merge(binder(X, V), H)))
=> #cfg(#k(asgn(Y, V)), 
        #state(merge(binder(X, V), H)))
        
   #cfg(#k(asgn(Y, succ(succ(succ(dereference(X)))))), 
        #state(merge(binder(X, V), H)))
=> #cfg(#k(asgn(Y, V)), 
        #state(merge(binder(X, V), H)))
        
        ......
        
\end{lstlisting}

Of course, given an initial configuration, we can write a Maude module \texttt{EXE} with a finite number of rules that reduces \emph{that} configuration all the way to the end. My argument is that we cannot write a Maude module which can reduce \emph{all} configurations, because that will need an infinite number of rules, unless we explicitly deal with context splitting, plugging, refocusing, etc.

I have been reading Traian's ``\emph{A rewriting logic approach to operational semantics}'' and I found many good references in that paper. I have been looking into reduction semantics and techniques (such as refocusing) that helps to implement an efficient interpreter. I will keep reading literatures for a while.

I also intend to implement an algorithm in Maude that calculates the equivalence class under context application of a pattern $\varphi$, which is the set of all pairs of context and redex $(C, R)$ such that $\varphi = C[R]$.

\noindent Yours, \newline
\noindent Xiaohong
        
\end{document}