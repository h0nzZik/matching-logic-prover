\documentclass{amsart}

\usepackage{amsmath,amsthm,amssymb,mathtools}
\usepackage{thmtools}
\usepackage{thm-restate}
\newtheorem{remark}{Remark}
\usepackage{comment}

\usepackage{amsaddr}


\usepackage{mathbbol}
\usepackage{prftree}

\usepackage{xspace}

\usepackage{longtable}
\usepackage{caption}
\usepackage{todonotes}


\newtheorem{definition}{Definition}
\newtheorem{example}{Example}
\newtheorem{lemma}{Lemma}
\newtheorem{proposition}{Proposition}
\newtheorem{corollary}{Corollary}
\newtheorem{theorem}{Theorem}

% New Command Defniitions

% matching logic syntax

% logic connectives
\newcommand{\imp}{\to}
\newcommand{\dimp}{\leftrightarrow}

% signatures
\newcommand{\Var}{\textnormal{\textsc{Var}}}
\newcommand{\sig}{\mathbb{\Sigma}}
\newcommand{\Pattern}{\textnormal{\textsc{Pattern}}}

\newcommand{\eVar}{\Var^+}
\newcommand{\esig}{\sig^+}
\newcommand{\ePattern}{\Pattern^+}

\newcommand{\newVar}{\eVar \setminus \Var}
\newcommand{\newvar}[1]{\mathbb{#1}}
\newcommand{\newx}{\newvar{x}}
\newcommand{\newy}{\newvar{y}}

% pattern sets and related symbols
\newcommand{\eGamma}{\Gamma^+}

\newcommand{\Gammai}[1]{\Gamma_i^{(#1)}}
\newcommand{\Gammaik}{\Gammai{k}}
\newcommand{\Deltai}[1]{\Delta_i^{(#1)}}
\newcommand{\Deltaik}{\Deltai{k}}
\newcommand{\pii}[1]{\pi_i^{(#1)}}


\newcommand{\psii}[1]{\psi_i^{(#1)}}


% matching logic semantics

% models
\newcommand{\MM}{\mathcal{M}}
\newcommand{\WW}{\mathcal{W}}
\newcommand{\YY}{\mathcal{Y}}

\newcommand{\interp}[1]{\__{#1}}
\newcommand{\interpM}{\interp{\MM}}
\newcommand{\interpY}{\interp{\YY}}
\newcommand{\interpW}{\interp{\WW}}

\newcommand{\sigmaM}{{\sigma_{\MM}}}
\newcommand{\sigmaW}{{\sigma_{\WW}}}
\newcommand{\sigmaY}{{\sigma_{\YY}}}

\newcommand{\widebar}[1]{\overline{#1}}

% valuations

\newcommand{\rhox}[1]{{\rho_{#1}}}
\newcommand{\rhop}{\rho'}
\newcommand{\rhopx}[1]{{\rhop_{#1}}}
\newcommand{\barrho}{\bar{\rho}}
\newcommand{\barrhop}{\widebar{\rhop}}
\newcommand{\barrhox}[1]{\widebar{\rhox{#1}}}
\newcommand{\barrhopx}[1]{\widebar{\rhopx{#1}}}

\newcommand{\simon}[1]{\overset{#1}{\sim}}
\newcommand{\simx}{\simon{x}}
\newcommand{\simz}{\simon{z}}


% contexts
\newcommand{\CSub}[1]{C_{#1}}
\newcommand{\Csigma}{\CSub{\sigma}}
\newcommand{\Csigmai}{\CSub{\sigma,i}}
\newcommand{\Csigmaapp}[1]{\CSub{\sigma}[#1]}
\newcommand{\Csigmaiapp}[1]{\CSub{\sigma,i}[#1]}
\newcommand{\Capp}[1]{C[#1]}

% compliment of symbols contexts
\newcommand{\Csigmabar}{\overline{\CSub{\sigma}}}
\newcommand{\sigmabar}{\bar{\sigma}}
\newcommand{\Cbar}{\bar{C}}

% name of the proof rules
\newcommand{\prule}[1]{\textsc{(#1)}}

\newcommand{\modusponens}{\prule{Modus Ponens}\xspace}
\newcommand{\universalgeneralization}{\prule{Universal Generalization}\xspace}
\newcommand{\necessitation}{\prule{Necessitation}\xspace}
\newcommand{\existence}{\prule{Existence}\xspace}
\newcommand{\singletonvariable}{\prule{Singleton Variable}\xspace}
\newcommand{\propagationbottom}{\prule{Propagation$_\bot$}\xspace}
\newcommand{\propagationvee}{\prule{Propagation$_\vee$}\xspace}
\newcommand{\propagationexists}{\prule{Propagation$_\exists$}\xspace}
\newcommand{\variablesubstitution}{\prule{Variable Substitution}\xspace}
\newcommand{\framing}{\prule{Framing}\xspace}
\newcommand{\propositionaltautology}{\prule{Propositional Tautology}\xspace}
\newcommand{\forallrule}{\prule{$\forall$}\xspace}
\newcommand{\membership}{\prule{Membership}\xspace}
\newcommand{\membershipintroduction}{\prule{Membership Introduction}\xspace}
\newcommand{\membershipelimination}{\prule{Membership Elimination}\xspace}
\newcommand{\membershipneg}{\prule{Membership$_\neg$}\xspace}
\newcommand{\membershipwedge}{\prule{Membership$_\wedge$}\xspace}
\newcommand{\membershipexists}{\prule{Membership$_\exists$}\xspace}
\newcommand{\equalityelimination}{\prule{Equality Elimination}\xspace}
\newcommand{\membershipsymbol}{\prule{Membership Symbol}\xspace}
\newcommand{\membershipvariable}{\prule{Membership Variable}\xspace}
\newcommand{\functionalsubstitution}{\prule{Functional Substitution}\xspace}

% \newcommand{\eframing}{Proposition~\ref{prop_framing}\xspace}
\newcommand{\eframing}{Frame reasoning\xspace}
\newcommand{\epropagation}{Propagation\xspace}

% Notations used in completeness proof.
\newcommand{\GP}{GP}  % generating path relation
\newcommand{\seq}[1]{\langle #1 \rangle} % sequence in angle brackets
\newcommand{\barpsi}{\bar{\psi}}

% Closure operations
\newcommand{\CC}{\mathcal{C}}
\newcommand{\Closure}[1]{\CC(#1)}
\newcommand{\ClosureA}[1]{\CC_1(#1)}
\newcommand{\ClosureB}[1]{\CC_2(#1)}
\newcommand{\ClosureGamma}{\Closure{\Gamma}}
\newcommand{\ClosureGammao}{\Closure{\Gamma_0}}

% Misc
\newcommand{\fv}{\mathit{FV}}
\newcommand{\FV}{\fv}

\newcommand{\hole}{\square}
\newcommand{\ddd}{,\dots,}
\newcommand{\cln}{{:}}
\newcommand{\fin}{\mathit{fin}}

\newcommand{\txtand}{\text{ and }}
\newcommand{\doubleslash}{//\xspace}

\DeclarePairedDelimiter{\ceil}{\lceil}{\rceil}
\DeclarePairedDelimiter{\floor}{\lfloor}{\rfloor}

\makeatletter
\newcommand{\rmnum}[1]{\romannumeral #1}
\newcommand{\Rmnum}[1]{\expandafter\@slowromancap\romannumeral #1@}
\makeatother


% TODO:: move the following commands to the above.
% TODO:: clean up commands!!

% canonical model
\newcommand{\sigmaMGammaoc}{\sigma_\WW}
% generated models
\newcommand{\My}{{Y_{\Gamma_0}}}
\newcommand{\Modely}{{\mathcal{Y}_{\Gamma_0}}}
\newcommand{\sigmaMGammaow}{\sigma_\Modely}
% completed models
\newcommand{\M}{{M_{\Gamma_0}}}
\newcommand{\Model}{{\mathcal{M}_{\Gamma_0}}}
\newcommand{\sigmaMGammao}{{\sigma_\Model}}



\newcommand{\Modelo}{{\Model_0}}
\newcommand{\rhoo}{{\rho_0}}


\newcommand{\vdashd}{\vdash_d}

\newcommand{\Ctau}{{C_\tau}}
\newcommand{\Cpi}{{C_\pi}}

\newcommand{\Deltao}{\Delta^0}




% Misc
\newcommand{\SigmaSub}[1]{\Sigma_{#1}}
\newcommand{\SetOF}[1]{\{ #1 \}}
\newcommand{\pset}[1]{2^{#1}}





% Commands for referencing theorems, figures, etc.
\newcommand{\Figure}[1]{Figure~\ref{#1}}
\newcommand{\Lemma}[1]{Lemma~\ref{#1}}
\newcommand{\Prop}[1]{Proposition~\ref{#1}}

\title{Matching $\mu$-Logic}
\author{Xiaohong Chen}
\author{Grigore Rosu}

\begin{document}
	
\begin{abstract}
We propose in this paper matching $\mu$-logic, 
an extension to matching logic with the fixpoint construct $\mu$-binder.
We investigate two notions of semantics of matching $\mu$-logic:
the standard semantics and the Henkin semantics. 
For the latter, a sound and complete proof system is given.
We show that many important logics in mathematics and computer science,
especially those about fixpoints and induction, are instances
of matching $\mu$-logic, including
linear temporal logic (LTL),
computation tree logic (CTL),
modal $\mu$-logic,
propositional dynamic logic (PDL),
and first-order logic with least fixpoints (FOL$_\mathrm{lfp}$).
We hope this paper offers some evidence that matching $\mu$-logic
can be taken as the unifying foundation of reasoning about fixpoints
in mathematics and computer science.
\end{abstract}

\maketitle

\tableofcontents

\todochen{Please, Xiaohong, please, revise the language. 
Definitions should be precise, rigorous, and compact. 
Proofs should be fluent, readable, intuitive, and, of course, sound.}

\section{Background and Motivation}

\section{Preliminaries}

\begin{definition}[Indexed families of sets]
Let $S$ be a set. 
A set $F = \{F_i\}_{i \in I}$ is called an $I$-indexed family of sets
over $S$
where $I$ is an index set and every $F_i $ is a subset of $S$.
When the concrete construction of the set $S$ is not important,
we omit it and simply assume it exists.
The family $F$ is called pairwise disjoint if
for any $i, j \in I$ with $i \neq j$, 
$F_i \cap F_j = \emptyset$.
When $F$ is a pairwise disjoint family,
we write $x \in F$ to mean that $x$ belongs to some set $F_i$ in the family. 
\end{definition}

\begin{notation}
Let $F = \{F_i\}_{i \in I}$ and $G = \{G_i\}_{i \in I}$ be two families. 
An $I$-indexed function $h = \{ h_i \}_{i \in I}$,
denoted shortly as $h \colon F \to G$,
is a set of
functions $h_i \colon F_i \to G_i$ for every index $i \in I$.
We use $\pset{F} = \{ \pset{F_i} \}_{i \in I}$ to denote the family of 
the power sets of the element sets in $F$.
\end{notation}

\begin{definition}
Given a set $M$, a function $F \colon \pset{M} \to \pset{M}$ is order-preserving
if
for every sets $A, B \subseteq M$ such that $A \subseteq B$, 
$F(A) \subseteq F(B)$. 
\end{definition}

\begin{theorem}[Knaster-Tarski theorem]
Given a set $M$ and an order-preserving function $F \colon \pset{M} \to
\pset{M}$,
the set of all fixpoints of $F$,
denoted as $\fixpoint(F) = \{ A \subseteq M \mid F(A) = A \}$,
is nonempty and forms a complete lattice w.r.t. the subset relation $\subseteq$,
with:
\begin{itemize}
\item $\mu F = \bigcap \{ A \subseteq M \mid F(A) \subseteq A \}$
as the least fixpoint; and
\item $\nu F = \bigcup \{ A \subseteq M \mid A \subseteq F(A) \}$
as the greatest fixpoint.
\end{itemize}
\end{theorem}

\begin{theorem}
\end{theorem}



\section{Syntax and Semantics of Matching $\mu$-Logic}

\subsection{Syntax}
\label{sec:MmL_sytax}

We assume readers are familiar with first-order logic
and modal $\mu$-logic, in particular the use of
$\exists$-binder and $\mu$-binder. 
We also assume readers are familiar with the notions
of many-sorted sets and functions.


\begin{notation}[Variables]
\label{not:variables}
Let $S$ be a nonempty set of sorts.
We need two kinds of variables. 
One is \emph{element variables}, denoted as
$x \cln s, y \cln s, \dots$.
The set of all element variables is denoted as $\EVar$.
The other is \emph{set variables}, denoted as
$X \cln s, Y \cln s, \dots$.
The set of all set variables is denoted as $\SVar$.
To keep our notation tidy,
we drop the sorts when they are clear from the context
and just write $x, y, X, Y$, etc.
We assume there are countably infinitely many 
element and set variables of each sort.
Finally, let $\Var = \EVar \cup \SVar$ be the set of all variables.
\end{notation}

\begin{notation}[Many-sorted symbols]
Let $S$ be a set of sorts. 
We write $\sigma(s_1,\dots,s_n) \cln s$ to mean
an $n$-ary symbol with argument sorts $s_1,\dots,s_n$
and return sort $s$.
We use $\Sigma$ to denote the set of many-sorted symbols,
and we sometimes write $\sigma \in \Sigma_{s_1 \dots s_n , s}$
to mean the same thing as $\sigma(s_1,\dots,s_n) \cln s$.
\end{notation}

\begin{definition}
Given a many-sorted signature $\sig = (S, \Var, \Sigma)$
with a sort set $S$, a variable set $\Var$, and a many-sorted symbol set $\Sigma$,
the set of \emph{matching $\mu$-logic patterns}, 
or simply \emph{pattern}, 
denoted as $\Pattern = \{\Pattern_s\}_{s \in S}$,
is inductively defined by the following grammar:
\begin{center}
\begin{tabular}{rcll}
$\varphi_s$
& $\Coloneqq$
& $x \cln s \in \EVar_s$
& \doubleslash Element variables
\\
& $|$
& $X \cln s \in \SVar_s$
& \doubleslash Set variables
\\
& $|$
& $\sigma(\varphi_{s_1} ,\dots, \varphi_{s_n})$
  where $\sigma(s_1,\dots,s_n) \cln s \in \Sigma$
& \doubleslash Structure
\\
& $|$
& $\neg \varphi_s$
& \doubleslash Complement
\\
& $|$
& $\varphi_s \wedge \varphi_s$
& \doubleslash Intersection
\\
& $|$
& $\exists x \cln s' \ldot \varphi_s$
  where $s' \in S$
& \doubleslash First-order binding
\\
& $|$
& $\mu X \cln s' \ldot \varphi_s$,
  where $s' \in S$ and $\varphi_s$ is positive in $X \cln s'$
& \doubleslash Least fixpoint binding
\end{tabular}
\end{center}
Notice that in $\exists x \cln s' \ldot \varphi_s$ and $\mu X \cln s' \ldot \varphi_s$,
the sort $s'$ can be different from $s$.
\end{definition}

The syntax has two binders: $\exists$ and $\mu$.
The $\exists$-binder in $\exists x \cln s' \ldot \varphi_s$
binds all free occurrences of $x \cln s'$ in $\varphi_s$,
just like in first-order logic.
The $\mu$-binder in $\mu X \cln s' \ldot \varphi_s$
binds all free occurrences of $X \cln s'$ in $\varphi_s$,
just like in modal $\mu$-logic.

When $\sigma$ is a constant symbol, 
we write the pattern $\sigma()$ as just $\sigma$.

We use $\varphi[\varphi_s / x \cln s]$ to denote
the result of substituting $\varphi_s$ for $x \cln s$
in $\varphi$, where $\alpha$-renaming happens implicitly to avoid
unintended capturing of free variables. 
Similarly, we use $\varphi[\varphi_s / X \cln s]$ to denote
the result of substituting $\varphi_s$ for $X \cln s$
in $\varphi$, and $\alpha$-renaming also happens implicitly.

Propositional connectives
such as $\varphi \to \varphi$, $\varphi \vee \varphi$,
$\varphi \dimp \varphi$
are defined in the usual way.
The top $\top_s \equiv \exists x \cln s \ldot x \cln s$ 
and bottom $\bot_s \equiv \mu X \cln s \ldot X \cln s$ of sort $s$
are defined in a way that they contain no free variables.  
It might not be clear why we define them as so,
but we will see the reason very soon in Section~\ref{sec:standard_semantics}
when we show that they have the intended semantics.

The $\exists$-binder and $\mu$-binder have their
dual versions defined in the following usual way:
\begin{align*}
\forall x \cln s \ldot \varphi \equiv \neg \exists x \cln s \ldot \neg \varphi
&&
\nu X \cln s . \varphi \equiv \neg \mu X \cln s . \neg \varphi[\neg X \cln s / 
X \cln s]
\end{align*}
It is easy to show that the $\nu$-binder
is well-defined, by verifying that $\neg \varphi[\neg X \cln s / X \cln s]$ is 
positive in
$X \cln s$ if $\varphi$ is so.

Finally, we feel free to drop the sorts whenever we can.


\subsection{Standard semantics}
\label{sec:standard_semantics}

\begin{definition}
A matching $\mu$-logic model $M$ of a give signature $\sig = (S, \Sigma)$
consists of
\begin{itemize}
\item a sortwise family of carrier sets $M = \{M_s\}_{s \in S}$; and
\item a function 
$\sigma_M \colon M_{s_1} \times \dots \times M_{s_n} \to \pset{M_s}$
for every symbol $\sigma(s_1,\dots,s_n) \cln s \in \Sigma$.
\end{itemize}
\end{definition}
We use the same notation $M$ to mean the model and its carrier sets. 
Notice that symbols are interpreted as relations. 
We tacitly use the same notion $\sigma_M$ for its
pointwise extension 
$\sigma_M \colon \pset{M_{s_1}} \times \dots \times \pset{M_{s_n}} 
\to \pset{M_s}$, which is defined as
\begin{align}
\label{eq:pointwise_extension}
\sigma_M(A_1,\dots,A_n) = \bigcup 
\{ \sigma_M(a_1,\dots,a_n) \mid  
a_1 \in A_1, \dots, a_n \in A_n
\}\\
\quad \text{for all $A_1 \subseteq M_{s_1}$, \dots, $A_n \subseteq M_{s_n}$}.
\end{align}

\begin{remark}

One can think of an $n$-ary symbol $\sigma$
as being an $(n+1)$-ary predicate.

\end{remark}

\begin{remark}

The pointwise extension of $\sigma_M$ is convenient but also sets a trap.
One might think that it can be any function
$f \colon \pset{M_{s_1}} \times \dots \times \pset{ M_{s_n}} \to \pset{M_s}$.
This is not true.
For example, $\sigma_M(\emptyset,\dots,\emptyset) = \emptyset$ always holds,
no matter which model $M$ we choose to interpret the symbol $\sigma$.
In fact, $\sigma_M$ can only be functions that are \emph{extensional},
in the sense of~\eqref{eq:pointwise_extension}.

\end{remark}

In what follows, we give the first semantics notion for
matching $\mu$-logic: the \emph{standard semantics}. 
The characteristic feature of the standard semantics
is that it requires the least fixpoint pattern $\mu X . \varphi$
to be interpreted as the true least fixpoint in the model.
Given that we have first-order binding $\exists$ in matching $\mu$-logic,
we know the standard semantics does not admit a complete proof system
that can prove all valid patterns.
We will formally prove this important result in Theorem~\ref{thm:std_semantics_incomplete}.

\begin{definition}

Given a model $M$, a valuation $\rho$ is a function such that
\begin{itemize}
\item $\rho(x \cln s) \in M_s$ for all element variable $x \cln s \in \EVar$;
\item $\rho(X \cln s) \in \pset{M_s}$
for all set variable $X \cln s \in \SVar$.
\end{itemize}
Given an element variable $x$ and an element $a \in M$, 
we define $\rho[a/x]$ as the valuation
that maps $x$ to $a$
and maps all other variables the same as $\rho$.
Similarly, $\rho[a_1/x_1, \dots, a_n/x_n]$ is the valuation
where we simultaneously substitute $a_1$ for $x_1$, \dots,
$a_n$ for $x_n$, where $x_1,\dots,x_n$ all distinct.
Given a set variable $X$ and a subset $A \subseteq M$,
we define $\rho[A / X]$ as the valuation
that maps $X$ to $A$ and maps all other variables the same as $\rho$.
We write $\rho \simon{x : s} \rho'$ 
to mean that two valuations $\rho$ and $\rho'$ 
agree on all variables except $x \cln s$. 

\end{definition}



\begin{definition}
\label{def:std_interpretation}

Given a model $M$ and a valuation $\rho$,
the \emph{standard interpretation} is a function
$\rhobar_\std \colon \Pattern \to M$
defined inductively as follows:
\begin{itemize}
\item $\rhobar_\std(x \cln s) = \{\rho(x \cln s)\}$, 
for all $x \cln s \in \EVar$;
\item $\rhobar_\std(X \cln s) = \rho(X \cln s)$
for all $X \cln s \in \SVar$;
\item $\rhobar_\std(\sigma(\varphi_1 ,\dots, \varphi_n))
= \sigmaM( \rhobar_\std(\varphi_1) ,\dots, \rhobar_\std(\varphi_n) )$,
for all $\sigma(s_1,\dots,s_n) \cln s \in \Sigma$
and appropriate $\varphi_1 ,\dots, \varphi_n$;
\item $\rhobar_\std(\neg \varphi) = M_s \setminus \rhobar_\std(\varphi)$,
      for every $\varphi \in \Pattern_s$;
\item $\rhobar_\std(\varphi_1 \wedge \varphi_2)
       = \rhobar_\std(\varphi_1) \cap \rhobar_\std(\varphi_2)$,
      for every $\varphi_1,\varphi_2$ of the same sort;
\item $\rhobar_\std(\exists x \ldot \varphi) 
       = \bigcup \{ \rhobarp_\std(\varphi) 
                    \mid \text{for every $\rhop \simon{x} \rho$} \}$.
\item $\rhobar_\std(\mu X \ldot \varphi) = \mu \Fcal$,
where 
$\Fcal \colon \pset{M} \to \pset{M}$ is defined as
$\Fcal(A) = \widebar{\rho[A/X]}_\std(\varphi)$ 
for any subset $A \subseteq M$.
\end{itemize}
One can easily show the interpretation of $\mu X \ldot \varphi$
is well-defined, by verifying that the function $\Fcal$ is monotone,
using the fact that $\varphi$ is positive in $X$.
\end{definition}

\begin{example}
Consider a matching $\mu$-logic signature that contains
one sort $\Nat$ for natural numbers and two symbols:
$0 \cln \Nat$ and $s(\Nat) \cln \Nat$.
We can use $\mu$-binder to specify an axiom that powers the induction reasoning
over the domain as follows:
\begin{align*}
\prule{Inductive Domain} \quad
\mu X . (0 \vee s(X)).
\end{align*}
\end{example}

\begin{example}
Consider a matching $\mu$-logic signature that contains
a sort $\Nat$ for natural numbers and a sort $\Map$ for finite maps,
i.e., partial functions from $\Nat$ to $\Nat$ with finite domains. 
Apart from the usual axioms, we can use $\mu$-binder to specify
the following axiom that powers induction:
\begin{align*}
\prule{Inductive Domain} \quad
\mu M . (\emp \vee (\exists x \exists y \ldot x \mapsto y) \merge M)
\end{align*}
\end{example}

So far, we have seen how we can use the $\mu$-binder to
define a \emph{set} as a least fixpoint. 
In practice, however, this is not enough. 
For example, we want to define multiplication 
$\_ \! \times \! \_ \colon \Sigma_{\Nat\Nat,\Nat}$ as
$$
x \times y =_\lfp ((x = 0) \wedge 0) 
\vee \exists x_1 \ldot (x = s(x_1) \wedge y + (x_1 \times y))
$$
For another example, we want to define a list abstraction on maps:
$$
\listit(x) =_\lfp ((x = 0) \wedge \emp) \vee \exists y \ldot (x \mapsto y \merge \listit(y))
$$
We will see how to define such recursive symbols using the $\mu$-binder
in Section~\ref{sec:recursive_symbols}.

The interpretation given in Definition~\ref{def:std_interpretation}
is called the \emph{standard semantics} because the $\mu$-binder
is interpreted as the least fixpoints in the model.
One can prove, using G\"odel's first incompleteness theorem,
that matching $\mu$-logic 
does not admit a complete deduction under the standard semantics.
However, a complete deduction is possible under a different semantics notion,
where the least fixpoint pattern 
$\mu X . \varphi$ 
is not asked to be interpreted as
the true least fixpoint in the model, but only the least \emph{definable} fixpoint.
This is known as the \emph{Henkin semantics}, which we defines in
Section~\ref{sec:hnk_semantics}.

\subsection{Henkin semantics}
\label{sec:hnk_semantics}

\quad\\
\todochen{Henkin semantics.}

\section{Proof System of Matching $\mu$-Logic}

\begin{figure}[hbtp]
\begin{tabular}{ll}
	\hline
	\prule{Proposition$_1$} &
	$\varphi_1 \imp (\varphi_2 \imp \varphi_1) $
	\\
	\prule{Proposition$_2$} &
	$(\varphi_1 \imp (\varphi_2 \imp \varphi_3))
	 \imp (\varphi_1 \imp \varphi_2)
	 \imp (\varphi_1 \imp \varphi_3)$
	\\
	\prule{Proposition$_3$} &
	$(\neg \varphi_1 \imp \neg \varphi_2)
	 \imp (\varphi_2 \imp \varphi_1)$
	\\
	\modusponens &
	$$
	\begin{prftree}
	{\varphi_1}{\varphi_1 \imp \varphi_2}
	{\varphi_2}
	\end{prftree}
	$$
	\\
	\hline
	\variablesubstitution &
	$\forall x . \varphi \imp \varphi[y/x]$
	\\
	\prule{$\forall$} &
	$\forall x . (\varphi_1 \imp \varphi_2) 
	 \imp (\varphi_1 \imp \forall x . \varphi_2)$
	 \quad if $x \not\in \FV(\varphi_1)$
	\\
	\universalgeneralization &
	$$
	\begin{prftree}
	{\varphi}
	{\forall x . \varphi}
	\end{prftree}
	$$
	\\
	\hline
	\propagationbottom &
	$\Csigma[\bot] \imp \bot$
	\\
	\propagationvee &
	$\Csigma[\varphi_1 \vee \varphi_2]
	 \imp \Csigma[\varphi_1] \vee \Csigma[\varphi_2] $
	\\
	\prule{Propagation$_\exists$} &
	$\Csigma[\exists x . \varphi]
	 \imp \exists x . \Csigma[\varphi]$
	 \quad if $x \not\in \FV(\Csigmaapp{\exists x . \varphi})$
	\\
	\hline
	\framing &
	$$
	\begin{prftree}
	{\varphi_1 \imp \varphi_2}
	{\Csigma[\varphi_1] \imp \Csigma[\varphi_2]}
	\end{prftree}
	$$
	\\
	\hline
	\prule{Pre-fixpoint} &
	$\varphi[\mu X \ldot \varphi / X] \imp \mu X \ldot \varphi
	$
	\\
	\prule{Knaster-Tarski} &
	$$
	\begin{prftree}
	{
	\varphi[\psi / X] \imp \psi
	}
	{
	\mu X . \varphi \imp \psi
	}
	\end{prftree}
	$$
	\\
	\hline
	\existence &
	$\exists x . x$
	\\
	\singletonvariable &
	$\neg (C_1[x \wedge \varphi] \wedge C_2[x \wedge \neg \varphi])$
	\\ & where $C_1$ and $C_2$ are symbol contexts.
	\\
	\hline
\end{tabular}
\caption{A sound and complete proof system of matching $\mu$ logic}
\label{fig:proof_system}
\end{figure}

\begin{theorem}
The proof system in Figure~\ref{fig:proof_system} is sound and complete
w.r.t. Henkin semantics.
\end{theorem}

\begin{theorem}
The proof system in Figure~\ref{fig:proof_system} is sound but not complete
w.r.t. standard semantics.
\end{theorem}

\section{Recursive Symbols}
\label{sec:recursive_symbols}

The $\mu$-binder in matching $\mu$-logic may not
seem to be powerful enough for some readers, especially
for those who are familiar with first-order logic
with least fixpoints (FOL$_\lfp$).
In FOL$_\lfp$, one can write a formula
\begin{equation}
\label{eq:transitive_closure_FOLlfp_definition}
[\lfp_{R,x,y} (x = y \vee \exists z \ldot ( E(x,z) \wedge R(z,y) ))](u,v)
\end{equation}
with free variables $u$ and $v$, that defines
the reflexive and transitive closure of the predicate $E$.
In~\eqref{eq:transitive_closure_FOLlfp_definition},
the least fixpoint binder $\lfp$ binds all free occurrences of
the predicate variable $R$ and variables $x$ and $y$
in formula
$x = y \vee \exists z \ldot ( E(x,z) \wedge R(z,y) )$.


In this section we show how to define 
recursive symbols using the $\mu$-binder.
We start with unary symbols, but the techniques can be applied
to symbols with any number of arguments. 

\subsection{An important isomorphism}
Let $s, t \in S$ be two sorts,
and $M_s, M_t$ be the corresponding carrier sets.
Our approach is based on the following isomorphism:
\begin{equation}
[M_s \to \pset{M_t}]
\xrightleftharpoons[H]{G}
\pset{M_s \times M_t}
\end{equation}
where the two mappings $G$ and $H$ are defined as follows:
\begin{align}
G(f) &= \{ (a,b) \mid b \in f(a) 
\text{ for all $a \in M_s$ and $b \in M_t$} \}, \\
H(\alpha)(a) &= \{ b \in M_t \mid (a,b) \in \alpha \},
\end{align}
One can verify that $G(H(\alpha)) = \alpha$ and $H(G(f)) = f$.
The subset $G(f)$ is called the \emph{graph} of the function $f$.
This isomorphism tells us that we can use a constant symbol
$\alpha_\sigma \cln s \times t$ to represent a unary symbol
$\sigma (s) \cln t$.

\begin{definition}
Let $s, t \in S$ be two sorts, not necessarily distinct.
Define a new sort $s \times t$ called the \emph{product sort}
of $s$ and $t$.
Two product sorts $s \times t$ and
and $s' \times t'$ are the same if
$s$ is the same sort as $s'$, and $t$ is the same sort as  $t'$.
Define a new symbol $(\_,\_) \in \Sigma_{s t , s \times t}$
called \emph{pairing}, parametric on $s$ and $t$.
Define a new symbol $\_[\_] \in \Sigma_{s \times t s , t}$
called \emph{projection}, parametric on $s$ and $t$.
We often drop the sort subscripts to keep our notation tidy.
We add the following axiom:
\begin{equation}
(x,\varphi)[x] = \varphi
\end{equation}
\end{definition}

\subsection{An example in separation logic}

Consider a signature with sorts $\Nat$ and $\Map$.
We want to define a recursive symbol $\listit(\Nat)\cln \Map$
such that, intuitively,
\begin{equation}
\label{eq:list_lfp_definition}
\listit(x) =_\lfp (x = 0 \wedge \emp) \vee \exists y \ldot 
(x \mapsto y \merge \listit(y)).
\end{equation}
Here, the notion $=_\lfp$ means we want $\listit$ to be the least symbol
that satisfies \eqref{eq:list_lfp_definition}.
Intuitively, we want $\listit(x)$ to be the set of
all maps which consists of a single linked list starting at $x$.
Using product sorts, we can achieve this by defining
the following notation:
\begin{equation}
\label{not:list}
\listit(x) \equiv 
\underbrace{( \mu \alpha \ldot \exists x \ldot 
( (x = 0 \wedge \emp) \vee (\exists y \ldot x \mapsto y \merge \alpha[y] )))
}_{\alpha_\listit}
[x]
\end{equation}
We claim the notation define in~\eqref{not:list} is a ``good'' one,
in the sense that we can indeed regard 
$\listit(x)$ as the pattern which is the ``symbol'' $\listit$ applied
to the argument $x$.
When it comes to semantics, it is indeed interpreted as ``the least symbol''
that satisfies~\eqref{eq:list_lfp_definition}.
In fact:
\begin{align}
\listit(\bot) &= \alpha_\listit[\bot] = \bot,
\\
\listit(\varphi_1 \vee \varphi_2) &= \alpha_\listit[\varphi_1 \vee \varphi_2]
= \alpha_\listit[\varphi_1] \vee \alpha_\listit[\varphi_2],
\\
\listit(\exists x \ldot \varphi) &=
\alpha_\listit[\exists x \ldot \varphi] =
\exists x \ldot \alpha_\listit[\varphi].
\end{align}
So the notation $\listit(x)$ defined in~\eqref{not:list}
indeed satisfies all three propagation rules for symbols.
In addition, the \prule{Framing} rule also holds:
\begin{align}
\text{if $\varphi_1 \imp \varphi_2$ then
$\alpha_\listit[\varphi_1] \imp \alpha_\listit[\varphi_2]$,
which is exactly
$\listit(\varphi_1) \imp \listit(\varphi_2)$.
}
\end{align}
Last but not the least, we can prove the following two rules hold,
which makes it possible to regard $\listit(x)$ as a recursive symbol:
\begin{align}
&\listit(x) = (x = 0 \wedge \emp \merge \exists y \ldot x \mapsto y \merge 
\listit(y)) \\
&\prftree[r]
{ % rule label
if $y \notin \FV(\psi)$
}
{
% premise
(x = 0 \wedge \emp \merge \exists y \ldot x \mapsto y \merge \psi[y/x]) 
\to \psi
}{
% conclusion
\listit(x) \to \psi
}
\end{align}


\subsection{A notation of unary recursive symbols in matching $\mu$-logic}

\subsection{A notation of recursive symbols taking any number of arguments}


\section{Instance: Modal Logic and Its Variants}

In this section, we give a list of important logics and calculi about
program reasoning that can be defined as matching logic theories or notations.
The main point is to convince that matching logic is capable of
serving as a unified logic for program verification,
that allows us to reason about any properties written in any logics,
about any programs written in any programming languages.

\subsection{Hybrid Modal Logic}
\label{sec_hybrid_modal_logic}

As shown in Section~\ref{sec_modal_logic_S5},
the modal logic S5 is definable in matching logic.
In this section, we show that
a first-order extension of modal logic, called hybrid modal logic,
is also definable in matching logic.
In particular, we show that the famous \prule{Barcan} axioms are special cases
of rule \propagationexists for unary symbols.

The syntax of hybrid modal logic contains
a set $\SVar$ of state variables,
a set $\Nom$ of nominals,
and a binder $\forall$ that binds a state variable in a formula:
\begin{center}
	$\varphi \Coloneqq
	p \in \AP \mid x \in \SVar \mid i \in \Nom \mid
	\neg \varphi \mid \varphi \wedge \varphi \mid
	\always \varphi \mid \forall x . \varphi \text{ where $x \in \SVar$}
	$
\end{center}
As in first-order logic,
we define $\exists x . \varphi \equiv \neg (\forall x . \neg \varphi)$.
As in modal logic S5,
we define $\eventually \varphi \equiv \neg (\always \neg \varphi)$.

A hybrid modal logic model
$\MM = (S, R, V)$ is a triple where
$S$ is a nonempty set of states,
$R$ is a binary relation on $S$,
and $V \colon \AP \cup \Nom \to \pset{S}$ is a valuation.
The valuation $V$ is called a standard valuation if $V(i)$ is a singleton set
for every $i \in \Nom$.
A model is standard if its valuation is standard.
An assignment $g \colon \SVar \to \pset{S}$ is called standard
if $g(x)$ is a singleton set for every $x \in \SVar$.
The semantics is defined inductively as follows:
\begin{itemize}
	\item $\MM,g,s \vDash_\hybridModalLogic p$
	if $s \in V(p)$ where $p \in \AP$;
	\item $\MM,g,s \vDash_\hybridModalLogic x$
	if $x \in g(x)$ where $x \in \SVar$;
	\item $\MM,g,s \vDash_\hybridModalLogic i$
	if $i \in V(i)$ where $i \in \Nom$;
	\item $\MM,g,s \vDash_\hybridModalLogic \neg \varphi$
	if $\MM,g,s \not \vDash_\hybridModalLogic \varphi$;
	\item $\MM,g,s \vDash_\hybridModalLogic \varphi_1 \wedge \varphi_2$
	if  $\MM,g,s \vDash_\hybridModalLogic \varphi_1$
	and $\MM,g,s \vDash_\hybridModalLogic \varphi_2$;
	\item $\MM,g,s \vDash_\hybridModalLogic \always \varphi$
	if $\MM,g,s' \vDash_\hybridModalLogic \varphi$
	for every $s'$ such that $s R s'$;
	\item $\MM,g,s \vDash_\hybridModalLogic \forall x . \varphi$
	if $\MM,g',s \vDash_\hybridModalLogic \varphi$
	for every $g'$ such that $g' \simx g$.
\end{itemize}
A hybrid modal logic formula is valid
if $\MM,g,s \vDash_\hybridModalLogic \varphi$ 
holds for any standard model $\MM$, standard assignment $g$, and state $s$
A sound and complete proof system of hybrid modal logic
is proposed in~\cite{}, as shown in the following.
We write $\vdash_\hybridModalLogic \varphi$ if $\varphi$ is provable in hybrid
modal logic.
\begin{center}
	\renewcommand{\arraystretch}{1.15}
	\begin{tabular}{lm{7cm}lm{3cm}}
		\multicolumn{4}{l}{\em Proof system of hybrid modal logic extends
			propositional calculus with the following:}
		\\\hline
		\prule{K}&
		$\always (\varphi \imp \psi) \imp
		(\always \varphi \imp \always \psi)$
		&
		\prule{N}&
		$\prftree{\varphi}{\always \varphi}$
		\\
		\prule{Q$_1$}&
		$\forall x . (\varphi \imp \psi) \imp (\varphi \imp \forall x . \psi)$,
		if $x \not\in \FV(\varphi)$
		&
		\prule{Gen} &
		$\prftree{\varphi}{\forall x . \varphi}$
		\\
		\prule{Q$_2$-Svar} &
		$\forall x . \varphi \imp \varphi[y/x]$
		&
		\prule{Q$_2$-Nom} &
		$\forall x . \varphi \imp \varphi[i/x]$
		\\
		\prule{Barcan} &
		$\forall x . \always \varphi \imp \always \forall x . \varphi$
		\\
		\prule{Nom}&
		$\forall x . (\eventually^m (x \wedge \varphi)
		\imp \always^n(x \imp \varphi))$,
		for every $m,n \in \nats$
		&
		\prule{Name}&
		$\exists x . x$
	\end{tabular}
	\renewcommand{\arraystretch}{1}
\end{center}

We can define a matching logic theory 
$\MLhybridML = (\sig, H)$
that faithfully captures hybrid modal logic.
The matching logic signature $\sig = (\SVar \cup \Nom , \{ \statesort \} ,
\Sigma)$
where the variable set contains all state variables and nominals
and the sort set contains exactly one sort $\statesort$ of states.
The symbol set $\Sigma$ consists of
\begin{itemize}
	\item a unary symbol $\eventually \in \Sigma_{\statesort , \statesort}$;
	Define the derived construct
	$\always \varphi \equiv \neg (\eventually \neg \varphi)$;
	\item a constant symbol $p \in \Sigma_{\lambda,\statesort}$ 
	for every atomic proposition $p \in \AP$;
\end{itemize}
The axiom set $H = \emptyset$.
With the above definitions,
\begin{center}
	\em
	Any hybrid modal logic formula $\varphi$ is a matching logic pattern
	in theory $\MLhybridML$.
\end{center}
In addition, the state/valuation models of hybrid modal logic
are essentially identical to the matching logic models of theory $\MLhybridML$,
as summarized as follows.

\begin{center}
	\begin{tabular}{ll}
		\multicolumn{1}{c}{Hybrid modal logic} & 
		\multicolumn{1}{c}{Matching logic theory $\MLhybridML$}
		\\\hline
		state set $S$ & carrier set $S$
		\\
		binary relation $R \subseteq S \times S$ & interpretation of $\eventually$ such
		that
		$s \in \eventually_\MM (t)$ if and only if $s R t$
		\\
		standard valuation $V \colon \AP \to \pset{S}$
		& interpretation of $p$ such that $p_\MM = V(p)$
		\\
		standard valuation $V \colon \Nom \to \pset{S}$
		& valuation $\rho \colon \Nom \to S$ such that
		$V(i) = \{ \rho(i) \}$
		\\
		standard valuation $g \colon \SVar \to \pset{S}$
		& valuation $\rho \colon \SVar \to S$ such that
		$g(x) = \{ \rho(x) \}$
		\\
		$\MM,g,s \vDash_\hybridModalLogic \varphi$
		& $s \in \rhobar(\varphi)$
	\end{tabular}
\end{center}
Immediately, we have the following conservative extension result for hybrid
modal logic.
\begin{theorem}[Conservative Extension for Hybrid Modal Logic]
	Let $\varphi$ be a hybrid modal logic formula.
	Then
	$\vdash_\hybridModalLogic \varphi$ if and only if
	$\vDash_\hybridModalLogic \varphi$ if and only if 
	$\MLhybridML \vDash \varphi$ if and only if
	$\MLhybridML \vdash \varphi$.
\end{theorem}

We point out that the famous \prule{N} and \prule{Barcan} axioms
in hybrid modal logic
are in fact special cases
of \propagationbottom and \propagationexists rules
in matching logic where the symbol takes exactly one argument.
For any symbol $\sigma \in \Sigma_{s,s}$,
define its complement $\sigmabar (\varphi) \equiv \neg \sigma(\neg \varphi)$.
We can show the following propositions hold for
any pattern set $H$ in matching logic:
\begin{itemize}
	\item $H \vdash \varphi$ implies $H \vdash \sigmabar(\varphi)$;
	\item $\emptyset \vdash \forall x \sigmabar(\varphi) \imp \sigmabar(\forall x .
	\varphi)$.
\end{itemize}

\subsection{Polyadic modal logic}

...


\section{Instance: Modal $\mu$-Logic}
\label{sec_mu_calculus}

Mu-calculus is the extension of modal logic with
induction and fixpoints.
The syntax of mu-calculus is parametric on a set $\Var$ of variables,
a set $\AP$ of atomic propositions,
and a set $\Label$ of labels.
Mu-calculus formulas are defined inductively as follows.
$$
\varphi \Coloneqq
p \in \AP \mid
x \in \Var \mid
\varphi \wedge \varphi \mid
\neg \varphi \mid 
[a] \varphi \mid
\mu x . \varphi \text{ if $x$ does not occur negatively in $\varphi$}
$$
where $a \in \Label$.
As in matching logic, define $\nu x . \varphi \equiv \neg \mu x . (\neg
(\varphi[\neg x / x]))$.
Notice that if $x$ does not occur negatively in $\varphi$,
so does $\neg (\varphi[\neg x / x])$.
Define $\angleBraces{a} \varphi \equiv \neg [a] (\neg \varphi)$.

Mu-calculus formulas are interpreted on structures.
A structure $\MM = (S, R, V)$ consists of
a set $S$ of states,
a family set $R = \{R_a\}_{a \in \Label}$ with a binary relation $R_a \subseteq
S \times S$
for each label $a$,
and a valuation $V \colon \AP \to \pset{S}$.
Given a structure $\MM$ and an assignment $g \colon \Var \to \pset{S}$,
the interpretation function
$\bracket{\_}_{\MM,g}$ is defined inductively as follows.
\begin{itemize}
	\item $\bracket{p}_{\MM,g} = V(p)$;
	\item $\bracket{x}_{\MM,g} = g(x)$;
	\item $\bracket{\varphi_1 \wedge \varphi_2}_{\MM,g} =
	\bracket{\varphi_1}_{\MM,g} \cap
	\bracket{\varphi_2}_{\MM,g}$;
	\item $\bracket{\neg \varphi}_{\MM,g} =
	S \setminus \bracket{\varphi}_{\MM,g}$;    
	\item $\bracket{[a]\varphi}_{\MM,g} =
	\{ s \mid \text{for all $t$ such that $s R_a t$, 
		$t \in \bracket{\varphi}_{\MM,g}$} \}$;
	\item $\bracket{\mu x . \varphi}_{\MM,g} =
	\bigcap \{ X \subseteq S \mid \bracket{\varphi}_{\MM,g'} \subseteq X
	\text{ where $g' \simx g$ and $g(x) = X$} \}$.
\end{itemize}
A mu-calculus formula $\varphi$ is valid, written as $\vDash_\mu \varphi$,
if $\bracket{\varphi}_{\MM,g} = S$ for every structure $\MM = (S,R,V)$ and
assignment $g$.
Sound and complete deduction for mu-calculus remained open for more than a
decade,
until Walukiewicz showed in~\cite{bibid} that
the following proof system, initially given by Kozen~\cite{bibid}, 
is indeed complete.
We write $\vdash_\mu \varphi$ if $\varphi$ is provable in mu-calculus.
\begin{center}
	\begin{tabular}{lllllm{2cm}}
		\multicolumn{6}{l}{\em Proof system of mu-calculus extends
			propositional calculus with the following:}
		\\\hline
		\prule{K}&
		$[a](\varphi \imp \psi) \imp
		([a]\varphi \imp [a]\psi)$
		&
		\prule{Mu$_1$}&
		$\varphi[(\mu x . \varphi) / x] \imp \mu x . \varphi$
		&
		\prule{Mu$_2$}&
		$\prftree{\varphi[\psi / x] \imp \psi}{\mu x . \varphi \imp \psi}$
	\end{tabular}
\end{center}

The way mu-calculus is defined in matching logic is similar
as other modal logics.
The theory $\MLMu = (\sig, H)$ is a single-sorted theory and
contains all definitions needed for 
the fixpoint constructs $\mu$ and $\nu$.
The variable set is $\Var$.
The signature set $\Sigma$ contains
\begin{itemize}
	\item a unary symbol $a$ for every label $a \in \Label$; 
	\item a constant symbol $p$ for every atomic proposition $p \in \AP$;
	\item a constant symbol $x$ for every variable $x \in \Var$.
\end{itemize}
Notice that we have both the variable $x$ and the symbol $x$ in matching logic.
This is because in mu-calculus, a variable is either a standalone formula or
the binding variable of a fixpoint construct.
Mu-calculus assignments assign variables to sets, while matching logic
valuations
map variables to values (singleton sets).
Therefore, we use the matching logic symbol $x$ if it is a standalone formula in
mu-calculus,
and use the variable $x$ if it is a binding variable in $\mu x . \varphi$ or
$\nu x . \varphi$.
In addition, we define $\angleBraces{a}\varphi \equiv a(\varphi)$
and $[a]\varphi \equiv \neg a(\neg \varphi)$.
With the above definitions and notations,
\begin{center}
	\em
	Any mu-calculus formula $\varphi$ is a closed matching logic pattern
	in theory $\MLMu$.
\end{center}

There is a one-to-one correspondence between
mu-calculus structures and the intended models of theory $\MLMu$,
as summarized in the following.
Since all mu-calculus formulas are closed,
it does not matter what matching logic valuation we use.
\begin{center}
	\begin{tabular}{ll}
		\multicolumn{1}{c}{Mu-calculus} & 
		\multicolumn{1}{c}{Matching logic theory $\MLMu$ with intended semantics}
		\\\hline
		state set $S$ & carrier set $S$
		\\
		binary relation $R_a \subseteq S \times S$ & interpretation of $a$ such that
		$s \in a_\MM (t)$ if and only if $s R_a t$
		\\
		valuation $V \colon \AP \to \pset{S}$
		& interpretation of $p$ such that $p_\MM = V(p)$
		\\
		assignment $g \colon \Var \to \pset{S}$
		& interpretation of $x$ such that $x_\MM = g(x)$
		\\
		least fixpoint $\bracket{\mu x . \varphi}_{\MM,g}$
		&
		intended interpretation $\rhobar(\mu x . \varphi) = \mu
		\bracket{\varphi}_{\MM,\rho}$
		\\&
		where $\rho$ is any valuation (it makes no difference which $\rho$ we use)
		\\
		$s \in \bracket{\varphi}_{\MM,g}$
		& $s \in \rhobar(\varphi)$
	\end{tabular}
\end{center}

Like hybrid modal logic (see Section~\ref{sec_hybrid_modal_logic}), 
mu-calculus and the theory $\MLMu$
admit the conservative extension result.
Unlike hybrid modal logic, the proof involves
both modal-theoretic and proof-theoretic approaches.
The reason is that the above one-to-one correspondence only works for
the \emph{intended models}, not \emph{all models}, of theory $\MLMu$.
Therefore, we can conclude only
that $\MLMu \vDash \varphi$ implies $\vDash_\mu \varphi$,
but not the other direction, as there may exist an unintended model, say $\MM'$,which satisfies $\MLMu$ and fails $\varphi$.
Such unintended models $\MM'$ are not considered at all in mu-calculus.

A proof-theoretic approach fills the gap.
Careful readers may already notice that 
all axioms and rules in mu-calculus are provable in matching logic.
Axiom \prule{K} is provable as shown in Section~\ref{sec_hybrid_modal_logic}.
Axiom \prule{Mu$_1$} is proved using \Fixmu,
and rule \prule{Mu$_2$} is proved using \Lfp.
Therefore, we conclude that $\vdash_\mu \varphi$ implies $\MLMu \vdash \varphi$,because we can mimic every mu-calculus proofs in the matching logic theory
$\MLMu$.
The conservative extension for mu-calculus is then obtained by 
completeness of both mu-calculus and matching logic.

We point out that the above reasoning,
as illustrated in Figure~\ref{fig_general_method_conservative_extension},
is very general.
In the following subsections, 
we will use the same technique to prove
the conservative extension results for LTL, CTL, CTL*, etc,
so it is important to understand what is needed for the proof.
As shown in Figure~\ref{fig_general_method_conservative_extension},
the diagram involves 7 steps, with 4 of them are either proved by definition,
or established results about matching logic.
Among the rest 3 steps,
Step ($\Longrightarrow_2$) requires to show all axioms and proof rules of
mu-calculus
are provable in theory $\MLMu$, 
which is not trivial and can involve some intelligence.
Step ($\Longrightarrow_6$) requires to show all mu-calculus models can be
regarded as
matching logic models (often the intended models) of theory $\MLMu$;
the proof is often by carrying out structural induction on formulas,
and thus we often need to prove both directions.
Notice that there is a game we can play when defining theory $\MLMu$.
We can add more axioms to $\MLMu$, which makes Step ($\Longrightarrow_2$) easier
to prove
but Step ($\Longrightarrow_6$) harder, as it rules out more models.
If we do not have enough axioms, Step ($\Longrightarrow_2$) may not be provable.Finally, we rely on the completeness of mu-calculus (Step
($\Longrightarrow_1$)),
which is also an established result but far from trivial.


\begin{figure}
	\begin{tabular}{|ccccc|}
		\hline
		$\vDash_\mu \varphi$ &
		$\Longrightarrow_1$ &
		$\vdash_\mu \varphi$ &
		$\Longrightarrow_2$ &
		$\MLMu \vdash \varphi$ \\
		\multirow{4}{*}{$\Bigg\Uparrow_7$}
		&&
		\multirow{4}{*}{\scalebox{3}{$\circlearrowright$}}
		&&$\bigg\Downarrow_3$ \\
		&&&&$\MLMu \vDash \varphi$ \\
		&&&&$\bigg\Downarrow_4$ \\
		$\MM \vDash_\mu \varphi$ &
		\multirow{3}{*}{$\Longleftarrow_6$} &
		$\MM \vDash \varphi$ &
		\multirow{3}{*}{$\Longleftarrow_5$} &
		$\MM \vDash \varphi$
		\\
		for all mu-calculus &&
		for all intended models $\MM$ &&
		for all models $\MM$
		\\
		models $\MM$
		&&
		such that $\MM \vDash \MLMu$
		&&
		such that $\MM \vDash \MLMu$
		\\ &&&&
		\\
		\multicolumn{5}{|l|}{
			($\Longrightarrow_1$)\ :\quad completeness of mu-calculus (nontrivial, known
			result)
		}\\
		\multicolumn{5}{|l|}{
			($\Longrightarrow_2$)\ :\quad prove all mu-calculus axioms and rules
			(nontrivial)
		}\\
		\multicolumn{5}{|l|}{
			($\Longrightarrow_3$)\ :\quad soundness of matching logic
		}\\
		\multicolumn{5}{|l|}{
			($\Longrightarrow_4$)\ :\quad definition
		}\\
		\multicolumn{5}{|l|}{
			($\Longrightarrow_5$)\ :\quad obvious
		}\\
		\multicolumn{5}{|l|}{
			($\Longrightarrow_6$)\ :\quad correspondence between mu-calculus models and
			intended models (nontrivial)
		}\\
		\multicolumn{5}{|l|}{
			($\Longrightarrow_7$)\ :\quad definition
		}
		\\\hline
	\end{tabular}
	\caption{
		A general method to prove conservative extension, using mu-calculus as an
		example.}
	\label{fig_general_method_conservative_extension}
\end{figure}

\begin{theorem}[Conservative extension for mu-calculus]
	
	Let $\varphi$ be a $mu$-calculus formula, and $\MLMu$ be the matching logic
	theory
	for mu-calculus defined as above.
	Then, $\vdash_\mu \varphi$ if and only if $\MLMu \vdash \varphi$.
\end{theorem}

\section{Instance: Transition Systems}

Transition systems are important in computer science.
They are used to model
various types of hardware and software systems,
and many algorithms and techniques are proposed to analyze transition systems
and to reason about their properties.
In mathematics, a transition system $\MM = (M, A, \{\xto{a}\}_{a \in A})$ is
a carrier set $M$ and a label set $A$, 
accompanied with a binary relation $\xto{a} \subseteq M \times M$
for every label $a$.
When $A$ contains exactly one label, we write $\MM = (M, \to)$ 
and call the transition system \emph{unlabeled}.

This subsection is dedicated to defining transition systems in matching logic.

\subsection{Strong next and weak next}

Let us first consider unlabeled transition systems.
Given an unlabeled transition system $\MM = (M,\to)$.
There are two ways to capture the transition relation $\to$.
One is to define a predecessor function $\pred \colon M \to \pset{M}$
such that $\pred(t) = \{s \mid s \to t\}$.
The other is to define a successor function $\succc \colon M\to \pset{M}$
such that $\succc(s) = \{ t \mid s \to t \}$.
It is easy to show that $\succc(s) = \{ t \mid s \in \pred(t) \}$.
Define a unary matching logic symbol $\snext$
and interpret it to the predecessor function $\snext_\MM(t) = \pred(t)$.
We write $\snext \varphi$ instead of $\snext(\varphi)$.
Define a derived construct
$\sprev(\varphi) \equiv \exists x . (x \wedge \floor{\varphi \imp
	\snext(\varphi)})$,
and one can prove that $\sprev$ is interpreted to the successor function
$\sprev_\MM(s) = \succc(s)$.
We call the symbol $\snext$ ``strong next'' and the construct $\sprev$ ``strong
previous'',
and we will see why.
Since ``strong next'' and ``strong previous'' are similar, let us only discuss
the ``strong next'' $\snext$.

Assume $\rho$ is any valuation
and $\rhobar \colon \Pattern \to \pset{M}$ is the corresponding extended
valuation
that interprets patterns to sets.
Let $s \in M$ and $\varphi$ be a pattern.
The symbol $\snext$ is called ``strong next'', because
$s \in \rhobar(\snext \varphi)$ if and only if \emph{there exists} $t \in M$
such that
$s \to t$ and $t \in \rhobar(\varphi)$.
In other words, $\snext \varphi$ holds in the state $s$ if
there exists a next state $t$ in which $\varphi$ holds.
We can define a dual construct $\wnext \varphi \equiv \neg \snext \neg \varphi$
called the ``weak next'', and show that
$s \in \rhobar(\wnext \varphi)$ if and only if
\emph{for all} $t \in M$ such that $s \to t$, $t \in \rhobar(\varphi)$.
In other words, $\wnext \varphi$ holds in the state $s$ if
for all next states $t$, $\varphi$ holds in $t$.
In particular, if $s$ has no next state, $\wnext \varphi$ holds in $s$
ly.
Figure~\ref{fig_snext_wnext} illustrates the meaning of $\snext \varphi$ and
$\wnext \varphi$.
\begin{figure}
	\begin{tikzpicture}
	\node at (2,1.2)[circle,draw,inner sep=1.1cm,label=left:$\wnext \varphi$] 
	{};
	\node at (2,2.8)[circle,draw,inner sep=1.1cm,label=left:$\snext \varphi$] 
	{};
	\node at (7,2)  [circle,draw,inner sep=1.1cm,label=right:$\varphi$] {};
	\node (down) at (2,0.8) [circle,fill=black,inner sep=0.6mm] {};
	\node (middle) at (2,2) [circle,fill=black,inner sep=0.6mm] {};
	\node (up) at (2,3.2) [circle,fill=black,inner sep=0.6mm] {};
	\node (r0) at (7,4) [circle,fill=black,inner sep=0.6mm] {};
	\node (r1) at (7,2.8) [circle,fill=black,inner sep=0.6mm] {};
	\node (r2) at (7,2) [circle,fill=black,inner sep=0.6mm] {};
	\node (r3) at (7,1.2) [circle,fill=black,inner sep=0.6mm] {};
	\draw [->] (middle) to (r2);
	\draw [->] (middle) to (r3);
	\draw [->] (up) to (r0);
	\draw [->] (up) to (r1);
	\draw [->] (up) to (r2);
	\end{tikzpicture}
	\caption{Strong next $\snext \varphi$ and weak next $\wnext \varphi$.}
	\label{fig_snext_wnext}
\end{figure}


\todochen{
	Add a diagram here showing why ``strong next'' $\snext$
	is interpreted as the ``predecessor function'' in models. 
}


\subsection{Constructs defined using fixpoints}

``Strong next'' $\snext$ together with fixpoint constructs $\mu$ and $\nu$
provide great expressive power about transition systems.
The following table summarizes a few important patterns and constructs about
transition systems
that are used later in the paper.
Notice that the right column presents the \emph{intended interpretation} of
these
patterns and constructs, where $\mu$ and $\nu$ are interpreted as true lfps and
gfps.
\begin{center}
	\small
	\begin{tabular}{ll}
		Matching logic          &  Intended semantics in the transition system $\MM$ \\
		patterns and constructs & Necessary and sufficient condition of $s \in
		\rhobar(\text{lhs})$
		for some state $s \in M$
		\\\hline
		$\snext \varphi$ &
		there exists a state $t$ such that $s \to t$ and $t \in \rhobar(\varphi)$
		\\
		$\wnext \varphi$ &
		for all states $t$ such that $s \to t$, $t \in \rhobar(\varphi)$
		\\
		$\snext \top$ &
		$s$ is a non-terminating state (has a next state)
		\\
		$\wnext \bot$ &
		$s$ is a terminating state (has no next state)
		\\
		$\mu f . \wnext f$ &
		all paths starting at $s$ are finite 
		\\
		$\eventually \varphi \equiv \mu f . (\varphi \vee \snext f)$ &
		there exists $n \ge 0$ and $t_1 ,\dots, t_n$ such that 
		$s \to t_1 \to \dots \to t_n$ and $t_n \in \rhobar(\varphi)$
		\\
		$\always \varphi \equiv \nu f . (\varphi \wedge \wnext f)$ &
		for all $n \ge 0$ and $t_1 ,\dots, t_n$ such that
		$s \to t_1 \to \dots \to t_n$, $t_n \in \rhobar(\varphi)$
		\\
		$\mu f . (\varphi_1 \vee (\varphi_2 \wedge \snext f))$ &
		there exists $n \ge 0$ and $t_1 ,\dots, t_n$ such that 
		$s \to t_1 \to \dots \to t_n$, 
		$t_n \in \rhobar(\varphi_2)$
		\\& and $s,t_1 ,\dots, t_{n-1} \in \rhobar(\varphi_1)$
		\\
		$\mu f . (\varphi_1 \vee (\varphi_2 \wedge \wnext f))$ &
		for all $n \ge 0$ and $t_1 ,\dots, t_n$ such that
		$s \to t_1 \to \dots \to t_n$, 
		there exists $m \le n$ 
		\\&such that $t_n \in \rhobar(\varphi)$
		$t_m \in \rhobar(\varphi_2)$ and $s,t_1 ,\dots, t_{m-1} \in 
		\rhobar(\varphi_1)$
	\end{tabular}
\end{center}

Using these constructs, we can write axioms to specify properties of transition
systems
and rule out those which do not satisfy the properties.
These axioms allow us to capture various interesting types of transition
systems,
which we summarize in the next table.
Notice that we do not claim these axiomatizations are \emph{complete},
because of the non-standard models of fixpoint constructs.
However, we will show in the following subsections that these axioms
\emph{completely} capture various important logics for transition systems,
including mu-calculus, linear temporal logic (LTL), computation tree logic
(CTL),
and CTL*.
\begin{center}
	\begin{tabular}{ll}
		Matching logic axioms & Transition system $\MM$ in the intended semantics
		\\\hline
		\Inf $\snext \top$ &
		Every state in $\MM$ has a next state
		\\
		\Fin $\mu f . \wnext f$ &
		$\MM$ has no infinite trace
		\\
		\Lin $\snext \varphi \imp \wnext \varphi$ &
		Every state in $\MM$, if it has next states, has a unique next state;
		\\& In other words, every state has a linear future
	\end{tabular}
\end{center} 

Labeled transition systems can be captured in the similar way.
Given a labeled transition system $\MM = (M, A, \{ \xto{a} \}_{a \in A})$.
Define a matching logic signature $\sig$ which contains a unary symbol
$\snext_a$
for every label $a \in A$.

\section{Instance: Linear Temporal Logic}
\label{sec_LTL}


Linear temporal logic (LTL) is parametric 
on a set $\AP$ of atomic propositions.
In literature, the term LTL often refers to the \emph{infinite-trace LTL},
where LTL formulas are interpreted on infinite traces.
In program verification and especially runtime verification~\cite{bibid},
finite execution traces play an important role, and thus
\emph{finite-trace LTL} is considered.
In this section, we will show how
both finite- and infinite-trace LTL can be defined in a uniform way in matching
logic.

\subsubsection{Infinite-trace linear temporal logic}

Readers should be more familiar with infinite-trace LTL, 
so let us consider that first.
The syntax of infinite-trace LTL 
extends the syntax of propositional calculus
with a ``next'' modality $\wnext$ 
and a ``strong until'' modality $\mathsf{U}_s$:
$$
\varphi \Coloneqq p \in \AP
\mid \varphi \wedge \varphi
\mid \neg \varphi
\mid \wnext \varphi
\mid \varphi \Us \varphi
$$
As usual,
we define $\eventually \varphi \equiv \true \Us \varphi$
and $\always \varphi \equiv \neg (\eventually \neg \varphi)$.
Infinite-trace LTL formulas are interpreted on 
\emph{infinite traces of sets of atomic propositions},
denoted as $\infTraces = [ \nats \to \pset{\AP}]$.
We use $\alpha = \alpha_0\alpha_1\dots$
to denote an infinite trace,
and use the conventional notation
$\alpha_{\ge i}$ to denote the suffix trace
$\alpha_i \alpha_{i+1} \dots$.
Infinite-trace LTL semantics $\alpha \vDash_\infLTL \varphi$ is defined 
inductively as follows:
\begin{itemize}
	\item $\alpha \vDash_\infLTL p$ if $p \in \alpha_0$ for atomic proposition $p$;
	\item $\alpha \vDash_\infLTL \varphi_1 \wedge \varphi_2$
	if $\alpha \vDash_\infLTL \varphi_1$ and $\alpha \vDash \varphi_2$;
	\item $\alpha \vDash_\infLTL \neg \varphi$
	if $\alpha \not\vDash_\infLTL \varphi$;
	\item $\alpha  \vDash_\infLTL \wnext \varphi$
	if $\alpha_{\ge 1} \vDash_\infLTL \varphi$;
	\item $\alpha \vDash_\infLTL \varphi_1 \Us \varphi_2$
	if there is $j \ge 0$ such that
	$\alpha_{\ge j} \vDash_\infLTL \varphi_2$ and for every $i < j$,
	$\alpha_{\ge i} \vDash_\infLTL \varphi_1$.
\end{itemize}
An infinite-trace LTL formula $\varphi$ is valid,
denoted as $\vDash_\infLTL \varphi$,
if $\alpha \vDash_\infLTL \varphi$ 
for every $\alpha \in \infTraces$.
A sound and complete proof system of infinite-trace LTL
is given as follows.
We write $\vdash_\infLTL \varphi$
if an infinite-trace LTL formula $\varphi$ is provable.
\begin{center}
	\begin{tabular}{lm{6cm}lm{3.8cm}}
		\multicolumn{4}{l}{
			\em
			Proof system of infinite-trace LTL extends propositional calculus with the
			following:
		}
		\\\hline
		\prule{K$_\wnext$}
		&
		$\wnext (\varphi_1 \imp \varphi_2) \imp (\wnext \varphi_1 \imp \wnext 
		\varphi_2)$
		&
		\prule{N$_\wnext$}
		&
		$\prftree{\varphi}{\wnext \varphi}$
		\\
		\prule{K$_\always$}
		&
		$\always (\varphi_1 \imp \varphi_2) \imp (\always \varphi_1 \imp \always 
		\varphi_2)$
		&
		\prule{N$_\always$}
		&
		$\prftree{\varphi}{\always \varphi}$
		\\
		\prule{Fun}
		&
		$\wnext \varphi \dimp \neg (\wnext \neg \varphi)$
		&
		\prule{U$_1$}
		&
		$(\varphi_1 \Us \varphi_2) \imp \eventually \varphi_2$
		\\
		\prule{U$_2$}
		&
		$(\varphi_1 \Us \varphi_2) 
		\dimp 
		(\varphi_2 \vee (\varphi_1 \wedge \wnext (\varphi_1 \Us \varphi_2)))$
		&
		\prule{Ind}
		&
		$\always(\varphi \imp \wnext \varphi) \imp (\varphi \imp \always \varphi)$
	\end{tabular}
\end{center}

We can define 
a single-sorted matching logic theory 
$\MLinfLTL = (\sig, H)$ that
faithfully captures infinite-trace LTL.
The theory $\MLinfLTL$ contains all definitions
that are needed for the fixpoint constructs $\mu$ and $\nu$.
The signature $\Sigma$ contains
\begin{itemize}
	\item a unary symbol $\snext$ called ``strong-next'';
	We write $\snext \varphi$ instead of $\snext(\varphi)$;
	\item a constant symbol $p$ for every atomic proposition $p \in \AP$.
\end{itemize}
We define the following derived matching logic constructs:
\begin{align*}
\wnext \varphi \equiv \neg (\snext \neg \varphi)
&&
\always \varphi \equiv \nu f . (\varphi \wedge \wnext f)
&&
\eventually \varphi \equiv \mu f . (\varphi \vee \snext f)
&&
\varphi_1 \Us \varphi_2 \equiv \mu f . (\varphi_2 \vee (\varphi_1 \wedge \snext
f))
\end{align*}
In addition to axioms about fixpoint constructs, the axiom set $H$ contains two
more axioms
\begin{align*}
\prule{Lin} \quad \snext \varphi \imp \wnext \varphi
&&
\prule{Inf} \quad \snext \top
\end{align*}
As we have seen in Section~\ref{sec_transition_systems},
axiom \Lin and \Inf give us linear and infinite traces, respectively. 
Notice that $\always \varphi$ is not defined as a matching logic symbol, 
but defined using the fixpoint construct~$\nu$.
By simple matching logic reasoning, we can show that
$
\always \varphi = \neg \eventually \neg \varphi
$
and 
$
\eventually \varphi = \top \Us \varphi
$, as expected.
Thanks to the above definitions and notations,
\begin{center}
	\emph{Any infinite-trace linear temporal logic formula is a pattern
		in theory $\MLinfLTL$.}
\end{center}
%
%\begin{table}[bpht]
%\begin{tabular}{lll}
%Infinite-trace LTL & ML encoding & Meaning 
%\\\hline
%$p, q, r,\dots$ & $p, q, r, \dots$ & atomic propositions
%\\\hline
%$\varphi_1 \wedge \varphi_2$ & $\varphi_1 \wedge \varphi_2$ & conjunction
%\\\hline
%$\neg \varphi$ & $\neg \varphi$ & negation
%\\\hline
%$\wnext \varphi$ & $\wnext \varphi$ & next
%\\\hline
%$\varphi_1 \Us \varphi_2$ & $\varphi_1 \Us \varphi_2$ & strong-until
%\end{tabular}
%\caption{Matching logic encoding of infinite-trace LTL}
%\label{tab_infLTL_in_ML}
%\end{table}

%A conservative extension result for infinite-trace LTL can be
%proved using the same technique as shown in
Figure~\ref{fig_general_method_conservative_extension},
%by showing that
%(1) all infinite-trace LTL rules are provable in theory $\MLinfLTL$;
%and (2) there exists a \emph{standard model} of $\MLinfLTL$
%such that for any infinite-trace LTL formula $\varphi$,
%$\varphi$ is valid if and only if it holds in the standard model.
%


As in Section~\ref{sec_mu_calculus}, we show
that
$\vdash_\infLTL \varphi$ implies $\MLinfLTL \vdash \varphi$ using the
proof-theoretic approach,
and that
$\MLinfLTL \vDash \varphi$ implies $\vDash_\infLTL\varphi$ using the
model-theoretic approach,
and let the completeness of both logics do the rest.
%Careful readers may already notice that the two axioms \Lin and \Inf in theory
$\MLinfLTL$
%are exactly the axiom $\prule{Fun}$ in infinite-trace LTL.
%We leave it to readers to show the rest infinite-trace LTL axioms and rules are
also provable in theory $\MLinfLTL$.
%Thus, $\vdash_\infLTL \varphi$ implies $\MLinfLTL \vdash \varphi$.
Here, we only show the model-theoretic part.
We define a matching logic model of theory $\MLinfLTL$ called the \emph{standard
	model},
denoted as $\MM$, whose carrier set is the set $\infTraces$ of all infinite
traces.
It adopts the intended interpretation for fixpoint constructs.
Other symbols and derived constructs in $\MLinfLTL$ are interpreted as follows:
\begin{itemize}
	\item $p_\MM = \{ \alpha \mid p \in \alpha_0 \}$ for every atomic proposition $p
	\in \AP$;
	\item $\alpha \in \snext_\MM(\beta)$ if $\beta = \alpha_{\ge 1}$, i.e.,
	$\beta$ is the immediate surfix of $\alpha$;
	%\item $\wnext(\beta) = \snext(\beta)$ for all infinite trace $\beta$;
	%\item $\alpha \in \always_\MM(\beta)$ if for all $i \ge 0$, $\alpha_i = \beta$;%\item $\alpha \in \eventually_\MM(\beta)$ if there exists $i \ge 0$ such that 
	%      $\alpha_i = \beta$;
	%\item $\alpha \in \Us_\MM(\beta, \gamma)$ if there exists $i \ge $
\end{itemize}
Notice that the standard model $\MM$ satisfies \Lin and \Fin, and thus is indeed
a model
of theory $\MLinfLTL$.
Let $\rho$ be any matching logic valuation.
By structural induction on infinite-trace LTL formulas,
one can show that for any infinite-trace LTL formula $\varphi$ and an infinite
trace $\alpha$,
$\alpha \vDash_\infLTL \varphi$ if and only if $\alpha \in \rhobar(\varphi)$.
And thus $\vDash_\infLTL \varphi$ if and only if $\MM \vDash \varphi$.
This finishes the reason diagram in
Figure~\ref{fig_general_method_conservative_extension},
and we conclude the conservative extension result for infinite-trace LTL.

\begin{theorem}[Conservative Extension for Infinite-trace Linear Temporal Logic]Let $\varphi$ be an infinite-trace LTL formula and $\MLinfLTL$ be the matching
	logic theory for infinite-trace LTL.
	Then,
	$\vdash_\infLTL \varphi$ if and only if
	$\MLinfLTL \vdash \varphi$.
	\label{thm_csrvext_infLTL}
\end{theorem}

\subsubsection{Finite-trace linear temporal logic}
Like infinite-trace LTL,
finite-trace LTL formulas are interpreted on linear structures, i.e., traces.
Unlike infinite-trace LTL, finite-trace LTL formulas
are interpreted on finite traces.
The syntax of finite-trace LTL is defined as follows:
$$
\varphi \Coloneqq
p \in \AP \mid
\varphi \wedge \varphi \mid
\neg \varphi \mid
\wnext \varphi \mid
\varphi \Uw \varphi
$$
As usual,
define $\snext \varphi \equiv \neg \wnext \neg \varphi$,
$\always \varphi \equiv \varphi \Uw \false$, and
$\eventually \varphi \equiv \neg (\always \neg \varphi)$.
Notice that we work with ``weak until'' $\Uw$, different from infinite-trace
LTL.
As a result, in finite-trace LTL ``always'' $\always$ is firstly defined,
followed by ``eventually'' $\eventually$.
The two until's are different in that
$\varphi_1 \Us \varphi_2$ requires $\varphi_2$ eventually holds while
$\varphi_1 \Uw \varphi_2$ does not.
We refer readers to~\cite{bibid} for a more detailed discussion about
finite-trace LTL and its relation with infinite-trace LTL.

Finite-trace LTL formulas are interpreted on nonempty finite traces
of sets of atomic propositions, denoted as
$\alpha = \alpha_0 \dots \alpha_n$.
We write $\finTraces$ to denote the set of all finite traces.
The semantics $\alpha \vDash_\finLTL \varphi$ is defined
similar to infinite-trace LTL.
Notice $\wnext \varphi$ holds in any singleton traces.
\begin{itemize}
	\item $\alpha_0 \dots \alpha_n 
	\vDash_\finLTL p$ if $p \in \alpha_0$ for atomic proposition $p$;
	\item $\alpha_0 \dots \alpha_n  \vDash_\finLTL \varphi_1 \wedge \varphi_2$
	if $\alpha_0 \dots \alpha_n  \vDash_\finLTL \varphi_1$ and 
	$\alpha_0 \dots \alpha_n  \vDash \varphi_2$;
	\item $\alpha_0 \dots \alpha_n  \vDash_\finLTL \neg \varphi$
	if $\alpha_0 \dots \alpha_n  \not\vDash_\finLTL \varphi$;
	\item $\alpha_0 \dots \alpha_n \vDash_\finLTL \wnext \varphi$
	if $n = 0$ or $\alpha_1 \dots \alpha_n \vDash_\finLTL \varphi$;
	\item $\alpha_0 \dots \alpha_n \vDash_\finLTL \varphi_1 \Uw \varphi_2$
	if either for every $i \le n$,
	$s_i \dots s_n \vDash_\finLTL \varphi_1$,
	or there is $j \le n$ such that
	$\alpha_j \dots \alpha_n \vDash_\finLTL \varphi_2$ and for every $i < j$,
	$\alpha_i \dots \alpha_n \vDash_\finLTL \varphi_1$.
\end{itemize}
Finite-trace LTL has a sound and complete proof system.
\begin{center}
	\begin{tabular}{lm{6cm}lm{3cm}}
		\multicolumn{4}{l}{
			\em
			Proof system of finite-trace LTL extends propositional calculus with the
			following:
		}
		\\\hline
		\prule{K$_\wnext$}
		&
		$\wnext (\varphi_1 \imp \varphi_2) \imp (\wnext \varphi_1 \imp \wnext 
		\varphi_2)$
		&
		\prule{N$_\wnext$}
		&
		$\prftree{\varphi}{\wnext \varphi}$
		\\
		\prule{K$_\always$}
		&
		$\always (\varphi_1 \imp \varphi_2) \imp (\always \varphi_1 \imp \always 
		\varphi_2)$
		&
		\prule{N$_\always$}
		&
		$\prftree{\varphi}{\always \varphi}$
		\\
		\prule{$\neg \wnext$}
		&
		$\neg \wnext \varphi \imp \wnext \neg \varphi$
		&
		\prule{coInd}
		&
		$\prftree{\wnext \varphi \imp \varphi}{\varphi}
		$
		\\
		\prule{Fix}
		&
		$(\varphi_1 \Uw \varphi_2) 
		\dimp 
		(\varphi_2 \vee (\varphi_1 \wedge \wnext (\varphi_1 \Uw \varphi_2)))$
	\end{tabular}
\end{center}

A matching logic theory $\MLfinLTL = (\sig, H)$ that captures finite-trace LTL
can be defined similarly as theory $\infLTL$,
while instead of axiom $\Inf$, we add axiom $\Fin$ to capture the finite-trace
semantics.
A conservative extension result is proved in the same way;
the standard model of theory $\MLfinLTL$ has $\infTraces$ as its carrier set.

We point out that the defining proof rule \prule{coInd} in finite-trace LTL
is provable from $\Fin$.
In fact, we have 
$\vdash \floor{\wnext \varphi \imp \varphi} \imp (\mu f . \wnext f \imp
\varphi)$
by axiom \Lfp,
and the rest is by simple matching logic reasoning.

We end this subsection by stating the conservative extension result for
finite-trace LTL.

\begin{theorem}[Conservative Extension for Finite-trace Linear Temporal Logic]
	\label{thm_csrvext_finLTL}
	Let  $\varphi$ be a finite-trace LTL formula
	and $\MLfinLTL$ be the matching logic theory for finite-trace LTL defined as
	above.
	Then,  $\vdash_\finLTL \varphi$ if and only if
	$\MLfinLTL \vdash \varphi$.
\end{theorem}

\subsubsection{Unifying infinite- and finite-trace linear temporal logics}

We propose a matching logic theory $\MLLTL = (\sig, H)$ that unifies both
infinite- and finite-trace LTLs.
The signature $\sig$ contains a unary symbol $\snext$ for ``strong next'',
and conventional temporal modalities are defined in their usual way.
The axiom set $H$ contains \Lin to capture the ``linear trace'' semantics
of both LTLs, and one can always add axioms, \Inf or \Fin, to obtain
infinite- or finite-trace LTL.
Therefore, the matching logic theory $\MLLTL$ provides a unified, flexible, and
extensible
way to reason about properties about linear structures.
Instead of designing new logics for every subclasses linear structures of
interest,
we can use a fixed logic (the matching logic), and 
write axioms to restrict models and structures.
We will see more examples in the following subsections.

\section{Instance: Computation Tree Logic}
\label{sec_CTL}

Computation tree logic (CTL) is another popular logic
to reason about properties of transition systems.
Unlike LTL, CTL is a \emph{branching time} logic.
It interprets its formulas on infinite trees
and has modalities that can quantify paths in a tree.
The syntax of CTL is parametric on a set $\AP$ of atomic propositions and
is defined as follows.
$$
\varphi \Coloneqq p \in \AP \mid
\varphi \wedge \varphi \mid
\neg \varphi \mid
\AX \varphi \mid
\EX \varphi \mid
\varphi \AU \varphi \mid
\varphi \EU \varphi
$$
Other CTL modalities are defined in the usual way:
\begin{align*}
\EF \varphi \equiv \true \EU \varphi &&
\AG \varphi \equiv \neg \EF \neg \varphi &&
\AF \varphi \equiv \true \AU \varphi &&
\EG \varphi \equiv \neg \AG \neg \varphi
\end{align*}
Every CTL modality contains two letters;
the first means either ``all-path'' $\AAA$ or ``one-path'' $\EEE$,
and the second means ``next'' $\XX$, ``until'' $\UU$, ``always'' $\GG$,
or ``eventually'' $\FF$.
Therefore, $\AX$ is ``all-path next'', and $\EU$ is ``one-path until'', etc.

Let $\infTrees$ be the set of all infinite trees over $\AP$.
An infinite tree $\tau$ has sets of $\AP$ as its nodes;
it is \emph{infinite} in the sense that it has no leaves
and every node has children.
We write $\rt(\tau)$ to denote the root of $\tau$.
We write $\tau \to \tau'$ 
if $\tau'$ is an immediate subtree of $\tau$.
CTL semantics $\tau \vDash_\CTL \varphi$ is defined inductively as follows.
\begin{itemize}
	\item $\tau \vDash_\CTL p$ if $p \in \rt(\tau)$ for atomic proposition $p \in
	\AP$;
	\item $\tau \vDash_\CTL \varphi_1 \wedge \varphi_2$ if
	$\tau \vDash_\CTL \varphi_1$ and $\tau \vDash_\CTL \varphi_2$;
	\item $\tau \vDash_\CTL \neg \varphi$ if
	$\tau \not\vDash_\CTL \varphi$;
	\item $\tau \vDash_\CTL \AX \varphi$ if
	for all $\tau'$ such that $\tau \to \tau'$, $\tau' \vDash_\CTL \varphi$;
	\item $\tau \vDash_\CTL \EX \varphi$ if
	there exists $\tau'$ such that $\tau \to \tau'$ and $\tau' \vDash_\CTL \varphi$;\item $\tau \vDash_\CTL \varphi_1 \AU \varphi_2$ if
	for all $\tau_0,\tau_1,\dots$ such that
	$\tau = \tau_0 \to \tau_1 \to \dots$,
	there exists $i \ge 0$ such that
	$\tau_i \vDash_\CTL \varphi_2$
	and for all $j < i$, $\tau_j \vDash_\CTL \varphi_1$;
	\item $\tau \vDash_\CTL \varphi_1 \EU \varphi_2$ if
	there exists $\tau_0,\tau_1,\dots$ such that
	$\tau = \tau_0 \to \tau_1 \to \dots$, and
	there exists $i \ge 0$ such that
	$\tau_i \vDash_\CTL \varphi_2$
	and for all $j < i$, $\tau_j \vDash_\CTL \varphi_1$;
\end{itemize}
We write $\vDash_\CTL \varphi$ if $\tau \vDash_\CTL\varphi$ for all $\tau$.
CTL admits a sound and complete proof system shown as follows.
We write $\vdash_\CTL\varphi$ if $\varphi$ is provable in CTL.

\begin{center}
	\begin{tabular}{lm{8cm}}
		\multicolumn{2}{l}{
			\em
			Proof system of computational tree logic extends
			propositional calculus with the following:
		}
		\\\hline
		\prule{CTL$_1$}
		&
		$\EX(\varphi_1 \vee \varphi_2) \dimp \EX \varphi_1 \vee \EX \varphi_2$
		\\
		\prule{CTL$_2$}
		&
		$\AX \varphi \dimp \neg (\EX \neg \varphi)$
		\\
		\prule{CTL$_3$}
		&
		$\varphi_1 \EU \varphi_2 \dimp 
		\varphi_2 \vee (\varphi_1 \wedge \EX (\varphi_1 \EU \varphi_2) )$
		\\
		\prule{CTL$_4$}
		&
		$\varphi_1 \AU \varphi_2 \dimp 
		\varphi_2 \vee (\varphi_1 \wedge \AX (\varphi_1 \AU \varphi_2) )$
		\\
		\prule{CTL$_5$}
		&
		$\EX \true \wedge \AX \true$
		\\
		\prule{CTL$_6$}
		&
		$\AG(\varphi_3 \imp (\neg \varphi_2 \wedge \EX \varphi_3))
		\imp (\varphi_3 \imp \neg (\varphi_1 \AU \varphi_2))$
		\\
		\prule{CTL$_7$}
		&
		$\AG(\varphi_3 \imp (\neg \varphi_2 \wedge (\varphi_1 \imp \AX 
		\varphi_3)))
		\imp (\varphi_3 \imp \neg (\varphi_1 \EU \varphi_2))$
		\\
		\prule{CTL$_8$}
		&
		$\AG(\varphi_1 \imp \varphi_2)
		\imp (\EX \varphi_1 \imp \EX \varphi_2)$
		\\
	\end{tabular}
\end{center}

We can define a matching logic theory $\MLCTL = (\sig, H)$
that faithfully captures CTL.
The theory $\MLCTL$ contains all definitions that are needed for fixpoint
constructs
$\mu$ and $\nu$
and the unary symbol ``strong next''~$\snext$.
Define ``weak next'' $\wnext \varphi \equiv \neg \snext \neg \varphi$ as usual.
In addition,  the signature $\sig$ contains a constant symbol $p$
for every atomic proposition $p \in \AP$.
The axiom set $H$ contains fixpoint axioms plus axiom \Inf to capture
the infinite tree semantics of CTL.
In other words, remove axiom \Lin from theory $\MLinfLTL$ and we obtain the
theory $\MLCTL$.
We define CTL modalities as derived constructs in matching logic as follows:
\begin{align*}
\AX \varphi &\equiv \wnext \varphi
&
\EX \varphi &\equiv \snext \varphi
&
\varphi_1 \AU \varphi_2 &\equiv 
\mu f . \varphi_2 \vee (\varphi_1 \wedge \wnext f)
&
\varphi_1 \EU \varphi_2 &\equiv 
\mu f . \varphi_2 \vee (\varphi_1 \wedge \snext f)
\end{align*}
With the above definitions and notations,
\begin{center}
	\emph{Any computational tree logic formula is a matching logic pattern of theory
		$\MLCTL$}.
\end{center}

The standard model of theory $\MLCTL$ has the set $\infTrees$ of all infinite
trees
as its carrier set.
It adopts intended semantics for fixpoint constructs $\mu$ and $\nu$.
Other symbols are interpreted as follows:
\begin{itemize}
	\item $p_\MM = \{ \tau \mid p \in \rt(\tau) \}$ for atomic proposition $p \in
	\AP$;
	\item $\tau \in \snext_\MM(\tau')$ if $\tau \to \tau'$;
\end{itemize}
One can prove that CTL validity coincides with the validity in this standard
model.
In addition, one can prove that all CTL proof rules and axioms are provable
in matching logic.
As in Section~\ref{sec_LTL},
this gives us the following conservative extension result for CTL.
\begin{theorem}[Conservative Extension for Computational Tree Logic]
	Let $\varphi$ be a computational tree logic formula
	and $\MLCTL$ be the matching logic theory for computational tree logic defined
	as above.
	Then,
	$\vdash_\CTL \varphi$ if and only if
	$\MLCTL \vdash \varphi$.
\end{theorem}

Before we end this subsection, we point out that matching logic
provides a uniform way to study and play with variants of CTL.
For example, CTL as presented here adopts infinite-tree semantics.
One can consider a variant of CTL with finite-tree semantics, and it cannot be
easier
to do that in matching logic.
One just needs to replace axiom $\Inf$ in theory $\MLCTL$ with axiom $\Fin$,
or simply remove it to capture both finite- and infinite CTLs.
Instead of designing a new logic, one writes axioms to capture
the intended semantics, and matching logic offers a sound and complete deduction
for free.
Even though the the axiomatization may not completely capture the intended
semantics,
it can faithfully and completely capture logics or calculi with complete
deduction
that are specifically designed for that semantics.


\section{Instance: Propositional Dynamic Logic}

Propositional dynamic logic (PDL) is an extension of modal logic
to reason about programs.
Its syntax is parametric on a set $\AP$ of atomic propositions
and a set $\APgm$ of atomic programs.
PDL has two types of formulas;
\emph{propositional formulas} are similar to formulas in modal logic or
mu-calculus,
and \emph{program formulas (terms)} represent programs built from atomic
programs
and primitive regular expression operators.
\begin{center}
	\begin{tabular}{ll}
		\emph{propositional formulas} &
		$\varphi \Coloneqq
		p \in \AP \mid
		\varphi \to \varphi \mid
		\false \mid
		[\alpha] \varphi$
		\\
		\emph{program formulas} &
		$\alpha \Coloneqq
		a \in \APgm \mid
		\alpha \PDLseq \alpha \mid
		\alpha \PDLunion \alpha \mid
		\alpha \PDLstar \mid
		\alpha \PDLquestion $
	\end{tabular}
\end{center}
Common propositional connectives can be defined
from $\varphi \imp \varphi$ and $\false$ in the usual way.
Common program constructs such as if-then-else,
while-do, and repeat-until statements can also be defined using the four
primitive constructs.
These are not our focus, so we refer readers to PDL literatures such
as~\cite{bibid}
for details.
Define $\angleBraces{a} \varphi \equiv \neg [\alpha] (\neg \varphi)$ as in
mu-calculus.

PDL formulas are interpreted on Kripke frames
$\MM = (M,\bracketM{\_})$ where $M$ is a state set and 
$\bracketM{\_}$ is a meaning function that
\begin{itemize}
	\item maps every atomic proposition $p \in \AP$ to a set of states $\bracketM{p}
	\subseteq M$;
	\item maps every atomic programs $a \in \APgm$ to a binary relation on states
	$\bracketM{a} \subseteq M \times M$.
\end{itemize}
Then, the meaning function is extended to all propositional and program formulasin the following mutual inductive way:
\begin{itemize}
	\item $\bracketM{\varphi_1 \imp \varphi_2} = 
	(M \setminus \bracketM{\varphi_1}) \cup \bracketM{\varphi_2}$;
	\item $\bracketM{\false} = \emptyset$;
	\item $\bracketM{[\alpha] \varphi}
	= \{ s \mid \text{for all $t \in M$ such that $(s,t) \in \bracketM{\alpha}$,                         $t \in \bracketM{\varphi}$ } \}$
	\item $\bracketM{\alpha \PDLseq \beta} =
	\{ (s,t) \mid \text{
		there exists a state $s'$ such that
		$(s,s') \in \bracketM{\alpha}$ and $(s',t) \in \bracketM{\beta}$
	} \}$
	\item $\bracketM{\alpha \PDLunion \beta} =
	\bracketM{\alpha} \cup \bracketM{\beta}$
	\item $\bracketM{\alpha \PDLstar} =
	\bigcup_{n \ge 0} \left(\bracketM{\alpha}\right)^n$
	\item $\bracketM{\varphi\PDLquestion} = \bracketM{\varphi} \times
	\bracketM{\varphi}$
\end{itemize}
A PDL formula $\varphi$ is valid, denoted as $\vDash_\PDL \varphi$,
if it holds in all Kripke frames.
PDL has a sound and complete proof system.
We write $\vdash_\PDL \varphi$ if $\varphi$ is provable in PDL.
\begin{center}
	\begin{tabular}{lm{5cm}lm{5cm}}
		\multicolumn{4}{l}{
			\em
			Proof system of propositional dynamic logic extends propositional calculus with
			the following:
		}
		\\\hline
		\prule{PDL$_1$} &
		$[\alpha] (\varphi_1 \imp \varphi_2) \imp ([\alpha] \varphi_1 \imp [\alpha]
		\varphi_2)$
		&
		\prule{PDL$_2$} &
		$[\alpha] (\varphi_1 \wedge \varphi_2) \dimp ([\alpha] \varphi_1 \wedge [\alpha]
		\varphi_2)$
		\\
		\prule{PDL$_3$} &
		$[\alpha \PDLunion \beta] \varphi \dimp [\alpha] \varphi \wedge [\beta] \varphi$&
		\prule{PDL$_4$} &
		$[\alpha \PDLseq \beta] \varphi \dimp [\alpha][\beta]\varphi$
		\\
		\prule{PDL$_5$} &
		$[\psi \PDLquestion] \varphi \dimp (\psi \imp \varphi)$
		&
		\prule{PDL$_6$} &
		$\varphi \wedge [\alpha][\alpha \PDLstar]\varphi \dimp [\alpha \PDLstar]\varphi$\\
		\prule{PDL$_7$} &
		$\varphi \wedge [\alpha\PDLstar](\varphi \imp [\alpha]\varphi) \imp [\alpha
		\PDLstar] \varphi$
		&
		\prule{Gen} &
		$\prftree{\varphi}{[\alpha]\varphi}$
	\end{tabular}
\end{center}

We compare PDL formulas $[\alpha] \varphi$
with mu-calculus formulas $[a]\varphi$.
In mu-calculus, the action set is a discrete set and actions have no structure;
while in PDL, programs have structures and are constructed from atomic ones withregular expression operators.
Recall that in Section~\ref{sec_mu_calculus},
we define a unary symbol $a$ for every mu-calculus action $a$ and
define $\angleBraces{a} \varphi \equiv a(\varphi)$.
This is known as a \emph{shallow embedding}.
In PDL, we adopt a different approach called \emph{deep embedding},
which we elaborate in detail as follows.

We can define a matching logic theory $\MLPDL = (\sig, H)$ that
faithfully captures PDL.
The signature $\sig = (S,\Sigma)$ contains
a sort set $S = \{ \statesort , \pgm \}$ with
a sort $\statesort$ for propositional formulas and
a sort $\pgm$ for program formulas.
The symbol set $\Sigma$ contains
\begin{itemize}
	\item a constant symbol $p \in \Sigma_{\lambda,\statesort}$
	for atomic proposition $p \in \AP$;
	\item a constant symbol $a \in \Sigma_{\lambda,\pgm}$
	for atomic program $a \in \APgm$;
	\item a unary symbol $\snext \in \Sigma_{\statesort,\statesort}$ called ``strong
	next'';
	\item a binary symbol $\_\PDLseq\_ \in \Sigma_{\pgm \ \pgm , \pgm}$;
	\item a binary symbol $\_ \PDLunion \_ \in \Sigma_{\pgm \ \pgm , \pgm}$;
	\item a unary symbol $\_ \PDLstar \in \Sigma_{\pgm , \pgm}$;
	\item a unary symbol $\_ \PDLquestion \in \Sigma_{\statesort, \pgm}$;
\end{itemize}
In addition, the theory $\MLPDL$ contains all definitions needed for fixpoint
constructs.
Define the notions
$\angleBraces{\alpha} \varphi \equiv \snext(\alpha,\varphi)$
and $[\alpha] \varphi \equiv \neg \angleBraces{\alpha} (\neg \varphi)$.
With the above definitions and notations,
\begin{center}
	\begin{tabular}{l}
		\em
		Any propositional formula of PDL is a matching logic pattern of sort
		$\statesort$;
		\\
		\em
		Any program formula of PDL is a matching logic pattern of sort $\pgm$;
	\end{tabular}
\end{center}

Theory $\MLPDL$ is called a deep embedding because
it defines the concrete syntax of PDL programs in sort $\pgm$,
and has separate axioms defining their semantics.
The axiom set $H$ contains the following four
defining axioms, one for each PDL program construct.
\begin{center}
	\begin{tabular}{llll}
		\prule{Choice} & $[\alpha \PDLunion \beta] \varphi = [\alpha] \varphi \wedge
		[\beta] \varphi$&
		\prule{Seq} & $[\alpha \PDLseq \beta] \varphi = [\alpha][\beta]\varphi$
		\\
		\prule{Test} & $[\psi \PDLquestion] \varphi = (\psi \imp \varphi)$ &
		\prule{Iter} & $[\alpha \PDLstar] \varphi = \nu f . (\varphi \wedge 
		[\alpha] f)$\end{tabular}
\end{center}

Obviously, axioms \prule{Choice}, \prule{Seq}, and \prule{Test}
imply PDL axioms \prule{PDL$_3$}, \prule{PDL$_4$}, and \prule{PDL$_5$},
respectively.
By easy fixpoint reasoning, one can prove that
axiom \prule{Iter} implies PDL axioms
\prule{PDL$_6$} and \prule{PDL$_7$}.
In addition, \prule{PDL$_1$}, \prule{PDL$_2$}, and \prule{Gen}
are general properties about ``strong next'' $\snext$ and also provable in
theory $\MLPDL$.
Therefore, $\vdash_\PDL\varphi$ implies $\MLPDL \vdash \varphi$.

We claim that the four matching logic axioms for PDL program constructs defined
in the above
are more natural to understand and easier to design than the original PDL
axioms.
The PDL proof system is a blend of
general modal logic reasoning, e.g., \prule{PDL$_1$}, \prule{PDL$_2$}, and
\prule{Gen},
and specific axioms about program constructs, e.g, \prule{PDL$_3$} --
\prule{PDL$_7$}.
Besides, axioms \prule{PDL$_6$} are \prule{PDL$_7$} are more like properties
rather than a definition, because
the formula $[\alpha \PDLstar] \varphi$ has multiple occurrences, while 
axiom \prule{Iter} is clearly a definition.

One may argue that \prule{Iter} uses the gfp construct $\nu$
and relies on axioms \Fix and \Gfp,
and \prule{PDL$_6$} and \prule{PDL$_7$} share the same style.
We agree.
And that is exactly why we think one should work in \emph{a uniform and fixed
	logic}
where general axioms about fixpoints can be defined.
Then, we can use these axioms to reason about fixpoint properties
and develop automatic tools and provers, rather than 
designing new logics and tools which all have their own ways to deal with
fixpoints
or induction axioms.
We showed how mu-calculus, LTL, CTL, and PDL can be completely captured in
matching logic
with fixpoint axioms, and we hope it demonstrate that matching logic can be
considered
as a candidate of such a uniform and fixed logic.

We end this subsection with a conservative extension result for PDL.
The result can be shown in the same way as in mu-calculus 
(see Figure~\ref{fig_general_method_conservative_extension}),
and we omit the proof.
\begin{theorem}[Conservative Extension for Propositional Dynamic Logic]
	Let $\varphi$ be a propositional formula in propositional dynamic logic
	and $\MLPDL$ is the matching logic for propositional dynamic logic defined as
	above.
	Then, $\vdash_\PDL\varphi$ if and only if $\MLPDL \vdash \varphi$.
\end{theorem}

\section{Instance: Reachability Logic, (One-Path)}

Reachability logic is a language-independent proof system for deriving
reachability properties
of systems and programs~\cite{bibid}.
Its defining feature is the \circularity proof rule that supports reasoning
about
circular behavior of iterative and recursive program constructs.
In historical literature~\cite{bibid},
reachability logic is proposed alongside matching logic for program
verification,
where matching logic is used for defining static structure and
program configurations, while reachability logic is used for reasoning about
dynamic behavior.


In this section, we show that reachability logic
is just an instance of matching $\mu$-logic.
What reachability logic offers is the power to specify and reason about
dynamic properties about programs.
In its essence, a program defines a transition system over \emph{configurations}.
We can very well achieve the same thing using matching $\mu$-logic.

\subsection{Reachability logic basics}

Reachability logic is parametric on a matching logic model called
the \emph{underlying configuration model},
so before we introduce reachability logic syntax and semantics,
let us first define the underlying configuration model.

Let $\sig_\cfg$ be a matching logic signature 
used to specify static program configurations.
Depending on the target programming language of interest,
the signature $\sig_\cfg$
may have various sorts and symbols, among which there is 
a distinguished sort~$\Cfg$ for program configurations. 
Let $M_\cfg$ be a matching logic $\sig_\cfg$-model
which we call the \emph{underlying configuration model}.
We write $M_\Cfg$ to mean the carrier set of sort $\Cfg$,
i.e., the set of all program configurations. 

Reachability has simple syntax.
The basic sentences are called \emph{reachability rules}, which have the form
\begin{align}
\varphi_1 \To \varphi_2, \quad \text{$\varphi_1,\varphi_2$ are patterns of sort $\Cfg$}.
\end{align}
A \emph{reachability system}, denoted as $T$, is a set of rules;
it then yields a transition system over configurations,
denoted as $\TT = (M_\Cfg, \to)$,
where $M_\Cfg$ is the domain of all configurations in the
underlying configuration model.
The transition relation $\to$ is defined such that
$s \to t$ if and only if
there exists a rule $\varphi_1 \To \varphi_2$ in $T$
and a matching logic valuation $\rho$ such that
$s \in \rhobar(\varphi_1)$ and $t \in \rhobar(\varphi_2)$.
Let $\to^* = \bigcup_{k \ge 0} (\to)^k$ 
be the transitive and reflexive closure of ${\to}$.

Let $\psi_1 \To \psi_2$ be any  reachability rule.
We say it is $\rho$-valid,
denoted as 
$T,\rho \vDash_\uRL \psi_1 \To \psi_2$,
if for every configuration $s$ such that $s \in \rhobar(\varphi_1)$,
either there is an infinite transition sequence
$s \to s' \to s'' \to \dots$ in the transition system $\TT$ yielded by $T$,
or there exists a configuration $t$
such that $s \to^* t$ and $t \in \rhobar(\varphi_2)$.
Rule $\psi_1 \To \psi_2$ is valid, denoted as $T \vDash_\uRL \psi_1 \To \psi_2$,if it is $\rho$-valid for every valuation $\rho$.
Notice that validity in reachability logic is defined
in the spirit of partial correctness.

 reachability logic
has a sound and relatively complete proof system as shown below.
Notice the proof system derives more general sequents of the form
$A \vdash_C \varphi_1 \To \varphi_2$
where $A$ and $C$ are sets of rules.
We call rules in $A$ \emph{axioms} and rules in $C$ \emph{circularities}.
\begin{figure}
\begin{tabular}{m{0.95\textwidth}}
	{
		\em
		Proof system of  reachability logic
		is parametric in a matching logic signature~$\sig$ 
		for configurations
		and an underlying configuration model $\MM$; it contains the following rules:
	}
	\\\hline
	$
	\prftree[r,l]
	{
		if $(\varphi_1 \To \varphi_2) \in A$ and $\psi$ is a predicate pattern
	}
	{\prule{Axiom}\quad}
	{\cdot}
	{A \vdash_C \varphi_1 \wedge \psi \To \varphi_2 \wedge \psi}
	$
	\\ 
	$
	\prftree[l]
	{\prule{Reflexivity} \quad}
	{\cdot}
	{A \vdash_\emptyset \varphi \To \varphi
	}
	$
	\qquad\quad
	$
	\prftree[l]
	{\prule{Transitivity} \quad}
	{A \vdash_C \varphi_1 \To \varphi_2}
	{A \cup C \vdash_\emptyset \varphi_2 \To \varphi_3}
	{A \vdash_C \varphi_1 \To \varphi_3}
	$
	\\
	$
	\prftree[l]
	{\prule{Consequence}\quad}
	{
		\MM \vDash \varphi_1 \imp \varphi'_1
	}
	{
		A \vdash_C \varphi'_1 \To \varphi'_2
	}
	{
		\MM \vDash \varphi'_2 \imp \varphi_2
	}
	{A \vdash_C \varphi_1 \To \varphi_2}
	$
	\\
	$
	\prftree[r,l]
	{if $x \not\in \FV(\varphi_2)$}
	{\prule{Abstraction} \quad}
	{A \vdash_C \varphi_1 \To \varphi_2}
	{A \vdash_C (\exists x . \varphi_1) \To \varphi_2}
	$
	\\
	$
	\prftree[l]
	{\prule{Circularity} \quad}
	{
		A \vdash_{C \cup \{\varphi_1 \To \varphi_2\}} \varphi_1 \To \varphi_2
	}
	{A \vdash_C \varphi_1 \To \varphi_2
	}
	$
	\quad
	$
	\prftree[l]
	{\prule{Case Analysis} \quad}
	{A \vdash_C \varphi_1 \To \varphi}
	{A \vdash_C \varphi_2 \To \varphi}
	{A \vdash_C \varphi_1 \vee \varphi_2 \To \varphi}
	$
	\end{tabular}
\caption{Reachability logic proof system}
\label{fig:RL_proof_system}
\end{figure}
Rule $\varphi_1 \To \varphi_2$ is provable in the reachability system $T$,
denoted as $T \vdash_\uRL \varphi_1 \To \varphi_2$,
if the sequent $T \vdash_\emptyset \varphi_1 \To \varphi_2$ can be derived
in the above proof system.
\begin{theorem}[Soundness and completeness of  
	reachability logic]
	\label{thm_relative_completeness_RL}
	Let $\sig$ be a matching logic signature for configurations
	and $\MM$ be the underlying configuration model.
	Let $T$ be a reachability system and 
	$\varphi_1 \To \varphi_2$ be an  reachability rule.
	Then $T \vDash_\uRL \varphi_1 \To \varphi_2$
	if and only if $T \vdash_\uRL \varphi_1 \To \varphi_2$.
\end{theorem}
The completeness of  reachability logic
is a \emph{relative} one 
because the proof system consults the underlying configuration model $\MM$
for validity in \prule{Consequence} rule.
In other words,  reachability logic is complete
relative to the completeness of $\MM$.
If $\MM$ can be completely axiomatized by a recursively enumerable
set of axioms in matching logic,
then $\MM \vDash \varphi \imp \varphi'$ is decidable,
and the  reachability logic 
(parametric on $\MM$) becomes ``absolutely'' complete, i.e.,
there exists an algorithm that decides the validity of 
reachability rules.
In practice, however, $\MM \vDash \varphi \imp \varphi'$ is often undecidable,
and thus no algorithm can decide the validity of reachability rules.
What Theorem~\ref{thm_relative_completeness_RL} tells us is that
this incompleteness origins in the undecidability of the underlying
configuration model $\MM$, not reachability logic itself.

\subsection{Reachability patterns}

Recall that $\sig_\cfg$ is the matching logic signature of configurations,
and $M_\cfg$ is the underlying configuration model.
An instance of reachability logic is then based 
on the underlying configuration model $M_\cfg$.

Let us define an extended signature
$\sig = \sig_\cfg \cup \{ \snext \in \Sigma_{\Cfg , \Cfg} \}$
where $\snext$ is the ``strong next'' symbol.
The common temporal modalities in linear temporal logic
and computation tree logic can be defined in the usual way.
The objective here is to find a pattern that captures the reachability rule
$\varphi_1 \To \varphi_2$. 
Let $T$ be a transition system where the reachability rule $\varphi_1 \To \varphi_2$
holds. 
According to the semantics of reachability logic, 
this means that for every configuration $\gamma_0 \vDash \varphi_1$, 
there exists a complete path $\tau = \gamma_0 \gamma_1 \gamma_2 \dots$ such that
$\tau$ is a finite path implies 
there exists $n \in \nats$ such that $\gamma_n \vDash \varphi_2$.
In other words, from configuration $\gamma_0$, 
either ``eventually'' $\varphi_2$ holds, in the usual LTL sense,
or there exists an infinite path.
This leads us to the following attempting definition of a reachability pattern:
\begin{align}
\varphi_1 \To \varphi_2 
\equiv 
\varphi_1 \to ( (\neg \mu X . \wnext X) \vee \eventually \varphi_2)
\end{align}
Recall that under standard semantics,
the pattern $\mu X . \wnext X$ is the set of states that are \emph{well-founded},
meaning that they terminate on all paths.
Therefore, $\neg \mu f . \wnext f$ is exactly the set of states 
that admit an infinite path. 

\begin{proposition}
For any pattern $\varphi$ and pattern set $\Gamma$, 
$\Gamma \vdash ( (\neg \mu X \ldot \wnext X) \vee \eventually \varphi ) = 
\nu X . \varphi \vee \snext\varphi$.
\end{proposition}

Let us define ``weak eventually'' 
$\eventually_w \varphi \equiv \nu X . \varphi \vee \snext \varphi$.


\subsection{Reachability logic as an instance of matching $\mu$-logic}
\newcommand{\transA}{\mathrm{Axiom2MmL}}
\newcommand{\transC}{\mathrm{Circ2MmL}}
\newcommand{\transR}{\mathrm{Rule2MmL}}

Reachability logic considers sequents that have the form
$A \vdash_C \varphi_1 \To \varphi_2$.
In the following, we define a translation 
from reachability logic sequents to 
matching $\mu$-logic obligations.
This translation will be more complicated than the ones
in LTL or CTL, so we define it gradually.

We first define the translation of reachability rules in the axiom set $A$.
We denote this translation $\transA$.
According to the semantics of reachability logic, 
an axiom $\psi_1 \To \psi_2 \in A$ means that
the transition system \emph{can make a step} from a configuration
where $\psi_1$ holds to a configuration where $\psi_2$ holds. 
This leads to the following definition of the translation $\transA$:
\begin{align*}
&\text{If the axiom set $A$ is empty:}\\
&\quad \transA(A) = \top; \\
&\text{If the axiom set $A = \{ \psi_1 \To \psi_2 \} \cup A'$:}\\
&\quad \transA(A) =
(\forall x_1 \dots \forall x_n \ldot (\psi_1 \imp \snext \psi_2))
\wedge \transA(A')
\\
& \text{ where $\{x_1,\dots,x_n\} = \FV(\psi_1) \cup \FV(\psi_2)$}
\end{align*}
Here, the pattern $\psi_1 \imp \snext \psi_2$ 
captures the intuitive meaning that 
the transition system \emph{can make a step} from any configurations that satisfies
$\psi_1$ to some configurations where $\psi_2$ holds. 
The $\snext$ symbol in $\snext \psi_2$ makes sure the system
indeed makes a step in order to satisfy $\eventually_w \psi_2$. 
This becomes crucial when we prove all the proof rules in the reachability logic
proof system in matching $\mu$-logic, especially when we prove
the \circularity rule. 
We point out that there are multiple ways to define the translation. 
For example, instead of $\psi_1 \to \snext \eventually_w \psi_2$,
one can use $\psi_1 \to \snext \psi_2$.

Next, let us define the translation of reachability rules in the circularity set $C$.
We denote this translation $\transC$. 
The intuition of a reachability rule $\psi_1 \To \psi_2$ in the circularity set
is that the transition system will satisfy $\psi_2$ if it
\emph{makes any step} from a configuration that satisfy $\psi_1$.
This leads to the following definition of the translation $\transC$:
\begin{align*}
&\text{If the circularity set $C$ is empty:}\\
&\quad \transC(C) = \top; \\
&\text{If the circularity set $C = \{ \psi_1 \To \psi_2 \} \cup C'$:}\\
&\quad \transC(C) =
(\forall x_1 \dots \forall x_n \ldot (\psi_1 \imp \wnext \eventually_w \psi_2))
\wedge \transC(C')
\\
& \text{ where $\{x_1,\dots,x_n\} = \FV(\psi_1) \cup \FV(\psi_2)$}
\end{align*}
Here, the pattern $\psi_1 \to \wnext \eventually_w \psi_2$ captures the intuitive
meaning that the transition system will eventually satisfy $\psi_2$
\emph{after any steps} it makes at any configuration that satisfies $\psi_1$. 
The $\wnext$ symbol in $\wnext \eventually_w \psi$ makes sure the system can satisfies
$\eventually_w \psi_2$ as long as it makes a step, and it does not matter
what step it makes. 

Finally, we define the translation of reachability rule $\varphi_1 \To \varphi_2$
that appears in a reachability sequent, as in $A \vdash_C \varphi_1 \To \varphi_2$.
Depending on if the circularity set $C$ is empty or not,
the reachability rule $\varphi_1 \To \varphi_2$ has different requirements
on if $\varphi_2$ may hold at the current configuration or must hold sometime
in the future. 
If $C$ is empty, then $\varphi_2$ may hold at the current configuration;
e.g., see \prule{Reflexivity} rule.
If $C$ is not empty, then $\varphi_2$ can only hold after at least one step
from the current configuration. 
This leads us to the next definition of transformation $\transR$,
which takes a reachability rule $\varphi_1 \To \varphi_2$
and a circularity set $C$ as arguments:
\begin{align*}
&\text{If the circularity set $C$ is empty:}\\
&\quad \transR(\varphi_1 \To \varphi_2, C) = \varphi_1 \imp \eventually_w \varphi_2; \\
&\text{If the circularity set $C$ is nonempty:}\\
&\quad \transR(\varphi_1 \To \varphi_2, C) = \varphi_1 \imp \snext \eventually_w \varphi_2.\\
\end{align*}
In other words, when the circularity set $C$ is not empty, 
the reachability sequent $A \vdash_C \varphi_1 \To \varphi_2$
asks to prove something stronger than when the circularity set $C$ is empty.

Now we are ready to transform any reachability logic sequence
$A \vdash_C \varphi_1 \To \varphi_2$ to the following proof obligation
in matching $\mu$-logic:
\begin{align*}
\vdash
\transA(A) \wedge \transC(C) \imp \transR(\varphi_1 \To \varphi_2, C)
\end{align*}

\begin{proposition}
Given any axiom set $A$ and circularity set $C$ of finitely many reachability rules,
if the reachability logic sequent $A \vdash_C \varphi_1 \To \varphi_2$
can be proved from the reachability logic proof system 
shown in Figure~\ref{fig:RL_proof_system},
then
$$
\transA(A) \wedge \transC(C) \imp \transR(\varphi_1 \To \varphi_2, C)
$$
is provable in matching $\mu$-logic.

\end{proposition}

\subsection{Get Material From Here}

\begin{center}
	\em
	Any reachability logic rule $\varphi_1 \To \varphi_2$
	is a matching logic pattern of sort $\Cfg$.
\end{center}

Let $\MLuRL$ be a matching logic theory of signature $\sig^\to$.
We use it to capture  reachability logic.
The theory $\MLuRL$ contains all definitions needed for fixpoint constructs.
In addition, it contains all valid patterns 
in the underlying configuration model $\MM$ as axioms.
This makes all implications needed for \prule{Consequence} rule
provable in theory $\MLuRL$.

Let $T$ be a reachability system 
and $\TT = (M_\Cfg, \to)$ be its yielded transition system.
It is not hard to phrase the transition system $\TT$ as a matching logic model
of theory $\MLuRL$ and prove that
$\TT \vDash \varphi_1 \To \varphi_2$
if and only if
$T \vDash_\uRL \varphi_1 \To \varphi_2$.
Extend theory $\MLuRL$ by adding all reachability rules in $T$ as axioms
and denote the extended theory as $\MLuRL_T$.
It follows that $\TT \vDash \MLuRL_T$, and that
$\MLuRL_T \vDash \varphi_1 \To \varphi_2$ implies
$T \vDash_\uRL \varphi_1 \To \varphi_2$.

We conjecture the following conservative extension theorem for
reachability logic, whose proof is postponed to future work.
\begin{conjecture}[Conservative Extension for  Reachability Logic]
	Let $\sig$ be a matching logic signature of configurations,
	$\MM$ be an underlying configuration model,
	$\varphi_1 \To \varphi_2$ be a reachability rule,
	$T$ be a reachability system,
	and $\MLuRL_T$ be the corresponding matching logic theory all as defined above.
	Then, $T \vdash_\uRL \varphi_1 \To \varphi_2$ if and only if
	$\MLuRL_T \vdash \varphi_1 \To \varphi_2$.
\end{conjecture}

To prove the conjecture, it suffices to prove that
$T \vdash_\uRL \varphi_1 \To \varphi_2$ implies
$\MLuRL_T \vdash \varphi_1 \To \varphi_2$.
In the following, we use an example originated from
program verification problems to show how reachability proofs,
especially application of \prule{Circularity} rule,
can be carried out in matching logic.

\todochen{Finish the detail of example \textsf{SUM}}

The invariant rule we want to prove is that
\begin{equation}\label{eq_RLexample_inv}
\exists n . (\varphi(n) \wedge n \ge 0) \To \psi \tag{Invariant}
\end{equation}
Firstly, let us list some facts that are needed in the proof.
Let us assume that the following two patterns are proved in advance.
\begin{align}
&\varphi(0) \imp \eventually \psi
\tag{Base Case}\label{eq_RLexample_base_case}
\\
&\varphi(n) \wedge n \ge 1 \imp \snext^k \varphi(n-1)
\qquad \text{for some $k \ge 1$}
\tag{Loop Body}\label{eq_RLexample_loop_body}
\end{align}
The following patterns are either valid in the underlying
configuration model $\MM$ or provable using fixpoint axioms.
\begin{align}
&\exists n (\varphi(n) \wedge n \ge 0) 
= \varphi(0) \vee \exists n . (\varphi(n) \wedge n \ge 1)
\tag{Domain}\label{eq_RLexample_domain}
\\
&\eventually \psi_1 = \snext^k \psi_1
\tag{Next Eventually}\label{eq_RLexample_next_eventually}
\\
&\forall n . \wnext^k \psi_1 = \wnext^k \forall n . \psi_1
\tag{Comm}\label{eq_RLexample_comm}
\\
&\mu f . \wnext f = \mu f . \wnext^k f
\tag{Fix Next}\label{eq_RLexample_fix_wnext}
\\
&\wnext^k(\psi_1 \imp \psi_2) \wedge \snext^k \psi_1 \imp \snext^k \psi_2
\tag{Progress}\label{eq_RLexample_progress}
\end{align}
Finally we show the proof of~\eqref{eq_RLexample_inv} in 
Figure~\ref{fig_RLexample_proof}.
The symbol ``$\Longleftarrow$'' in the proof should be read as
``to prove the above, it suffices to prove the below''.

\begin{figure}
	{
		\small
		\begin{align*}
		&&
		\exists n . (\varphi(n) \wedge n \ge 0) \To \psi
		\\
		&\xif{by definition}&
		\exists n . (\varphi(n) \wedge n \ge 0) \imp (\mu f . \wnext f 
		\imp \eventually\psi)
		\\
		&\xif{by~\eqref{eq_RLexample_domain}}&
		\varphi(0) \vee \exists n . (\varphi(n) \wedge n \ge 1)
		\imp (\mu f . \wnext f \imp\eventually \psi)
		\\
		&\xif{by propositional reasoning}&
		(\varphi(0) \imp (\mu f . \wnext f \imp\eventually \psi))
		\wedge (\exists n . (\varphi(n) \wedge n \ge 1)
		\imp (\mu f . \wnext f \imp\eventually \psi))
		\\
		&\xif{by~\eqref{eq_RLexample_base_case}}&
		\exists n . (\varphi(n) \wedge n \ge 1)
		\imp (\mu f . \wnext f \imp\eventually \psi)
		\\
		&\xif{by FOL reasoning}&
		\mu f . \wnext f \imp \forall n .
		(\varphi(n) \wedge n \ge 1 \imp \eventually \psi)
		\\
		&\xif{by~\eqref{eq_RLexample_fix_wnext}}&
		\mu f . \wnext^k f \imp \forall n .
		(\varphi(n) \wedge n \ge 1 \imp \eventually \psi)
		\\
		&\xif{by~\Lfp}&
		\wnext^k \forall n . (\varphi(n) \wedge n \ge 1 \imp \eventually \psi)
		\imp
		\forall n . (\varphi(n) \wedge n \ge 1 \imp \eventually \psi)
		\\
		&\xif{by~\eqref{eq_RLexample_comm}}&
		\forall n . \wnext^k  (\varphi(n) \wedge n \ge 1 \imp \eventually \psi)
		\imp
		\forall n . (\varphi(n) \wedge n \ge 1 \imp \eventually \psi)
		\\
		&\xif{by FOL reasoning}&
		\wnext^k  (\varphi(n-1) \wedge n-1 \ge 1 \imp \eventually \psi)
		\imp
		(\varphi(n) \wedge n \ge 1) \imp \eventually \psi
		\\
		&\xif{by propositional reasoning}&
		\wnext^k  (\varphi(n-1) \wedge n-1 \ge 1 \imp \eventually \psi)
		\wedge
		\varphi(n) \wedge n \ge 1 \imp \eventually \psi
		\\
		&\xif{by~\eqref{eq_RLexample_loop_body}}&
		\wnext^k  (\varphi(n-1) \wedge n-1 \ge 1 \imp \eventually \psi)
		\wedge
		\snext^k \varphi(n-1) \wedge n \ge 1 \imp \eventually \psi
		\\
		&\xif{by propositional reasoning}&
		\wnext^k  (\varphi(n-1) \wedge n \ge 2 \imp \eventually \psi)
		\wedge \snext^k \varphi(n-1) \wedge n \ge 2 \imp \eventually \psi
		\\&&
		\wedge\quad
		\wnext^k  (\varphi(n-1) \wedge n \ge 2 \imp \eventually \psi)
		\wedge \snext^k \varphi(n-1) \wedge n = 1 \imp \eventually \psi
		\\
		&\xif{by propositional reasoning}&
		\wnext^k  (\varphi(n-1) \wedge n \ge 2 \imp \eventually \psi)
		\wedge \snext^k \varphi(n-1) \wedge n \ge 2 \imp \eventually \psi
		\\&&
		\wedge\quad
		\snext^k \varphi(0) \imp \eventually \psi
		\\
		&\xif{by~\eqref{eq_RLexample_next_eventually}}&
		\wnext^k  (\varphi(n-1) \wedge n \ge 2 \imp \eventually \psi)
		\wedge \snext^k \varphi(n-1) \wedge n \ge 2 \imp \eventually \psi
		\\&&
		\wedge\quad
		\snext^k \varphi(0) \imp \snext^k \eventually \psi
		\\
		&\xif{by~\eqref{eq_RLexample_base_case} and frame reasoning}&
		\wnext^k  (\varphi(n-1) \wedge n \ge 2 \imp \eventually \psi)
		\wedge \snext^k \varphi(n-1) \wedge n \ge 2 \imp \eventually \psi
		\\
		&\xif{by matching logic reasoning}&
		\wnext^k  (\varphi(n-1) \wedge n \ge 2 \imp \eventually \psi)
		\wedge \snext^k (\varphi(n-1) \wedge n \ge 2) \imp \eventually \psi
		\\
		&\xif{by~\eqref{eq_RLexample_progress}}&
		\snext^k  (\eventually \psi) \imp \eventually \psi
		\\
		&\xif{by~\eqref{eq_RLexample_next_eventually}}&
		\QED
		\end{align*}
	}
	\caption{A proof of~\eqref{eq_RLexample_inv}.}
	\label{fig_RLexample_proof}
\end{figure}

\section{Instance: First-Order Logic with Least Fixpoints}

\section{Instance: Separation Logic}








\section{Get Material From Here}


Intuitively,
$\rhobar(\varphi)$ 
is the set of elements that match the pattern $\varphi$.
Derived constructs are defined as follows for convenience:
\begin{center}
	\begin{tabular}{rclp{1cm}rcl}
		$\top_s$ & $\equiv$ & $\exists x \cln s . x \cln s$
		&&
		$\bot_s$ & $\equiv$ & $\neg \top_s$
		\\
		$\varphi_1 \vee \varphi_2$ & $\equiv$ & 
		$\neg (\neg \varphi_1 \wedge \neg \varphi_2)$
		&&
		$\varphi_1 \imp \varphi_2$ & $\equiv$ &
		$\neg \varphi_1 \vee \varphi_2$
		\\
		$\varphi_1 \dimp \varphi_2$ & $\equiv$ &
		$(\varphi_1 \imp \varphi_2) \wedge (\varphi_2 \imp \varphi_1)$
		&&
		$\forall x . \varphi$ & $\equiv$ &
		$\neg (\exists x . \neg \varphi)$
	\end{tabular}
\end{center}
Interested readers are encouraged to prove these derived constructs
have the intended semantics,
or refer to~\cite{bibid} for details.
We often drop the sort subscripts when there is no confusion.

Given a model $\MM$ and a valuation $\rho$,
we say $\MM$ and $\rho$ satisfy a pattern $\varphi_s$,
denoted as $\MM,\rho \vDash \varphi_s$,
if $\rhobar(\varphi) = M_s$.
We say $\MM$ satisfies $\varphi_s$
or $\varphi_s$ holds in $\MM$,
denoted as $\MM \vDash \varphi_s$,
if $\MM , \rho \vDash \varphi_s$ for every valuation $\rho$.
We say $\varphi_s$ is valid if
it holds in every model.
Let $\Gamma$ be a pattern set.
We say $\MM$ satisfies $\Gamma$, if
$\MM \vDash \varphi$ for every $\varphi \in \Gamma$.
We say $\Gamma$ semantically entails $\varphi$,
denoted as $\Gamma \vDash \varphi$,
if for every model $\MM$ such that $\MM \vDash \Gamma$,
$\MM \vDash \varphi$.
When $\Gamma$ is the empty set, we abbreviate
$\emptyset \vDash \varphi$ as just $\vDash \varphi$,
which is equivalent to say $\varphi$ is valid.


Given a signature $\sig$, matching logic gives us all $\sig$-models.
Sometimes, we are only interested in some models, instead of all models.
There are typically two ways to restrict models.
One way is to define a syntactic matching logic theory
$(\sig, H)$ where $H$ is a set of patterns called axioms.
A model $\MM$ belongs to the theory $(\sig, H)$ if $M \vDash H$.
Syntactic theories are preferred if the models of interest can be
axiomatized by a recursively enumerable set $H$.
Most syntactic theories defined in this paper have finite axiom sets.
Alternatively, we can define a semantic matching logic theory
$(\sig, \CC)$ where $\CC$ is a collection of $\sig$-models.
A model $\MM$ belongs to the theory $(\sig, \CC)$ if
$\MM \vDash \varphi$ for all $\varphi$ that holds in all models in $\CC$.
Usually, the collection $\CC$ is a singleton set containing exactly one model
$\II$,
often referred as the intended model or the standard model,
and we abbreviate $(\sig, \{ \II \})$ as $(\sig, \II)$.
Semantic theories are preferred if the intended model is not axiomatized by any
recursively enumerable set of axioms;
this is often the case for 
initial algebra semantics or domains which are defined by induction, such as
natural numbers (with multiplication) and finite maps.
We will see an example of semantic theories in
Section~\ref{sec_separation_logic}.

Syntactic theories support a notion of proofs (see
Section~\ref{sec_pure_matching_logic})
as they have an axiom set,
but semantic theories offer arbitrary expressiveness.
They both are means to restricting models,
and we simply say ``theories'' when their distinction is not the emphasis.


\todorosu{
	I think it is also important to state that when all models are assumed, 
	then ML has been shown to have the same expressiveness as FOL=, 
	but like it is the case with capturing SL, 
	or in FOL with induction, 
	or in initial algebra semantics, 
	one can reduce the set of models and thus get arbitrary expressiveness.
}

\subsection{Known results about matching logic expressiveness power}





A few important logics and calculus are shown to be definable 
in matching logic, including
propositional calculus, predicate logic,
algebraic specification, first-order logic with equality,
modal logic S5, and separation logic.
In this section, we only discuss a few of them and refer interested readers
to~\cite{bibid} for more details.

\subsubsection{Definedness, equality, membership, functions, and partial 
functions}

As we have seen, patterns are interpreted as sets.
When it comes to classical reasoning for existing mathematical domains,
we need a way to interpret patterns in a conventional, two-value way;
for example, the total set means true and the empty set means false.
We also want to lift reasoning with sort $s_1$ to sort $s_2$.
In matching logic, the above is methodologically achieved by
\emph{definedness} symbols.
For any two sorts $s$ and $s'$ which need not be distinct, 
the definedness symbol
$\ceil{\_}_{s}^{s'} \in \SigmaSub{s,s'}$
is a unary matching logic symbol which has an axiom
$
\ceil{x \cln s}_{s}^{s'}
$
called the definedness axiom. 
The axioms makes $\ceil{\_}_{s}^{s'}$ behaves like a predicate
that checks definedness:
\begin{align*}
& \rhobar(\ceil{\varphi_s}_{s}^{s'}) = M_{s'}
  \quad \text{if $\rhobar(\varphi_s) \neq \emptyset$}
& \rhobar(\ceil{\varphi_s}_{s}^{s'}) = \emptyset
  \quad \text{if $\rhobar(\varphi_s) = \emptyset$}
\end{align*}
Definedness symbols allow us to define many useful derived constructs, including\begin{align*}
&\floor{\varphi}_s^{s'} \equiv \neg \ceil{\neg \varphi}_s^{s'}
&& \text{that checks totality}
&x \in_s^{s'} \varphi \equiv \ceil{x \wedge \varphi}_s^{s'}
&& \text{that checks membership}
\\
&\varphi_1 =_s^{s'} \varphi_2 \equiv \floor{\varphi_1 \dimp \varphi_2}_s^{s'}
&& \text{that checks equality}
&\varphi_1 \subseteq_s^{s'} \varphi_2 
\equiv \floor{\varphi_1 \imp \varphi_2}_s^{s'}
&& \text{that checks containment}
\end{align*}
These derived constructs have the intended semantics.
For example,
\begin{align*}
& \rhobar(\varphi_1 =_{s}^{s'} \varphi_2) = M_{s'}
  \quad \text{if $\rhobar(\varphi_1) = \rhobar(\varphi_2)$}
& \rhobar(\varphi_1 =_{s}^{s'} \varphi_2) = \emptyset
  \quad \text{if $\rhobar(\varphi_1) \neq \rhobar(\varphi_2)$}
\end{align*}
We drop their sort subscripts if there is no confusion.

In matching logic, symbols are interpreted relationally
$\sigmaM \colon M_{s_1} \times ,\dots, \times M_{s_n} \to 2^{M_s}$.
Sometimes, we want to state a symbol $\sigma$ is to be interpreted
as a function in all models.
This is achieved by adding the axiom
$\exists y . \sigma(x_1 ,\dots, x_n) = y$
where $x_1 ,\dots, x_n , y$ are distinct.
Partial functions can be defined in the similar way, by adding the axiom
$\neg \sigma(x_1 ,\dots, x_n) \vee \exists y . \sigma(x_1 ,\dots, x_n) = y$.
We point out that partial functions has been the main subject of research
in partial first-order logic, with various logics and axioms proposed
to capture the desired properties of definedness and undefinedness;
while matching logic allows us elegantly define definedness and partial 
functions, without a need to develop a new logic.

\subsubsection{Modal logic S5}
\label{sec_modal_logic_S5}
One of the most popular modal logics, S5, is definable in 
matching logic.
S5 is parametric on a set of atomic propositions $\AP$,
and 
its syntax, as shown below, extends propositional calculus with a unary modality$\always$ that captures necessity:
$\always \varphi$ is read as ``it is necessary that $\varphi$''.
A dual modality $\eventually \varphi \equiv \neg (\always \neg \varphi)$
is defined to capture possibility.
\begin{equation*}
\varphi \Coloneqq
\AP \mid \varphi \imp \varphi
\mid \neg \varphi
\mid \always \varphi
\end{equation*}
S5 formulas are interpreted on a set $W$ of worlds
with a valuation $v \colon \AP \times W \to \{ \true, \false \}$
stating that each proposition holds in a given subset of worlds.
The valuation is then extended to S5 formulas inductively:
\begin{itemize}
\item $v(\neg \varphi, w) = \true$ if $v(\varphi, w) = \false$;
      otherwise,
      $v(\neg \varphi, w) = \false$.
\item $v(\varphi_1 \imp \varphi_2, w) = \true$
      if $v(\varphi_1, w) = \false$ or $v(\varphi_2, w) = \true$;
      otherwise,
      $v(\varphi_1 \imp \varphi_2, w) = \false$.
\item $v(\always \varphi, w) = \true$ if
      $v(\varphi, w') = \true$ for every $w' \in W$;
      otherwise,
      $v(\always \varphi, w) = \false$.
\end{itemize}
An S5-formula is valid, denoted as $\vDash_{S5} \varphi$, if for every 
world set $W$ and valuation $v$,
$v(\varphi, w) = \true$ for every $w \in W$.
S5 admits the following sound and complete proof system
which can derive all valid S5-formulas:
\begin{center}
\begin{tabular}{lm{5cm}ll}
\multicolumn{4}{l}{
\em 
Proof system of Modal logic S5 
extends propositional calculus proof system with the following:
}
\\\hline
\prule{N}
&
$\prftree{\varphi}{\always \varphi}$
&
\prule{K}
&
$\always(\varphi_1 \imp \varphi_2) 
 \imp (\always \varphi_1 \imp \always \varphi_2)$
\\
\prule{M}
&
$\always \varphi \imp \varphi$
&
\prule{5}
&
$\eventually \varphi \imp \always \eventually \varphi$
\end{tabular}
\end{center}

Modal logic S5 can be faithfully captured by
a matching logic theory which we refer to as $\MLSfive$.
The theory $\MLSfive$ has a signature $\sig = (S, \Sigma)$ where
$S$ contains exactly one sort, say $\world$,
and $\Sigma$ contains a unary symbol
$\eventually \in \Sigma_{\world,\world}$,
plus a constant symbol $p \in \Sigma_{\lambda,\world}$
for every atomic proposition $p \in \AP$.
The theory $\MLSfive$ has only one axiom, $\eventually w$, which states that
the symbol $\eventually$ is the definedness symbol.
The totality $\always \varphi \equiv \neg (\eventually \neg \varphi)$
is defined as a derived construct.
With the above definitions,
\begin{center}
\emph{
Any S5-formula is a matching logic pattern in the theory $\MLSfive$.
}
\end{center}

We establish an important result known as the ``conservative extension'',
which is proved via a model-theoretic approach.
We elaborate the proof of conservative extension for S5 here as an example,
since the same proof idea will show up multiple times in the rest of the paper.
In short, we will show there is a one-to-one correspondence between
pairs $(W,v)$ of an S5 world set and a valuation, 
and matching logic models of the theory $\MLSfive$.
The correspondence works in two directions.
For the direction from S5 to matching logic,
the correspondence takes $W$ as the carrier set of sort $\world$,
and interprets $\eventually$ as the definedness predicate
$\eventually_\world (w) = W$ for every $w \in W$.
Moreover, every atomic proposition $p \in \AP$ is interpreted to
the subset of worlds $p_\world = \{ w \in W \mid v(p, w) = \true \}$.
The other direction from matching logic to S5 is left for 
the interested readers.
Under this one-to-one correspondence, 
we can prove by structural induction that for every S5-formula $\varphi$,
\begin{center}
$v(\varphi, w) = \true$
\quad if and only if \quad
$w \in \rhobar(\varphi)$.
\end{center}
Notice that S5-formulas contain no variables as matching logic patterns,
so it does not matter which valuation $\rho$ we pick in the above.
Immediately, we know $\vDash_{S5} \varphi$ if and only if
$\MLSfive \vDash \varphi$ for every S5-formula $\varphi$.
By the completeness results of both S5 and matching logic,
we know $\vdash_{S5} \varphi$ if and only if $\MLSfive \vdash \varphi$
for every S5-formula $\varphi$.

\subsubsection{First-order logic and its variants}
TBC.
\todochen{
This section should say that
FOL= and ML have the same expressiveness when considering all models.
We may also show the reduction to FOL without equality, 
or why ML is more expressive than FOL without equality.
We may show FOL+lfp here, too.
}

\subsubsection{Separation logic}
\label{sec_separation_logic}

Separation logic is a logic specifically designed for reasoning about
heap structures.
The syntax of separation logic extends first-order logic with
some heap operations:
\begin{equation*}
\varphi \Coloneqq
\text{(first-order logic syntax)}
\mid \emp \mid \Nat \mapsto \Nat \mid \varphi \merge \varphi
\mid \varphi \simp \varphi
\end{equation*}
Let $s \colon \Var \pto \Nat$ be a partial function called a store
and $h \colon \Nat \pto \Nat$ be a partial function called a heap.
Separation logic formulas are interpreted in the pair $(s,h)$
inductively as follows:
\begin{itemize}
\item $(s,h) \vDash_{\SL} \varphi$
if $s \vDash_{\FOL} \varphi$ and $\varphi$ is a first-order logic formula;\item $(s,h) \vDash_\SL \emp$ if the domain of $h$ is empty;
\item $(s,h) \vDash_\SL e_1 \mapsto e_2$
      if $\bar{s}(e_1) \neq 0$, the domain of $h$ is $\{ \bar{s}(e_1) \}$, and 
      $h( \bar{s}(e_1) ) = \bar{s}(e_2)$;
\item $(s,h) \vDash_\SL \varphi_1 \merge \varphi_2$ if
      there exist disjoint $h_1$ and $h_2$ such that
      $h = h_1 \merge h_2$ and
      $(s,h_1) \vDash_\SL \varphi_1$
      and $(s, h_2) \vDash_\SL \varphi_2$;
\item $(s,h) \vDash_\SL \varphi_1 \simp \varphi_2$ if
      for every $h_1$ disjoint with $h$,
      if $(s,h_1) \vDash_\SL \varphi_1$ then
      $(s,h_1 \merge h) \vDash_\SL \varphi_2$.
\end{itemize}
A separation logic formula $\varphi$ is valid,
denoted as $\vDash_\SL \varphi$,
if $(s,h)\vDash_\SL \varphi$ for every store $s$ and heap $h$.

Separation logic is faithfully captured by a matching logic theory
we which denote as $\MLSL$.
The theory $\MLSL$ has the signature
$\sig = (S, \Sigma)$ 
where $S$ contains a sort $\Nat$ for natural numbers and
a sort $\Map$ for heaps. 
The symbol set $\Sigma$ contains
a constant symbol $\emp \in \Sigma_{\lambda,\Map}$,
a binary symbol $* \in \Sigma_{\Nat\ \Nat,\Map}$,
and a binary symbol $\mapsto \in \Sigma_{\Map\ \Map,\Map}$.
We write $*$ and $\mapsto$ in mixfix form.
The separation conjunction is defined as an alias
\begin{equation*}
\varphi_1 \simp \varphi_2 \equiv
\exists h . ( h \wedge \floor{ h \merge \varphi_1 \imp \varphi_2 } )
\end{equation*}
With the above definitions,
\begin{center}
\em Any separation logic formula is a matching logic pattern of sort $\Map$
in the theory $\MLSL$.
\end{center}



Unlike in modal logic S5 where the theory $\MLSfive$ is defined with a set of
axioms,
the theory $\MLSL$ is defined by 
a particular matching logic model 
of the above signature
known as the intended model or the standard model, denoted as $\Map$.
The intended model $\Map$ has the set of natural numbers $\nats$ as the carrier
set of sort $\Nat$,
and the set of partial functions 
$ \{ h \mid h \colon \nats \pto \nats \} $
as the carrier set of sort $\Map$.
Symbols $\emp$, $\merge$, and $\mapsto$ are interpreted in $\Map$ in the
intended way.
Notice that separation logic formulas contain no free variables of sort $\Map$
as matching logic patterns,
so the valuation of variables of sort $\Map$ does not matter.
In addition, there is a one-to-one correspondence between 
Valuations of variables of sort $\Nat$ and separation logic states.
Under this correspondence, we can prove by structural induction that
\begin{center}
$(s,h) \vDash_\SL \varphi$
 if and only if 
$h \in \rhobar(\varphi)$
\quad \doubleslash $s$ and $\rho$ conform to the one-to-one correspondence
\end{center}
As a corollary,
$\vDash_\SL \varphi$ if and only if
$\MLSL \vDash \varphi$ for every separation logic formula $\varphi$.



\end{document}