\usepackage{comment}
\usepackage{booktabs}                           
\usepackage{subcaption} 
\usepackage{mathtools,amsthm,amssymb,prftree,amsfonts}
\usepackage{newtxtext}
\usepackage{newtxmath}
\usepackage[bbgreekl]{mathbbol} % must go after newtxtext & newtxmath
\usepackage{tikz}\usetikzlibrary{arrows}\usepgflibrary{arrows}
\usepackage{tabularx,multirow}
\usepackage{xspace}
\usepackage{wrapfig}
\usepackage{array}
\usepackage{todonotes}
\usepackage[normalem]{ulem}

% new theorems

\newtheorem{theorem}{Theorem}[section]
\newtheorem{proposition}[theorem]{Proposition}
\newtheorem{conjecture}[theorem]{Conjecture}
\newtheorem{lemma}[theorem]{Lemma}
\newtheorem{notation}[theorem]{Notation}

\theoremstyle{definition}
\newtheorem{definition}[theorem]{Definition}

\theoremstyle{remark}
\newtheorem{remark}[theorem]{Remark}
\newtheorem{example}[theorem]{Example}

% comments
\newcommand{\inlinetodo}[1]{\todo[inline]{#1}}
\newcommand{\todorosu}[1]{\todo[inline,color=green!10]{(Grigore). #1}}
\newcommand{\todochen}[1]{\todo[inline,color=blue!10]{(Xiaohong). #1}}

% angle braces < >
\newcommand{\la}{\langle}
\newcommand{\ra}{\rangle}

% texts
\newcommand{\z}{{\!\_\!}}
\newcommand{\nats}{\mathbb{N}} % set of natural numbers
\newcommand{\reals}{\mathbb{R}} % set of reals
\newcommand{\K}{$\mathbb{K}$\xspace} % \K framework
\newcommand{\MmL}{M$\mu$L\xspace}
\newcommand{\pfun}{\rightharpoonup} % partial function arrow
\newcommand{\dom}{\mathrm{dom}}
\newcommand{\curry}{\mathit{curry}}
\newcommand{\uncurry}{\mathit{uncurry}}

% sorts using mathit fonts
\newcommand{\Nat}{\mathit{Nat}}
\newcommand{\Map}{\mathit{Map}}
\newcommand{\Cfg}{\mathit{Cfg}}
\newcommand{\Bool}{\mathit{Bool}}

% prefix symbols using mathit fonts
\newcommand{\emp}{\mathit{emp}}
\newcommand{\lsegleft}{\lsegleftit}
\newcommand{\lsegright}{\lsegrightit}

% mathrm font
\newcommand{\cfg}{\mathrm{cfg}}
\newcommand{\lfp}{\mathrm{lfp}}
\newcommand{\gfp}{\mathrm{gfp}}

% mathit font
\newcommand{\State}{\mathit{State}}
\newcommand{\init}{\mathit{init}}
\newcommand{\boxit}{\mathit{box}}
\newcommand{\succit}{\mathit{succ}}
\newcommand{\listit}{\mathit{list}}
\newcommand{\col}{\mathit{col}}
\newcommand{\row}{\mathit{row}}

\newcommand{\lsegleftit}{\mathit{ls\hspace{-0.02em}e\hspace{-0.04em}gleft}}
\newcommand{\lsegrightit}{\mathit{ls\hspace{-0.02em}e\hspace{-0.04em}gright}}

% mathsf font
\newcommand{\LTL}{\mathsf{LTL}}
\newcommand{\hLTL}{\mathsf{hLTL}}

% mathbb font

% mathcal font
\newcommand{\Fcal}{\mathcal{F}}

% basic maths
\newcommand{\pset}[1]{\mathcal{P}(#1)} % powerset
\newcommand{\fixpoint}{\mathsc{Fixpoint}}
\newcommand{\FP}{\mathsc{fp}}
% an implementation of widebar
% obtained from [[https://tex.stackexchange.com/questions/16337/
% can-i-get-a-widebar-without-using-the-mathabx-package]]
\makeatletter
\let\save@mathaccent\mathaccent
\newcommand*\if@single[3]{%
	\setbox0\hbox{${\mathaccent"0362{#1}}^H$}%
	\setbox2\hbox{${\mathaccent"0362{\kern0pt#1}}^H$}%
	\ifdim\ht0=\ht2 #3\else #2\fi
}
%The bar will be moved to the right by a half of \macc@kerna, which is computed by amsmath:
\newcommand*\rel@kern[1]{\kern#1\dimexpr\macc@kerna}
%If there's a superscript following the bar, then no negative kern may follow the bar;
%an additional {} makes sure that the superscript is high enough in this case:
\newcommand*\widebar[1]{\@ifnextchar^{{\wide@bar{#1}{0}}}{\wide@bar{#1}{1}}}
%Use a separate algorithm for single symbols:
\newcommand*\wide@bar[2]{\if@single{#1}{\wide@bar@{#1}{#2}{1}}{\wide@bar@{#1}{#2}{2}}}
\newcommand*\wide@bar@[3]{%
	\begingroup
	\def\mathaccent##1##2{%
		%Enable nesting of accents:
		\let\mathaccent\save@mathaccent
		%If there's more than a single symbol, use the first character instead (see below):
		\if#32 \let\macc@nucleus\first@char \fi
		%Determine the italic correction:
		\setbox\z@\hbox{$\macc@style{\macc@nucleus}_{}$}%
		\setbox\tw@\hbox{$\macc@style{\macc@nucleus}{}_{}$}%
		\dimen@\wd\tw@
		\advance\dimen@-\wd\z@
		%Now \dimen@ is the italic correction of the symbol.
		\divide\dimen@ 3
		\@tempdima\wd\tw@
		\advance\@tempdima-\scriptspace
		%Now \@tempdima is the width of the symbol.
		\divide\@tempdima 10
		\advance\dimen@-\@tempdima
		%Now \dimen@ = (italic correction / 3) - (Breite / 10)
		\ifdim\dimen@>\z@ \dimen@0pt\fi
		%The bar will be shortened in the case \dimen@<0 !
		\rel@kern{0.6}\kern-\dimen@
		\if#31
		\overline{\rel@kern{-0.6}\kern\dimen@\macc@nucleus\rel@kern{0.4}\kern\dimen@}%
		\advance\dimen@0.4\dimexpr\macc@kerna
		%Place the combined final kern (-\dimen@) if it is >0 or if a superscript follows:
		\let\final@kern#2%
		\ifdim\dimen@<\z@ \let\final@kern1\fi
		\if\final@kern1 \kern-\dimen@\fi
		\else
		\overline{\rel@kern{-0.6}\kern\dimen@#1}%
		\fi
	}%
	\macc@depth\@ne
	\let\math@bgroup\@empty \let\math@egroup\macc@set@skewchar
	\mathsurround\z@ \frozen@everymath{\mathgroup\macc@group\relax}%
	\macc@set@skewchar\relax
	\let\mathaccentV\macc@nested@a
	%The following initialises \macc@kerna and calls \mathaccent:
	\if#31
	\macc@nested@a\relax111{#1}%
	\else
	%If the argument consists of more than one symbol, and if the first token is
	%a letter, use that letter for the computations:
	\def\gobble@till@marker##1\endmarker{}%
	\futurelet\first@char\gobble@till@marker#1\endmarker
	\ifcat\noexpand\first@char A\else
	\def\first@char{}%
	\fi
	\macc@nested@a\relax111{\first@char}%
	\fi
	\endgroup
}
\makeatother



% on usage of colons:
% - use \colon for functions. E.g., f: Int -> Int;
% - use : for binary relations with space around;
% - use \cln or binary relations without space around, x:s, f(Int):Int.
\newcommand{\cln}{\mathbin{\!:\!}}
\newcommand{\ldot}{\mathbin{\!.\!}} % lower dot, used in exists x . phi
\newcommand{\imp}{\to}
\newcommand{\dimp}{\leftrightarrow}
\newcommand{\To}{\Rightarrow}
\newcommand{\FV}{\mathit{FV}}

\newcommand{\mathsc}[1]{{\normalfont\textsc{#1}}} % small caps in math mode
\newcommand{\Name}{\textnormal{\textsc{Name}}}
\newcommand{\Var}{\mathsc{Var}}
\newcommand{\EVar}{\mathsc{EVar}}
\newcommand{\SVar}{\mathsc{SVar}}
\newcommand{\Nom}{\textnormal{\textsc{Nom}}}
\newcommand{\sig}{\mathbb{\Sigma}}
\newcommand{\Pattern}{\textnormal{\textsc{Pattern}}}

\DeclarePairedDelimiter{\bracket}{\llbracket}{\rrbracket}
\newcommand{\std}{\mathrm{std}}
\newcommand{\hnk}{\mathrm{hnk}}
\newcommand{\rhop}{\rho'}
\newcommand{\rhobar}{\bar{\rho}}
\newcommand{\rhobarp}{\widebar{\rhop}}

\newcommand{\until}{\mathbin{U}}
\newcommand{\slashslash}{//\xspace}


%%%%%%%%%%%%%%%%%%%%%%%%%%%%%%%%%%%%%%%%%%%%%%%%%

\newcommand{\rhox}[1]{{\rho_{#1}}}
\newcommand{\rhopx}[1]{{\rhop_{#1}}}
\newcommand{\rhobarx}[1]{\widebar{\rhox{#1}}}
\newcommand{\rhobarpx}[1]{\widebar{\rhopx{#1}}}
\newcommand{\simon}[1]{\overset{#1}{\sim}}
\newcommand{\simx}{\simon{x}}
\newcommand{\simz}{\simon{z}}
\newcommand{\sigmaM}{\sigma_M}
\newcommand{\sigmaW}{\sigma_W}
\newcommand{\sigmaY}{\sigma_Y}


\newcommand{\doubleslash}{//\xspace}
\newcommand{\CC}{\mathcal{C}}
\newcommand{\MM}{\mathcal{M}}
\newcommand{\WW}{\mathcal{W}}
\newcommand{\YY}{\mathcal{Y}}
\newcommand{\II}{\mathcal{I}}
\newcommand{\interp}[1]{\__{#1}}
\newcommand{\interpM}{\interp{\MM}}
\newcommand{\interpY}{\interp{\YY}}
\newcommand{\interpW}{\interp{\WW}}

\newcommand{\SigmaSub}[1]{\Sigma_{#1}}
\newcommand{\SetOF}[1]{\{ #1 \}}


\newcommand{\txtand}{\text{ and }}
\newcommand{\MLLambda}{\mathsf{\Lambda}}
\newcommand{\MLMu}{\mathsf{Mu}}
\newcommand{\MLLTL}{\mathsf{LTL}}
\newcommand{\MLCTL}{\mathsf{CTL}}
\newcommand{\MLSfive}{\mathsf{S5}}
\newcommand{\MLSL}{\mathsf{SL}}
\newcommand{\MLhybridML}{\mathsf{hybridML}}
\newcommand{\MLpolyadicML}{\mathsf{polyML}}
\newcommand{\MLTS}{\textnormal{\textsf{TS}}\xspace}
\newcommand{\MLUTS}{\textnormal{\textsf{UTS}}\xspace}
\newcommand{\MLinfLTL}{\mathsf{infLTL}}
\newcommand{\MLfinLTL}{\mathsf{finLTL}}
\newcommand{\MLPDL}{\mathsf{PDL}}
\newcommand{\MLuRL}{\mathsf{uRL}}
\newcommand{\MLRL}{\mathsf{RL}}
\newcommand{\hybridModalLogic}{\textit{hybrid}}
\newcommand{\polyadicModalLogic}{\textit{polyadic}}
\DeclarePairedDelimiter{\ceil}{\lceil}{\rceil}
\DeclarePairedDelimiter{\floor}{\lfloor}{\rfloor}
\newcommand{\infLTL}{\mathrm{infLTL}}
\newcommand{\finLTL}{\mathrm{finLTL}}
\newcommand{\uRL}{\textsc{uRL}}
\newcommand{\SSS}{\mathcal{S}}

% name of the proof rules
\newcommand{\prule}[1]{\textnormal{\textsc{(#1)}}}

\newcommand{\modusponens}{\prule{Modus Ponens}\xspace}
\newcommand{\universalgeneralization}{\prule{Universal Generalization}\xspace}
\newcommand{\necessitation}{\prule{Necessitation}\xspace}
\newcommand{\existence}{\prule{Existence}\xspace}
\newcommand{\singletonvariable}{\prule{Singleton Variable}\xspace}
\newcommand{\propagationbottom}{\prule{Propagation$_\bot$}\xspace}
\newcommand{\propagationvee}{\prule{Propagation$_\vee$}\xspace}
\newcommand{\propagationexists}{\prule{Propagation$_\exists$}\xspace}
\newcommand{\variablesubstitution}{\prule{Variable Substitution}\xspace}
\newcommand{\framing}{\prule{Framing}\xspace}
\newcommand{\propositionaltautology}{\prule{Propositional Tautology}\xspace}
\newcommand{\forallrule}{\prule{$\forall$}\xspace}
\newcommand{\membership}{\prule{Membership}\xspace}
\newcommand{\membershipintroduction}{\prule{Membership Introduction}\xspace}
\newcommand{\membershipelimination}{\prule{Membership Elimination}\xspace}
\newcommand{\membershipneg}{\prule{Membership$_\neg$}\xspace}
\newcommand{\membershipwedge}{\prule{Membership$_\wedge$}\xspace}
\newcommand{\membershipexists}{\prule{Membership$_\exists$}\xspace}
\newcommand{\equalityelimination}{\prule{Equality Elimination}\xspace}
\newcommand{\membershipsymbol}{\prule{Membership Symbol}\xspace}
\newcommand{\membershipvariable}{\prule{Membership Variable}\xspace}
\newcommand{\functionalsubstitution}{\prule{Functional Substitution}\xspace}
\newcommand{\circularity}{\prule{Circularity}\xspace}
\newcommand{\Lfp}{\prule{Lfp}\xspace}
\newcommand{\Gfp}{\prule{Gfp}\xspace}
\newcommand{\Fix}{\prule{Fix}\xspace}
\newcommand{\Fixmu}{\prule{Fix$_\mu$}\xspace}
\newcommand{\Fixnu}{\prule{Fix$_\nu$}\xspace}
\newcommand{\FIX}{\Fix}
\newcommand{\LFP}{\Lfp}
\newcommand{\xif}[1]{\xLeftarrow{\text{#1}}}




\newcommand{\simp}{\mathbin{-*}}
\newcommand{\merge}{\mathbin{*}}
\newcommand{\pto}{\rightharpoonup}
\newcommand{\SL}{{\text{SL}}}
\newcommand{\FOL}{{\text{FOL}}}
\newcommand{\TT}{\mathcal{T}}

% contexts
\newcommand{\CSub}[1]{C_{#1}}
\newcommand{\Csigma}{\CSub{\sigma}}
\newcommand{\Csigmai}{\CSub{\sigma,i}}
\newcommand{\Csigmaapp}[1]{\CSub{\sigma}[#1]}
\newcommand{\Csigmaiapp}[1]{\CSub{\sigma,i}[#1]}
\newcommand{\Capp}[1]{C[#1]}

% compliment of symbols contexts
\newcommand{\Csigmabar}{\overline{\CSub{\sigma}}}
\newcommand{\sigmabar}{\bar{\sigma}}
\newcommand{\Cbar}{\bar{C}}

\newcommand{\Prop}{\textsc{Prop}}
\newcommand{\AP}{\textsc{AP}}
\newcommand{\APgm}{\textsc{APgm}}
\newcommand{\Label}{\textsc{Label}}
\newcommand{\true}{\mathit{true}}
\newcommand{\false}{\mathit{false}}
\newcommand{\wnext}{{\circ}}
\newcommand{\snext}{{\bullet}}
\newcommand{\sprev}{{\widebar{\snext}}}
\newcommand{\wprev}{{\widebar{\wnext}}}
\newcommand{\always}{{\square}}
\newcommand{\eventually}{{\lozenge}}
\newcommand{\AAA}{{\mathsf{A}}}
\newcommand{\EEE}{{\mathsf{E}}}
\newcommand{\EE}{{\mathsf{E}}}
\newcommand{\XX}{{\mathsf{X}}}
\newcommand{\UU}{\mathbin{\mathsf{U}}}
\newcommand{\FF}{{\mathsf{F}}}
\newcommand{\GG}{{\mathsf{G}}}
\newcommand{\AG}{{\mathsf{AG}}}
\newcommand{\EG}{{\mathsf{EG}}}
\newcommand{\AF}{{\mathsf{AF}}}
\newcommand{\EF}{{\mathsf{EF}}}
\newcommand{\AX}{\mathsf{AX}}
\newcommand{\AU}{\mathbin{\mathsf{AU}}}
\newcommand{\EU}{\mathbin{\mathsf{EU}}}
\newcommand{\EX}{\mathsf{EX}}
\newcommand{\Us}{\mathbin{\mathsf{U}_s}}
\newcommand{\Uw}{\mathbin{\mathsf{U}_w}}
\newcommand{\rt}{\mathsf{root}}
\newcommand{\subtrees}{\mathsf{subtrees}}
\newcommand{\fullpaths}{\mathsf{fullpaths}}
\newcommand{\world}{\mathit{world}}
\newcommand{\statesort}{\mathit{state}}
\newcommand{\pgm}{\mathit{pgm}}
\newcommand{\hole}{\square}

\newcommand{\bracketM}[1]{{\bracket{#1}_\MM}}
\DeclarePairedDelimiter{\angleBraces}{\langle}{\rangle}

\newcommand{\Words}{\textsc{Words}}
\newcommand{\infTraces}{\textsc{Traces$^\omega$}\xspace}
\newcommand{\finTraces}{\textsc{Traces$^*$}\xspace}
\newcommand{\infTrees}{\textsc{Trees$^\omega$}\xspace}
\newcommand{\Inf}{\prule{Inf}\xspace}
\newcommand{\Fin}{\prule{Fin}\xspace}
\newcommand{\Lin}{\prule{Lin}\xspace}
\newcommand{\LL}{\mathcal{L}}
\newcommand{\NN}{\mathbb{N}}
\newcommand{\QED}{\text{QED}}

\newcommand{\textiff}{\text{ iff }}
\newcommand{\xto}[1]{\xrightarrow{#1}}
\newcommand{\xtoa}{\xto{a}}
\newcommand{\pred}{\mathit{pred}}
\newcommand{\succc}{\mathit{succ}}
\newcommand{\CTL}{{\textsc{CTL}}}
\newcommand{\PDL}{\textsc{PDL}}
\newcommand{\textthen}{\text{ then }}
\newcommand{\PDLseq}{\mathbin{;}}
\newcommand{\PDLunion}{\mathbin{\cup}}
\newcommand{\PDLstar}{^*}
\newcommand{\PDLquestion}{{?}}


\newcommand{\sfk}{\textsf{k}\xspace}
\newcommand{\sfstate}{\textsf{state}\xspace}
\newcommand{\ttInt}{\texttt{Int}\xspace}
\newcommand{\ttId}{\texttt{Id}\xspace}
\newcommand{\ttExp}{\texttt{Exp}\xspace}
\newcommand{\ttplus}{\ \texttt{+}\ }
\newcommand{\ttminus}{\ \texttt{-}\ }
\newcommand{\KResult}{\texttt{KResult}\xspace}

\newcommand{\Int}{\textit{Int}}
\newcommand{\Exp}{\textit{Exp}}
\newcommand{\AExp}{\textit{BExp}}
\newcommand{\BExp}{\textit{BExp}}
\newcommand{\ttrue}{\textit{true}}
\newcommand{\ffalse}{\textit{false}}
\newcommand{\Id}{\textit{Id}}
\newcommand{\Stmt}{\textit{Stmt}}
\newcommand{\ite}{\textit{ite}}

\newcommand{\parametric}[2]{{#1}\raisebox{.2ex}{\texttt{\footnotesize{\{}}}#2\raisebox{.2ex}{\texttt{\footnotesize{\}}}}}
\newcommand{\parametricscript}[2]{{#1}\raisebox{.2ex}{\texttt{\tiny{\{}}}#2\raisebox{.2ex}{\texttt{\tiny{\}}}}}

\newcommand{\Context}{\mathit{Context}}
\newcommand{\compose}{\circ}
\newcommand{\strict}[1]{\textsf{strict(#1)}}