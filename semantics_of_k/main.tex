\documentclass[UTF8]{article}

  \usepackage[linesnumbered,ruled,vlined]{algorithm2e}

  %\usepackage{xcolor}
  \usepackage[pdftex,dvipsnames]{xcolor}  % Coloured text etc.
  \usepackage{xargs}                      % Use more than one optional
                                          % parameter in a new commands

  % Select what to do with todonotes: 
  % \usepackage[disable]{todonotes} % notes not showed
  %  \usepackage[draft]{todonotes} % notes showed
  \usepackage[colorinlistoftodos,prependcaption,textsize=tiny]{todonotes}
  \newcommandx{\unsure}[2][1=]
    {\todo[linecolor=red,backgroundcolor=red!25,bordercolor=red,#1]{#2}}
  \newcommandx{\change}[2][1=]
    {\todo[linecolor=blue,backgroundcolor=blue!25,bordercolor=blue,#1]{#2}}
  \newcommandx{\info}[2][1=]
    {\todo[linecolor=OliveGreen,backgroundcolor=OliveGreen!25,bordercolor=OliveGreen,#1]{#2}}
  \newcommandx{\improvement}[2][1=]
    {\todo[linecolor=Plum,backgroundcolor=Plum!25,bordercolor=Plum,#1]{#2}}
  \newcommandx{\thiswillnotshow}[2][1=]
    {\todo[disable,#1]{#2}}

  % Select what to do with command \comment:  
  % \newcommand{\comment}[1]
      {}                                    %comment not showed
  \newcommand{\comment}[1]
    {\par {\bfseries \color{blue} #1 \par}} %comment showed
    
  \usepackage{amsmath}
  \usepackage{amssymb}
  
  \usepackage{amsthm}
  
  % Declare a global counter for theorem environments:
  \newcounter{thmcounter}
  
  % Define new theorem styles:
  \theoremstyle{plain}
  \newtheorem{theorem}[thmcounter]{Theorem}
  \newtheorem{corollary}[thmcounter]{Corollary}
  \newtheorem{lemma}[thmcounter]{Lemma}
  \newtheorem{proposition}[thmcounter]{Proposition}
  \theoremstyle{definition}
  \newtheorem{definition}[thmcounter]{Definition}
  \newtheorem{example}[thmcounter]{Example}
  \theoremstyle{remark}
  \newtheorem{remark}[thmcounter]{Remark}
  \newtheorem{notation}[thmcounter]{Notation}
  
  
  % Package for changing fonts in the Verbatim environment:
  \usepackage{xcolor}
  \usepackage{fancyvrb}
  
  % Package for writing captions for align environment:
  \usepackage{capt-of}
  
  % Package for URLs:
  \usepackage{hyperref}  
  
  % Define serif fonts for ML theories used in math mode:
  \newcommand{\PA}{\mathsf{PA}}
  \newcommand{\SEQ}{\mathsf{SEQ}}
  \newcommand{\HEAP}{\mathsf{HEAP}}
  \newcommand{\IMP}{\mathsf{IMP}}
  \newcommand{\FIX}{\mathsf{FIX}}
  \newcommand{\LAMBDA}{\mathsf{LAMBDA}}
  \newcommand{\CTXT}{\mathsf{CTXT}}
  \newcommand{\DEF}{\mathsf{DEF}}
  \newcommand{\METALEVEL}{\mathsf{META-LEVEL}}
  
  % Define serif fonts for symbols:
  \newcommand{\impite}{\mathsf{ite}}
  \newcommand{\impwhile}{\mathsf{while}}
  \newcommand{\imptt}{\mathsf{tt}}
  \newcommand{\impff}{\mathsf{ff}}
  \newcommand{\impskip}{\mathsf{skip}}
  \newcommand{\impseq}{\mathsf{seq}}
  \newcommand{\impasgn}{\mathsf{asgn}}
  
  \newcommand{\impmapsto}{\mathsf{mapsto}}
  \newcommand{\impmerge}{\mathsf{merge}}
  
  \newcommand{\up}{\mathsf{\#up}}
  \newcommand{\down}{\mathsf{\#down}}
  \newcommand{\isGround}{\mathsf{isGround}}
  \newcommand{\xfalse}{\mathsf{false}}
  \newcommand{\xtrue}{\mathsf{true}}
  
  
  % Define identity context
  \newcommand{\I}{\mathsf{I}}
  
  % Define the colon ":" that is used in "x:s"
  % with less spacing around.
  \newcommand{\cln}{{:}}
 
  
  % Define ceiling and flooring symbols:
  \usepackage{mathtools}
  \DeclarePairedDelimiter{\ceil}{\lceil}{\rceil}

  % Package for underlining and strikethrough texts.
  \usepackage[normalem]{ulem}
  
  % Define text over equality
  \usepackage{mathtools}
  \newcommand{\xeq}[1]
    {\stackrel{\mathclap{\normalfont\tiny\mbox{#1}}}{=}}
    
  % Package for display-mode quotations.
  \usepackage{csquotes}
  
  % Define double-bracket [[P]] (known as the semantics bracket)
  \usepackage{stmaryrd}
  \newcommand{\Bracket}[1]
    {\llbracket#1\rrbracket}
    
  % Package for writing inference rules and formal proofs
  \usepackage{proof}
  % Writing labels about inference rules
  \newcommand{\rl}[1]{\text{\scriptsize{(#1)}}}


  % Define sorts and symbols in the calculus K.
  \newcommand{\KBool}{\mathit{KBool}}
  \newcommand{\Ktrue}{\mathit{Ktrue}}
  \newcommand{\Kfalse}{\mathit{Kfalse}}
  \newcommand{\KandBool}{\mathit{KandBool}}
  \newcommand{\KorBool}{\mathit{KorBool}}
  \newcommand{\KnotBool}{\mathit{KnotBool}}
  \newcommand{\KimpliesBool}{\mathit{KimpliesBool}}
  
  \newcommand{\KString}{\mathit{KString}}
    
  \newcommand{\KSort}{\mathit{KSort}}
  \newcommand{\KSortList}{\mathit{KSortList}}
  \newcommand{\Ksort}{\mathit{Ksort}}
  
  \newcommand{\KSymbol}{\mathit{KSymbol}}
  \newcommand{\KSymbolList}{\mathit{KSymbolList}}
  \newcommand{\Ksymbol}{\mathit{Ksymbol}}
  
  \newcommand{\KPattern}{\mathit{KPattern}}
  \newcommand{\KPatternList}{\mathit{KPatternList}}
  \newcommand{\Kvariable}{\mathit{Kvariable}}
  \newcommand{\Kand}{\mathit{Kand}}
  \newcommand{\Kor}{\mathit{Kor}}
  \newcommand{\Kimplies}{\mathit{Kimplies}}
  \newcommand{\Kiff}{\mathit{Kiff}}
  \newcommand{\Knot}{\mathit{Knot}}
  \newcommand{\Kapplication}{\mathit{Kapplication}}
  \newcommand{\Kexists}{\mathit{Kexists}}
  \newcommand{\Kforall}{\mathit{Kforall}}
  \newcommand{\Kequals}{\mathit{Kequals}}
  \newcommand{\Kcontains}{\mathit{Kcontains}}
  \newcommand{\Ktop}{\mathit{Ktop}}
  \newcommand{\Kbottom}{\mathit{Kbottom}}
  
  \newcommand{\KgetFvs}{\mathit{KgetFvs}}
  \newcommand{\KfreshName}{\mathit{KfreshName}}
  
  \newcommand{\KSignature}{\mathit{KSignature}}
  \newcommand{\Ksignature}{\mathit{Ksignature}}
  
  \newcommand{\KwellFormed}{\mathit{KwellFormed}}
  \newcommand{\KgetSort}{\mathit{KgetSort}}
  \newcommand{\KisSort}{\mathit{KisSort}}
  \newcommand{\Ksubstitute}{\mathit{Ksubstitute}}
  
  \newcommand{\KTheory}{\mathit{KTheory}}
  \newcommand{\Ktheory}{\mathit{Ktheory}}
  
  \newcommand{\Kdeduce}{\mathit{Kdeduce}}

  % Title and authors
  \title{The Semantics of K}
  \author{Formal Systems Laboratory \\
          University of Illinois}

\begin{document}

\maketitle

\comment{Please feel free to contribute to this report in all ways.
You could add new contents, remove redundant ones, refactor and
organize the texts, and correct typos, but please follow the FSL rules for editing, though; e.g., $<$80 characters per line,
each sentence on a new line, etc. }

\section{Matching Logic}

\newcommand{\Var}{\textit{Var}}
\newcommand{\Nat}{\textit{Nat}}

Let us recall the basic grammar of matching logic
from~\cite{rosu-2017-lmcs}.\improvement{Add references.}
~
Assume a matching logic \emph{signature} $(S, \Sigma)$, and
let $\Var_s$ be a countable set of
\emph{variables} of sort $s$, where the sets of \emph{sorts} $S$
and of \emph{symbols} $\Sigma$ are enumerable sets.
We partition $\Sigma$ in sets of symbols
$\Sigma_{s_1 \ldots s_n, s}$ of \emph{arity} $s_1\ldots s_n,s$, where
$s_1,\ldots, s_n, s \in S$.
Then \emph{patterns} of sort $s \in S$ are generated by the following grammar:
\begin{align*}
\varphi_s \Coloneqq\  &x \cln s \quad \text{where $x \in \Var$} \\
\mid\  &\varphi_s \wedge \varphi_s \\
\mid\  &\neg \varphi_s \\
\mid\  &\exists x \cln s' . \varphi_s \quad \text{where $x \in N$ and $s' \in S$} \\
\mid\  &\sigma(\varphi_{s_1},\dots,\varphi_{s_n}) \quad \text{where $\sigma \in \Sigma$ has $n$ arguments, and \dots}
\end{align*}
\begingroup\vspace*{-\baselineskip}
\captionof{figure}{The grammar of matching logic.}
\label{ml-grammar}
\vspace*{\baselineskip}\endgroup

The grammar above only defines the syntax of (well-formed) patterns of sort
$s$.
It says nothing about their semantics.
For example, patterns $x\cln s \wedge y \cln s$ and
$y\cln s \wedge x \cln s$ are distinct elements in the language
of the grammar, in spite of them being semantically/provably equal
in matching logic.

For notational convenience, we take the liberty to use mix-fix syntax for
operators in $\Sigma$,
parentheses for grouping, and omit variable sorts when understood.
For example, if $\Nat \in S$ and
$\_+\_, \_*\_ \in \Sigma_{\Nat \times \Nat, \Nat}$
then we may write $(x + y)*z$ instead of
$\_*\_(\_+\_(x\cln\Nat,y\cln\Nat),z\cln\Nat)$.
More notational convenience and conventions will be introduced along the way as use them. 

A matching logic \emph{theory} is a triple $(S, \Sigma, A)$ where
$(S,\Sigma)$ is a signature and $A$ is a set of patterns called \emph{axioms}.
Like in many logics, sets of patterns may be presented as \emph{schemas}
making use of meta-variables ranging over patterns, sometimes constrained
to subsets of patterns using side conditions.
For example:
$$
\begin{array}{rl}
\varphi[\varphi_1/x] \wedge (\varphi_1 = \varphi_2) \rightarrow \varphi[\varphi_2/x]
&\textrm{where $\varphi$ is any pattern and $\varphi_1$, $\varphi_2$} \\
& \textrm{are any patterns of same sort as $x$}
\\[2ex]
(\lambda x . \varphi)\varphi' = \varphi[\varphi'/x]
& \textrm{where $\varphi$, $\varphi'$ are \emph{syntactic patterns}, that is,}
\\
& \textrm{ones formed only with variables and symbols}
\\
& \textrm{\color{blue} This is not true. Pattern $\varphi$ contains quantifiers.}
\\[2ex]
\varphi_1 \mathrel{\texttt{+}} \varphi_2 = \varphi_1 +_\Nat \varphi_2
& \textrm{where $\varphi$, $\varphi'$ are \emph{ground} syntactic patterns}
\\
&\textrm{of sort $\Nat$, that is, patterns built only}
\\
&\textrm{with symbols \texttt{zero} and \texttt{succ}}
\\[2ex]
(\varphi_1 \rightarrow \varphi_2) \rightarrow
(\varphi[\varphi_1 / x] \rightarrow \varphi[\varphi_2 / x])
& \textrm{where $\varphi$ is a \emph{positive context in $x$}, that is,}
\\
& \textrm{a pattern
containing only one occurrence}
\\
&\textrm{of $x$ with no negation ($\neg$) on the path to}
\\
&\textrm{$x$, and where $\varphi_1$, $\varphi_2$ are any patterns}
\\
&\textrm{having the same sort}
\end{array}
$$

One of the major goals of this paper is to propose a formal language
and an implementation, that allows us to write such pattern schemas.

\section{A Calculus of Matching Logic}

In this section, we define a matching logic theory $K = (S_K, \Sigma_K, A_K)$ 
as \emph{the calculus of matching logic},
where $S_K, \Sigma_K$, and $A_K$ are sets of sorts, symbols, and axioms, 
respectively.

\subsection{Boolean Algebra}

The matching logic theory of Boolean algebra is included in $K$,
and the corresponding sort is named $\KBool \in S_K$.
Constructors of the sort $\KBool$ are two functional symbols
\begin{equation*}
\Ktrue \colon \to \KBool \qquad \Kfalse \colon \to \KBool.
\end{equation*}
Common Boolean operators are defined as functional symbols with their
corresponding axioms
\begin{align*}
&\KnotBool \colon \KBool \to \KBool
&\KnotBool(\Ktrue) = \Kfalse
\\
&\KandBool \colon \KBool \times \KBool \to \KBool
&\KnotBool(\Kfalse) = \Ktrue
\\
&\KorBool \colon \KBool \times \KBool \to \KBool
&\KandBool(\Ktrue, b) = b
\\
&\KimpliesBool \colon \KBool \times \KBool \to \KBool
&\KandBool(\Kfalse, b) = \Kfalse
\end{align*}
The symbols $\KorBool$ and $\KimpliesBool$ are defined in terms of the symbols
$\KnotBool$ and $\KandBool$ in the usual way
\begin{align*}
&\KorBool(b_1, b_2) = \KnotBool(\KandBool(\KnotBool(b_1), \KnotBool(b_2)))
\\
&\KimpliesBool(b_1, b_2) = \KorBool(\KnotBool(b_1), b_2).
\end{align*}

\begin{notation}
	If $b$ is a term pattern of sort $\KBool$, then we will write just $b$ instead of $b = \Ktrue$ so that we can use Boolean expressions in any sort context.
\end{notation}

\begin{notation}
	To write Boolean expressions compactly, we adopt the following abbreviations if there is no confusion
	\begin{align*}
	&\neg b \equiv \KnotBool(b)
	\\
	&b_1 \wedge b_2 \equiv \KandBool(b_1, b_2)
	\\
	&b_1 \vee b_2 \equiv \KorBool(b_1, b_2)
	\\
	&b_1 \to b_2 \equiv \KimpliesBool(b_1, b_2).
	\end{align*}
\end{notation}

\subsection{Strings}
Introduce the matching logic theory for strings here.



\subsection{Matching Logic Sorts and Symbols}

The sort $\KSort$ is the sort for matching logic sorts.
The only constructor of the sort $\KSort$ is the functional symbol:
\begin{equation*}
\Ksort \colon \KString \to \KSort.
\end{equation*}

The sort $\KSymbol$ is the sort for matching logic symbols.
The only constructor of the sort $\KSymbol$ is the functional symbol:
\begin{equation*}
\Ksymbol \colon \KString \to \KSymbol.
\end{equation*}

\subsection{Lists}
\improvement{This is just a notation convention which allows us to have sorts like $\KPatternList$ and $\mathit{KSortList}$ without defining each of them. We are NOT introducing any parametric modules here.}
Whenever we introduce a sort, say $\mathit{X}$, to $S_K$, we feel free to 
use $\mathit{XList}$ as the sort of lists over $\mathit{X}$ 
with the following symbols implicitly declared:
\begin{align*}
&\mathit{nilXList} \ \colon \to \mathit{XList} \\
&\mathit{appendXList} \ \colon \mathit{XList} \times \mathit{XList} \to 
\mathit{XList}\\
&\mathit{inXList} \ \colon \mathit{X} \times \mathit{XList} \to \KBool \\
&\mathit{XListAsX} \ \colon \mathit{X} \to \mathit{XList},
\end{align*}
with axioms saying that $\mathit{appendXList}$ is associative and $\mathit{nilXList}$ is its identity.
We adopt the following shorthands:
\begin{align*}
&\mathit{nil}  \quad \text{as a shorthand of $\mathit{nilXList}$} \\
&\varphi_e \in \varphi_l \quad \text{as a shorthand of $\mathit{inXList}(\varphi_e, \varphi_l) = \Ktrue$}\\
&\varphi_e \not\in \varphi_l \quad \text{as a shorthand of $\mathit{inXList}(\varphi_e, \varphi_l) = \Kfalse$}\\
&\mathit{appendXList}()  \quad \text{as a shorthand of $\mathit{nil}$} \\
&\mathit{appendXList}(\varphi) \quad \text{as a shorthand of $\mathit{XListAsX}(\varphi)$}\\
&\mathit{appendXList}(\varphi_1,\dots,\varphi_n) \quad \text{as a shorthand of}\\ 
&\qquad \mathit{appendXList}(\mathit{XListAsX}(\varphi_1),\\
&\qquad \mathit{appendXList}(\mathit{XListAsX}(\varphi_2), \\
&\qquad \qquad \dots,\\
&\qquad \mathit{appendXList}(\mathit{XListAsX}(\varphi_n), \mathit{nil})))
\text{ when $n\ge 2$.}
\end{align*}

\subsection{Matching Logic Patterns}

The sort $\KPattern$ is the sort for matching logic patterns.
Constructors of the sort $\KPattern$ are the following functional symbols:
\begin{align*}
&\Kvariable \colon \KString \times \KSort \to 
\KPattern
\\
&\Kand, \Kor, \Kimplies, \Kiff \colon \KPattern \times \KPattern \times \KSort \to \KPattern
\\
&\Knot \colon \KPattern \times \KSort \to \KPattern 
\\
&\Kapplication \colon \KSymbol \times \KPatternList \to \KPattern 
\\
&\Kexists, \Kforall \colon \KString \times \KSort \times \KPattern \times \KSort \to \KPattern 
\\
&\Kequals, \Kcontains \colon \KPattern \times 
\KPattern \times \KSort \times \KSort \to \KPattern \\
&\Ktop, \Kbottom \colon \KSort \to \KPattern. \\
\end{align*}

\begin{notation}
	As a convention, we use $b$ for $\KBool$ variables, $x, y, z$ for $\KString$ variables, $s$ for $\KSort$ variables, $\sigma$ for $\KSymbol$ variables, and $p, q, r$ for $\KPattern$ variables. 
\end{notation}

The functional symbol 
$$\KgetFvs \colon \KPattern \to \KPatternList$$
collects all free variables in a pattern. \todo{Add corresponding axioms.}

The functional symbol
$$\KfreshName \colon \KPatternList \to \KString$$ 
generates a variable name that does not occur free in the argument patterns. \todo{Add corresponding axioms.}

The symbol
$$\Ksubstitute \colon \KPattern \times \KPattern \times \KPattern \to \KPattern$$
takes a target pattern $\varphi$, a ``find''-pattern $\psi_1$, and a ``replace''-pattern $\psi_2$, and returns $\varphi[\psi_2 / \psi_1]$. 
The following axioms define $\Ksubstitute$:\todo{Add corresponding axioms.}
\begin{align*}
&\Ksubstitute(r, q, r) = q\\
&\Ksubstitute(\Kand(p_1, p_2), q, r) \\ 
&\quad = \Kand(\Ksubstitute(p_1, q, r), \Ksubstitute(p_2, q, r))\\
&\Ksubstitute(\Kor(p_1, p_2), q, r) \\
&\quad = \Kor(\Ksubstitute(p_1, q, r), \Ksubstitute(p_2, q, r))\\
&\dots\\
&\Ksubstitute(\Kexists(x\cln\mathit{String}, s, p), q, r)\\
&\quad = \Kexists(\KfreshName(p, q, r), s,\\
&\qquad \quad \Ksubstitute((\Ksubstitute(p, \Kvariable(\KfreshName(p, q, r), s), \\
&\qquad \quad \quad \Kvariable(x\cln\mathit{String}, s), q, r))\\
&\dots
\end{align*}

\subsection{Matching Logic Signatures}

The sort $\KSignature$ is the sort for matching logic signatures, and it has just one constructor symbol:
\begin{equation*}
\Ksignature \colon \KSortList \times \KSymbolList \to \KSignature.
\end{equation*}

The functional symbol $\KwellFormed \colon \KPattern \times \KSignature \to \KBool$ return $\Ktrue$ if the argument pattern is well-formed in the argument signature. The corresponding axioms are:\todo{Add corresponding axioms.}
\begin{equation*}
TODO
\end{equation*}

\subsection{Matching Logic Theories}

\todo{WIP}

The sort $\KTheory$ is the sort of matching logic theories. The only constructor symbol is 
$$\Ktheory \colon \KSignature \times \KPatternList \to \KTheory.$$

\subsection{Matching Logic Proof System}

The functional symbol
$$
\Kdeduce \colon \KTheory \times \KPattern \to \KBool
$$
returns $\Ktrue$ if the argument pattern is deducible in the argument theory.

A proof system is a theorem generator. 
In $K$, the proof system of matching logic is captured by the functional symbol 
$\mathit{deducible} \colon \KPattern \to \KBool$, which returns $\Ktrue$ 
iff the argument pattern is a theorem. 

We introduce the double bracket $\Bracket{\_}$, known as the semantics bracket, as follows:
\begin{equation*}
\Bracket{\varphi} \equiv 
\left(\textit{deducible}\left(\mathit{lift}[\varphi]\right) = true\right).
\end{equation*}
Intuitively, $\Bracket{\varphi}$ means that ``$\varphi$ is deducible''.
Whenever there is an inference rule (axioms are considered as rules with zero 
premise)
$$
	\infer{\psi}
	{\varphi_1,\dots,\varphi_n}
$$
in matching logic, there is a corresponding axiom in $K$:
$$
	\Bracket{\varphi_1} \wedge \dots \wedge \Bracket{\varphi_n} \to \Bracket{\psi}.
$$
Inference modulo theories can be considered in the same way. 
For any (syntactic) matching logic theory $T$ whose axiom set is $A$, we add
$$
\Bracket{\varphi} \quad \text{for all $\varphi \in A$}
$$
as axioms to $K$. We sometimes denote the extended theory as $\mathit{lift}[T]$ 
and call it 
the \emph{meta-theory for} $T$.

\subsection{Faithfulness}
It remains a question whether the calculus $K$ faithfully captures matching logic reasoning. 
The following definition of \emph{faithfulness} is inspired by~\cite{?}.
\begin{definition}
	The calculus $K$ is said to be faithful for matching logic, if for any 
	matching logic syntactic theory $T$ and its meta-theory $\mathit{lift}[T]$,
	\begin{equation*}
	  \text{$\varphi$ is a theorem in $T$ iff $\Bracket{\varphi}$ is a theorem 
	  in $\mathit{lift}[T]$, for any pattern $\varphi$.}
	\end{equation*}
\end{definition}
\begin{theorem}
	The calculus $K$ is faithful for matching logic.
\end{theorem}
\begin{proof}
	TBC.
\end{proof}

Having a faithful calculus for matching logic has at least the following two 
benefits. 
Firstly, any implementation of the calculus is guaranteed to be able to conduct 
any reasoning in matching logic. Secondly, it allows us to define a matching 
logic theory $T$ by defining its meta-theory $\mathit{lift}[T]$ in the 
calculus $K$. 
The second 
point is of great importance if we want a formal language to define matching 
logic theories. 
We notice that there are many theories whose definitions involve notations that 
do not belong to the logic itself. For example, in the ($\beta$) axiom
$$
\lambda x . e [e'] = e[e' / x] \quad \text{where $e$ and $e'$ are 
$\lambda$-terms},
$$
we use the notation for substitution $\_[\_/\_]$, meta-variables $e$ and $e'$, 
and their range ``$\lambda$-terms''. 
None of those can be given a formal semantics in the object-logic, but can be 
defined in the calculus $K$.

\section{The Kore Language}

We have shown $K$, a calculus for matching logic in which we can specify 
everything about matching logic and matching logic theories, such as 
whether a pattern is well-formed, what sort a patter has, which patterns are 
deducible, free variables, fresh variables generation, substitution, etc.
The calculus $K$ provides a universe of pattern ASTs and the sound and complete 
proof system of matching logic. 
On the other hand, it is usually easier to work at object-level rather than 
meta-level. 
Even if all reasoning in a matching logic theory $T$ can be faithfully lifted 
to and conducted in its meta-theory $\mathit{lift}[T]$, it does not mean one 
should always do so. 

The Kore language is proposed to define matching logic theories using the 
calculus $K$. 
At the same time, it also provides a nice surface syntax 
(syntactic sugar) to write object-level patterns. 
We will firstly show the formal grammar of Kore in 
Section~\ref{sec:syntax-of-kore}, followed by some examples in 
Section~\ref{sec:examples-of-kore}.
After that, we will introduce a transformation from Kore definitions to 
meta-theories as the formal semantics of Kore in 
Section~\ref{sec:semantics-of-kore}.

\subsection{Syntax and Semantics of Kore}
\label{sec:syntax-of-kore}

\begin{Verbatim}[fontsize=\small]
// Namespaces for sorts, variables, metavariables,
// symbols, and Kore modules.
Sort           = String
VariableId     = String
MetaVariableId = String
Symbol         = String
ModuleId       = String

Variable       = VariableId:Sort
MetaVariable   = MetaVariableId::Sort

Pattern        = Variable | MetaVariable
               | \and(Pattern, Pattern)
               | \not(Pattern)
               | \exists(Variable, Pattern)
               | Symbol(PatternList)

Sentence       = import ModuleId
               | syntax Sort
               | syntax Sort ::= Symbol(SortList)
               | axiom Pattern
Sentences      = Sentence | Sentences Sentences

Module         = module ModuleId
                   Sentences
                 endmodule
\end{Verbatim}

In Kore syntax, the backslash ``\verb|\|'' is reserved for matching logic connectives and the sharp ``\verb|#|'' is reserved for the meta-level, i.e., the $K$ sorts and symbols. 
Therefore, the sorts $\KBool$, $\KString$, $\KSymbol$, $\KSort$, and 
$\KPattern$ in the calculus $K$ are denoted as \verb|#Bool|, \verb|#String|, 
\verb|#Symbol|, \verb|#Sort|, and \verb|#Pattern| in Kore respectively.
Symbols in $K$ are denoted in the similar way, too. 
For example, the constructor symbol $\Kvariable \colon \KString \times \KSort \to \KPattern$ is denoted as \verb|#variable| in Kore. 

A Kore module definition begins with the keyword \verb|module| followed by the name of the module-being-defined, and ends with the keyword \verb|endmodule|. The body of the definition consists of some \emph{sentences}, whose meaning are introduced in the following.

The keyword \verb|import| takes an argument as the name of the module-being-imported, and looks for that module in previous definitions. 
If the module is found, the body of that module is copied to the current module.
Otherwise, nothing happens. 
The keyword \verb|syntax| leads a \emph{syntax declaration}, which can be either a \emph{sort declaration} or a \emph{symbol declaration}.
Sorts declared by sort declarations are called \emph{object-sorts}, in comparison to the five \emph{meta-sorts}, \verb|#Bool|, \verb|#String|, \verb|#Symbol|, \verb|#Sort|, and \verb|#Pattern|, in $K$. 
Symbols whose argument sorts and return sort are all object-sorts (meta-sorts) are called \emph{object-symbols} (\emph{meta-sorts}).

Patterns are written in prefix forms. 
A pattern is called an \emph{object-pattern} (\emph{meta-pattern}) if all sorts and symbols in it are object (meta) ones.
Meta-symbols will be added to the calculus $K$, while object-sorts and object-symbols will not.
They only serve for the purpose to parse an object pattern. 

The keyword \verb|axiom| takes a pattern and adds an axiom to the calculus $K$.
If the pattern is a meta-pattern, it adds the pattern itself as an axiom.
If the pattern $\varphi$ is an object-pattern, it adds $\Bracket{\varphi}$ as an axiom to the calculus $K$.

Recall that we have defined the semantics bracket as
\begin{equation*}
\Bracket{\varphi} \equiv 
\left(\textit{deducible}\left(\mathit{lift}[\varphi]\right) = true\right),
\end{equation*}
where $\varphi$ is a pattern of the grammar in Figure~\ref{ml-grammar}.
However, here in Kore we allow $\varphi$ containing \emph{meta-variables}.
As a result, we modify the definition of the semantics bracket as $$\Bracket{\varphi} \equiv \mathit{mvsc}[\varphi] \to (deducible\left(\mathit{lift}[\varphi]\right) = true),$$
where the lifting function $\mathit{lift}[\_]$ and the meta-variable sort constraint $\mathit{mvsc}[\_]$ are defined in Algorithm~\ref{alg:liftingfunction} and~\ref{alg:mvsc}, respectively.
Intuitively, meta-variables in an object-pattern $\varphi$ are lifted to variables of the sort $\KPattern$ with the corresponding sort constraints. 
For example, the meta-variable $x \cln\cln s$ is lifted to a variable $x \cln \KPattern$ in $K$ with the constraint that $\KgetSort(x \cln \KPattern) = sort(s)$. The function $\mathit{mvsc}[\_]$ collects all such meta-variable sort constraint in an object-pattern is implemented in Algorithm~\ref{alg:mvsc}.

\begin{algorithm}
	\KwIn{An object-pattern $\varphi$.}
	\KwOut{The meta-representation (ASTs) of $\varphi$ in $K$}
	\uIf{$\varphi$ is $x \cln s$}{Return $\mathit{variable(x, sort(s))}$}
	\uElseIf{$\varphi$ is $x \cln \cln s$}{Return $x \cln \KPattern {\ \wedge\ } (\mathit{sort(s)} = \KgetSort(x \cln \KPattern)) $}
	\uElseIf{$\varphi$ is $\varphi_1 \wedge \varphi_2$}
	{Return $\Kand(\mathit{lift}[\varphi_1], \mathit{lift}[\varphi_2]$}
	\uElseIf{$\varphi$ is $\neg \varphi_1$}
	{Return $\Knot(\mathit{lift}[\varphi_1])$}
	\uElseIf{$\varphi$ is $\exists x \cln s . \varphi_1$}
	{Return $\Kexists(x, \mathit{sort(s)}, \mathit{lift}[\varphi_1])$}
	\uElseIf{$\varphi$ is $\sigma(\varphi_1,\dots,\varphi_n)$ and $\sigma \in \Sigma_{s_1,\dots,s_n,s}$}
	{Return $\Kapplication(\mathit{symbol}(\sigma, (\Ksort(s_1), \dots, \Ksort(s_n)), \Ksort(s)),$
	$\mathit{lift}[\varphi_1],\dots,\mathit{lift}[\varphi_n])$}
	\caption{Lifting Function $\mathit{lift}[\_]$}
	\label{alg:liftingfunction}
\end{algorithm}

\begin{algorithm}
	\KwIn{An object-pattern $\varphi$}
	\KwOut{The meta-variable sort constraint of $\varphi$}
	Collect in set $W$ all meta-variables appearing in $\varphi$\;
	Let $C = \emptyset$\;
	\ForEach{$x \cln \cln s \in W$}
	{$C = C \cup (\mathit{sort(s)} = \KgetSort(x \cln \KPattern))$}
	Return $\bigwedge C$\;
	\caption{Meta-Variable Sort Constraint Collection $\mathit{mvsc}$}
	\label{alg:mvsc}
\end{algorithm}


\subsection{Examples of Kore}
\label{sec:examples-of-kore}

\todo[inline, author=Xiaohong]{Add more examples and texts here.}

\paragraph{The {\small BOOL} module.}\quad
\begin{Verbatim}[fontsize=\small]
module BOOL
  syntax Bool
  syntax Bool ::= true | false | notBool(Bool)
                | andBool(Bool, Bool) | orBool(Bool, Bool)
  axiom \or(true(), false())
  axiom \exists(X:Bool, \equals(X:Bool, true()))
  axiom \equals(andBool(B1::Bool, B2::Bool), 
                andBool(B2::Bool, B1::Bool))
  axiom ... ...
endmodule
\end{Verbatim}

\paragraph{The {\small BOOL} module (desugared).}\quad
\begin{Verbatim}[fontsize=\small]
module BOOL
  axiom \equals(
    #true,
    #deducible(#or(#application(#symbol("true", #nilSort, #sort("Bool")),
                                #nilPattern), 
                   #application(#symbol("false", #nilSort, #sort("Bool")),
                                #nilPattern))))
  axiom \equals(
    #true,
    #deducible(#exists("X", #sort("Bool"), 
               #equals(#variable("X", #sort("Bool")), 
                       #application(#symbol("true", #nilSort, #sort("Bool")),
                                    #nilPattern)))))
  axiom \implies(
    \and(\equals(#getSort(B1:Pattern), #sort("Bool")), 
         \equals(#getSort(B2:Pattern), #sort("Bool"))),
    \equals(
      #true,
      #deducible(#equals(#application(#symbol("andBool", 
                                              (#sort("Bool"), #sort("Bool"))
                                              #sort("Bool")), 
                                      (B1:Patern, B2:Pattern)), ---- TODO
                         #application(#symbol("andBool", 
                                              (#sort("Bool"), #sort("Bool"))
                                              #sort("Bool")), 
                                      (B2:Patern, B1:Pattern))))))
  axiom ... ...
endmodule
\end{Verbatim}

\paragraph{The {\small LAMBDA} module}\quad
\begin{Verbatim}[fontsize=\small]
module LAMBDA
  syntax Exp
  syntax Exp ::= app(Exp, Exp) | lambda0(Exp, Exp)
  syntax #Bool ::= isLTerm(#Pattern)

  axiom \equals(
    isLTerm(#variable(X:String, #sort("Exp"))), 
    true)
  axiom \equals(
    isLTerm(#application(
              #symbol("app", (#sort("Exp"), #sort("Exp")), #sort("Exp")),
              (E:Pattern, E':Pattern))),
    andBool(isLTerm(E:Pattern), isLTerm(E':Pattern)))
  axiom \equals(
    isLTerm(#exists(X:String, #sort("Exp"),
                    #application(#symbol("lambda0",
                                         (#sort("Exp"), #sort("Exp")),
                                         #sort("Exp")),
                                 (#variable(X:String, #sort("Exp")),
                                  E:Pattern))),
    isLTerm(E:Pattern))
  axiom \implies(\equals(true, 
                         andBool(isLTerm(E:Pattern), 
                                 isLTerm(E':Pattern))),
                 \equals(true,
                         deducible(#equals(...1,
                                           ...2)))) 
endmodule
\end{Verbatim}


\end{document}
